%%%%%% -*- Coding: utf-8-unix; Mode: latex

\documentclass[polish]{aghengthesis}

\usepackage{polski}
\usepackage[utf8]{inputenc}
\usepackage{url}
\usepackage{subfigure}
\usepackage{ragged2e}
\usepackage{multirow}
\usepackage{grffile}
\usepackage{indentfirst}
\usepackage{caption}
\usepackage{listings}
\usepackage[ruled,linesnumbered,lined]{algorithm2e}
\usepackage[bookmarks=false]{hyperref}

\usepackage{tabularx}
\usepackage{booktabs}
\usepackage{array}
\usepackage[table]{xcolor}

\hypersetup{colorlinks,
    linkcolor=blue,
    citecolor=blue,
    urlcolor=blue}

\usepackage[svgnames]{xcolor}
\usepackage{inconsolata}

\usepackage{csquotes}
\DeclareQuoteStyle[quotes]{polish}
{\quotedblbase}
{\textquotedblright}
[0.05em]
{\quotesinglbase}
{\fixligatures\textquoteright}
\DeclareQuoteAlias[quotes]{polish}{polish}

\usepackage[nottoc]{tocbibind}

\usepackage[
    style=numeric,
    sorting=none,
    isbn=false,
    doi=true,
    url=true,
    backref=false,
    backrefstyle=none,
    maxnames=10,
    giveninits=true,
    abbreviate=true,
    defernumbers=false,
    backend=biber]{biblatex}
\addbibresource{bibliografia.bib}
\bibliographystyle{plain}

\lstdefinelanguage{terminal}{
    breaklines=true,
    breakatwhitespace=false,
}

\lstdefinelanguage{Scala}{
    morekeywords={
        abstract,case,catch,class,def,do,else,
        enum,export,extends,false,final,finally,for,
        given,if,implicit,import,lazy,match,new,
        null,object,override,package,private,protected,return,
        sealed,super,then,throw,trait,true,try,
        type,val,var,while,with,yield,
        as,derives,end,extension,infix,inline,opaque,open,transparent,using},
    otherkeywords={<-,=>,<:,>:,@,=>>,?=>,|,*},
    sensitive=true,
    morecomment=[l]{//},
    morecomment=[n]{/*}{*/},
    morecomment=[n]{/**}{*/},
    morestring=[b]",
    morestring=[b]"""
}[keywords,comments,strings]

\usepackage{fontspec}
\setmonofont{JetBrains Mono}[Contextuals=Alternate]

\lstset{
    basicstyle=\ttfamily\footnotesize,
    backgroundcolor=\color{white},
    commentstyle=\it\color{Green},
    keywordstyle=\color{Red},
    stringstyle=\color{Blue},
    numberstyle=\tiny\color{Black},
    escapeinside=`',
    frame=single,
    tabsize=2,
    rulecolor=\color{black!30},
    title=\lstname,
    breaklines=true,
    breakatwhitespace=true,
    framextopmargin=2pt,
    framexbottommargin=2pt,
    extendedchars=false,
    captionpos=b,
    abovecaptionskip=5pt,
    keepspaces=true,
    numbers=left,
    numbersep=5pt,
    showspaces=false,
    showstringspaces=false,
    showtabs=false,
    tabsize=2
}

\SetAlgorithmName{\LangAlgorithm}{\LangAlgorithmRef}{\LangListOfAlgorithms}
\newcommand{\listofalgorithmes}{\tocfile{\listalgorithmcfname}{loa}}

\renewcommand{\lstlistingname}{\LangListing}
\renewcommand\lstlistlistingname{\LangListOfListings}

\renewcommand{\lstlistoflistings}{\begingroup
\tocfile{\lstlistlistingname}{lol}
\endgroup}

% Definicje nowych rodzajów kolumn w tabeli
\newcolumntype{C}{>{\centering\arraybackslash}m{0.15\linewidth}}
\newcolumntype{L}{>{\centering\arraybackslash}m{0.15\linewidth}}

\captionsetup[figure]{skip=5pt,position=bottom}
\captionsetup[table]{skip=5pt,position=top}

%%%%%%%%%%%%%%%%%%%%%%%%%%%%%%%%%%%%%%%%%%%%%%%%%%%%%%%%%%%%%%%%%%%%%%%%%%%%%%%
\author{Bartosz Buczek, Bartłomiej Kozak}

\titlePL{Implementacja narzędzi lex i yacc z wykorzystaniem metaprogramowania}
\titleEN{Implementation of lexical analyzer (lex) and parser generator (yacc) tools using metaprogramming techniques}

\fieldofstudy{Informatyka}

%\typeofstudies{Stacjonarne}

\supervisor{dr inż.\ Tomasz Służalec}

\date{\the\year}

%%%%%%%%%%%%%%%%%%%%%%%%%%%%%%%%%%%%%%%%%%%%%%%%%%%%%%%%%%%%%%%%%%%%%%%%%%%%%%%
\begin{document}

    \maketitle

    \tableofcontents

    \chapter{Cel prac i wizja projektu}
\label{ch:cel-wizja}


\section{Charakterystyka problemu}
\label{sec:charakterystyka-problemu}
Leksery i parsery są kluczowymi elementami w procesie tworzenia interpreterów i kompilatorów języków programowania.
Pozwalają one przekształcić kod źródłowy napisany przez programistę na reprezentację wewnętrzną, wykorzystywaną później przez dalsze etapy przetwarzania kodu.

Analiza leksykalna wykonywana przez lekser polega na rozdzieleniu kodu źródłowego na jednostki logiczne, zwane leksemami.
Parser natomiast wykonuje analizę składniową w celu ustalenia struktury gramatycznej tekstu i jej zgodności z gramatyką języka.

Celem pracy inżynierskiej jest stworzenie narzędzia \textit{ALPACA} (Another Lexer Parser And Compiler Alpaca) w języku Scala, które implementuje funkcjonalności powszechnie stosowane w budowie lekserów i parserów.


\section{Motywacja projektu}
\label{sec:motywacja-projektu}

Projekt ma na celu stworzenie nowoczesnego narzędzia do generowania lekserów i parserów w języku Scala, łączącego zalety istniejących rozwiązań z nowoczesnym podejściem technologicznym.
Jego główne cele to:
\begin{enumerate}
    \item Stworzenie intuicyjnego API\@.
    \item Opracowanie obszernej dokumentacji.
    \item Rozbudowana diagnostyka błędów.
    \item Poprawa wydajności względem rozwiązań w języku Python.
    \item Integracja z popularnymi środowiskami programistycznymi (IDE).
\end{enumerate}

Proponowane rozwiązanie łączy nowoczesne podejście technologiczne z praktycznym zastosowaniem w edukacji i programowaniu.
Może on służyć jako narzędzie dydaktyczne, ułatwiając naukę teorii kompilacji, w pracach badawczych, a także jako kompleksowe narzędzie do tworzenia praktycznych rozwiązań.


\section{Przegląd istniejących rozwiązań}
\label{sec:przeglad-istniejacych-rozwiazan}

Dostępne na rynku rozwiązania umożliwiają tworzenie analizatorów, jednak charakteryzują się ograniczeniami związanymi z wydajnością, wysokim progiem wejścia i diagnostyką błędów.

\subsection{Lex, Yacc}
\label{subsec:lex-yacc}

\textit{Lex}\cite{lesk1975lex} i \textit{Yacc}\cite{johnson1975yacc} to klasyczne, dobrze ugruntowane narzędzia, które odegrały kluczową rolę w tworzeniu setek współczesnych języków programowania.
Definicja leksera i parsera w tych systemach odbywa się poprzez specjalnie zaprojektowaną składnię konfiguracyjną.
Mimo pewnych zalet, jego złożoność i wysoki próg wejścia mogą stanowić wyzwanie.

Ponieważ \textit{Lex} i \textit{Yacc} zostały zaprojektowane do współpracy z językiem C, ich integracja z nowoczesnymi językami programowania bywa utrudniona.
Rozszerzanie tych narzędzi o dodatkowe, specyficzne funkcjonalności jest skomplikowane, co ogranicza ich elastyczność.
Brak wsparcia dla współczesnych środowisk programistycznych (IDE) dodatkowo obniża komfort użytkowania w porównaniu z nowoczesnymi alternatywami.
\lstinputlisting[language=c,caption={Fragment definicji parsera Ruby w technologii Yacc},label={lst:ruby-parser}]{listings/chapter1/parse.y}

\subsection{PLY, SLY}
\label{subsec:ply-sly}

\textit{PLY}\cite{ply} i jego nowszy odpowiednik \textit{SLY}\cite{sly} to biblioteki inspirowane narzędziami Lex i Yacc.
Oferują elastyczne podejście do budowy parserów, umożliwiając samodzielną implementację obsługi leksemów, budowę drzewa AST, czy dodatkowe funkcjonalności takie jak obliczanie numeru linii w lekserze.

Głównym ograniczeniem PLY i SLY jest implementacja w języku Python.
Ze względu na interpretowany charakter oraz dynamiczne typowanie, parsery te charakteryzują się niską wydajnością, a brak statycznego typowania utrudnia wykrywanie błędów na etapie kompilacji.
Przy implementacji parserów z użyciem biblioteki SLY w środowisku PyCharm obserwuje się wiele ostrzeżeń dotyczących potencjalnych naruszeń reguł, co często wymaga zastosowania mechanizmów supresji, aby uniknąć fałszywie pozytywnych wyników analizy statycznej kodu.
Ponadto należy zaznaczyć, iż autor projektu informuje o braku dalszego rozwoju tych narzędzi\cite{sly-github}.

Przykład \ref{lst:python-parser} ilustruje kilka nieintuicyjnych, automatycznych mechanizmów obecnych w bibliotece \textit{SLY}.
\begin{itemize}
    \item Operator \verb|@_()| jest zdefiniowany, aby automatycznie analizować tekst przy pomocy wyrażeń regularnych.
    Literały muszą być zawarte w cudzysłowie, a „zmienna” odpowiada za matchowany „typ”.
    \item Nazwa metody oznacza „typ” zwracany przez daną produkcję, czyli dla definicji \verb|IF| należy najpierw odszukać wszystkie metody, które mają nazwę \verb|condition|, gdyż są to możliwe produkcje.
    \item W krotce (sic!) \verb|precedence| definiujemy pierwszeństwo operatorów, jednakże dodanie \verb|% prec| pozwala nadpisać priorytet dla konkretnej reguły składniowej.
    \item  Argument \verb|p| pozwala na dostęp do kontekstu produkcji (np. numeru linii), ale także do zmiennych w patternu match w adnotacji.
    Jeśli zdefiniowany jest więcej niż jeden, to dodajemy numer do accesora, np. \verb|expr1| jest odwołaniem się do drugiego wyrażenie \verb|expr|.
    Jednocześnie, można to zrobić także poprzez odwołanie się do konkretnego indeksu obiektu \verb|p|.
\end{itemize}

\lstinputlisting[language=Python,caption={Fragment definicji parsera w Pythonie, wykorzystujacy bibliotekę SLY},label={lst:python-parser}]{listings/chapter1/sly-parser.py}

Komunikaty błędów w bibliotece \textit{SLY} są bardzo ograniczone, co obrazuje przykład \ref{lst:not-working-sly}, który po uruchomieniu informuje użytkownika błędem z fragmentu kodu \ref{lst:lstlisting}.
Okazuje się, że problemem był brak atrybutu \verb|ignore_comment| w definicji \verb|Lexer|.
\lstinputlisting[language=Python,caption={Fragment niedziałajacego kodu w Pythonie, wykorzystujacy bibliotekę SLY},label={lst:not-working-sly}]{listings/chapter1/not-working-sly.py}

\lstinputlisting[language=terminal, caption={Przykładowy komunikat błędu w bibliotece \textit{SLY}},label={lst:sly-result}]{listings/chapter1/not-working-result}

\subsection{ANTLR}
\label{subsec:antlr}

\textit{ANTLR}\cite{parr2004s} to kolejne rozwiązanie inspirowane narzędziami \textit{Lex} i \textit{Yacc}, oferujące zaawansowane mechanizmy analizy składniowej.
Jego twórcy opracowali dedykowany język DSL, znany jako Grammar v4, który umożliwia definiowanie składni analizowanego języka.
Na podstawie tej definicji \textit{ANTLR} generuje parser w wybranym przez użytkownika języku programowania, takim jak Python, Java, C++ lub JavaScript.

Wspomaganie pracy z \textit{ANTLR} w znacznym stopniu ułatwiają dedykowane wtyczki do środowisk Visual Studio Code oraz IntelliJ IDEA. Oferują one funkcjonalności, takie jak kolorowanie składni, autouzupełnianie kodu, nawigację do definicji leksemów oraz walidację błędów, co znacząco przyspiesza proces tworzenia parserów.

Jedną z kluczowych różnic \textit{ANTLR} w porównaniu do innych narzędzi jest wykorzystanie gramatyki LL(*), podczas gdy klasyczne rozwiązania, takie jak Yacc czy SLY, implementują LALR(1).
LL(*) jest bardziej intuicyjna i czytelna dla programistów, co ułatwia definiowanie reguł składniowych.
Jednakże, jej zastosowanie wiąże się z większym zużyciem pamięci oraz niższą wydajnością w porównaniu do LALR(1).

Dodatkowym wyzwaniem podczas korzystania z \textit{ANTLR} jest konieczność nauki składni DSL Grammar v4 oraz ograniczenie wsparcia dla narzędzi deweloperskich.
Pełne wykorzystanie możliwości \textit{ANTLR} wymaga korzystania z jednego z dedykowanych środowisk, co może stanowić istotne ograniczenie dla użytkowników preferujących inne IDE\@.

\subsection{Scala parser combinators}
\label{subsec:scala-parser-combinators}

Biblioteka \textit{Scala parser combinators}\cite{moors2008parser} była popularnym sposobem na tworzenie parserów, lecz jak wynika z dokumentacji, „Trudno jest jednak zrozumieć ich działanie i jak zacząć.
Po skompilowaniu i uruchomieniu kilku pierwszych przykładów, mechanizm działania staje się bardziej zrozumiały, ale do tego czasu może to być zniechęcające, a standardowa dokumentacja nie jest zbyt pomocna”\cite{parser-combinators-readme}.

\subsection{ScalaBison}
\label{subsec:scala-bison}

Z podsumowania artykułu na temat \textit{ScalaBison}\cite{boyland2010tool} wiadomo, że to praktyczny generator parserów dla języka Scala oparty na technologii rekurencyjnego wstępowania i zstępowania, który akceptuje pliki wejściowe w formacie \textit{bison}.
Parsery generowane przez \textit{ScalaBison} używają bardziej informacyjnych komunikatów o błędach niż te generowane przez pierwowzór \textit{bison}, a także szybkość parsowania i wykorzystanie miejsca są znacznie lepsze niż \textit{scala-combinators}, ale są nieco wolniejsze niż najszybsze generatory parserów oparte na JVM.

Dodatkowo należy zaznaczyć, iż jest to rozwiązanie już niewspierane i stworzone w celach akademickich.
Korzysta z przestarzałej wersji Scali, nie posiada wyczerpującej dokumentacji i liczba funkcjonalności jest bardzo ograniczona w porównaniu do np. technologii \textit{SLY}.

\subsection{parboiled2}
\label{subsec:parboiled-2}

\textit{parboiled2}\cite{myltsev2019parboiled2} to biblioteka w Scali umożliwiająca lekkie i szybkie parsowanie dowolnego tekstu wejściowego.
Implementuje ona oparty na makrach generator parsera dla gramatyk wyrażeń parsujących (PEG), który działa w czasie kompilacji i tłumaczy definicję reguły gramatycznej na odpowiadający jej bytecode JVM. Niestety próg wejścia ze względu na skomplikowany i nieintuicyjny DSL jest wysoki.
Zgodnie z przykładem \ref{lst:parboiled2-error}, raportowanie błędów jest bardzo ograniczone (problem z implementacją wynika jedynie z różnic w liczbie parametrów funkcji).

\lstinputlisting[firstline=7, lastline=20,language=terminal,caption={[Fragment błędu wygenerowanego przez bibliotekę parboiled2]
Niewielki fragment (14 z 133 linii) błędu wygenerowanego przez bibliotekę \textit{parboiled2}, który pochodzi z prezentacji Li Haoyi na temat \textit{FastParse}\cite{fastparse-talk}.
},label={lst:parboiled2-error}]{listings/chapter1/parboiled2-error}

\subsection{FastParse}
\label{subsec:fastparse}

FastParse\textit{FastParse}\cite{fastparse-docs} to opracowana przez Li Haoyi, wysokowydajna biblioteka kombinatorów parserów dla Scali, zaprojektowana w celu uproszczenia tworzenia parserów tekstu strukturalnego.
Umożliwia ona programistom definiowanie parserów rekurencyjnych, dzięki czemu nadaje się do parsowania języków programowania, formatów danych, takich jak JSON, czy DSL-i.
Cechą charakterystyczną FastParse jest równowaga między użytecznością a wydajnością.
Parsery są konstruowane poprzez łączenie mniejszych parserów za pomocą operatorów, takich jak \verb&~& dla sekwencjonowania i \verb&|& dla alternatyw, przy jednoczesnym zachowaniu czytelności zbliżonej do formalnych definicji gramatyki.
Według dokumentacji\cite{fastparse-docs}, parsery \textit{Fastparse} zajmują 1/10 kodu w porównaniu do ręcznie napisanego parsera rekurencyjnego.
W porównaniu do narzędzi generujących parsery, takich jak \textit{ANTLR} lub \textit{Lex} i \textit{Yacc}, implementacja nie wymaga żadnego specjalnego kroku kompilacji lub generowania kodu.
To sprawia, że rozpoczęcie pracy z \textit{Fastparse} jest znacznie łatwiejsze niż w przypadku bardziej tradycyjnych narzędzi do generowania parserów.
Przykładowo, parser wyrażeń arytmetycznych może być zwięźle napisany, aby obsługiwać zagnieżdżone nawiasy, pierwszeństwo operatorów i raportowanie błędów w mniej niż 20 liniach kodu\cite{fastparse-slides}.
Biblioteka kładzie również nacisk na debugowanie, generując szczegółowe komunikaty o błędach, które wskazują dokładną lokalizację i przyczynę niepowodzeń parsowania, takich jak niedopasowane nawiasy lub nieprawidłowe tokeny.

\subsection{Podsumowanie}
\label{subsec:podsumowanie}

\begin{table}[ht]
    \centering
    \begin{tabular}{L|CCCC}
        \toprule
        \large{Narzędzie}       & \textbf{Lex\&Yacc}                        & \textbf{PLY/SLY}     & \textbf{ANTLR}     & \textbf{scala-bison} \\
        \midrule
        Język implementacji     & C                                         & Python               & Java               & Scala (nad Bisonem)  \\
        \arrayrulecolor{gray}
        \hline
        Język użycia            & regex, BNF, akcje w C                     & DSL                  & DSL oparty na EBNF & BNF, akcje w Scali   \\
        \hline
        Wydajność               & wysoka                                    & niska                & umiarkowana        & wysoka               \\
        \hline
        Łatwość użycia          & średnia                                   & umiarkowana          & wysoka             & średnia              \\
        \hline
        Aktywne wsparcie        & brak                                      & nie                  & tak                & nie                  \\
        \hline
        Diagnostyka błędów      & słaba                                     & średnia              & dobra              & słaba                \\
        \hline
        Dokumentacja            & dobra                                     & średnia, nieaktualna & dobra              & słaba                \\
        \hline
        Popularność             & wysoka                                    & średnia              & wysoka             & niska                \\
        \hline
        Integracja IDE          & nieoficjalny plugin                       & ograniczona          & oficjalny plugin   & brak                 \\
        \hline
        Wsparcie do debugowania & brak                                      & dobre                & częściowe          & dobre                \\
        \hline
        Generowania kodu        & nie                                       & nie                  & tak                & nie                  \\
        \hline
        \toprule
        Narzędzie               & \textbf{Scala parser\newline combinators} & \textbf{parboiled2}  & \textbf{FastParse} & \textbf{ALPACA} \\
        \midrule
        Język implementacji     & Scala                                     & Scala                & Scala              & Scala                \\
        \hline
        Język użycia            & DSL w Scali                               & DSL w Scali          & DSL w Scali        & Scala                \\
        \hline
        Wydajność               & wysoka                                    & umiarkowana          & wysoka             & TODO                 \\
        \hline
        Łatwość użycia          & niska                                     & średnia              & średnia            & TODO                 \\
        \hline
        Aktywne wsparcie        & nie                                       & nie                  & tak                & TODO                 \\
        \hline
        Diagnostyka błędów      & dobra                                     & niska                & dobra              & TODO                 \\
        \hline
        Dokumentacja            & słaba                                     & bardzo dobra         & bardzo dobra       & TODO                 \\
        \hline
        Popularność             & średnia                                   & niska                & rosnąca            & TODO                 \\
        \hline
        Integracja IDE          & wsparcie dla Scali                        & wsparcie dla Scali   & wsparcie dla Scali & TODO                 \\
        \hline
        Wsparcie do debugowania & dobre                                     & dobre                & dobre              & TODO                 \\
        \hline
        Generowania kodu        & nie                                       & nie                  & nie                & TODO                 \\
        \bottomrule
    \end{tabular}
    \caption{Porównanie wybranych narzędzi do generowania lekserów i parserów}
    \label{tab:porownanie-alternatyw}
\end{table}


    \chapter{Metaprogramowanie w Scali 3}\label{ch:metaprogramowanie-w-scali-3}


\section{Wprowadzenie}\label{sec:wprowadzenie}
Scala 3, znana również jako Dotty, wprowadza całkowicie przeprojektowany system metaprogramowania, stanowiący fundamentalną zmianę w stosunku do eksperymentalnych makr dostępnych w Scali~2\cite{scala3-dropped-scala2-macros,scala3-metaprogramming}.

Metaprogramowanie w Scali 3 zostało zaprojektowane z naciskiem na bezpieczeństwo typów, przenośność oraz skalowalność, oferując programistom możliwość generowania i analizowania kodu w czasie kompilacji przy zachowaniu pełnej ekspresywności języka\cite{stucki2024infoscience,stucki2020inlining}.
W przeciwieństwie do poprzedniego systemu, który eksponował wewnętrzne mechanizmy kompilatora i był źródłem problemów z kompatybilnością między wersjami\cite{stucki2020thesis}, nowy system metaprogramowania jest zaprojektowany jako stabilny i przenośny interfejs programistyczny.
Podstawą teoretyczną systemu metaprogramowania w Scali 3 jest programowanie wieloetapowe (multi-stage programming), paradygmat pozwalający na odróżnienie różnych etapów wykonania programu\cite{scala3-staging,stucki2020thesis}.
W tym modelu kod może być wykonywany w różnych fazach: w czasie kompilacji (compile-time), w czasie wykonania (runtime)\cite{scala3-staging}.

\subsection{Quotes i splices}\label{subsec:cytaty-i-wstawki}
Kluczowymi koncepcjami w systemie metaprogramowania Scali 3 są quotes i splices\cite{stucki2018unification,stucki2021multistage}.
Quotes, oznaczane jako \verb|'{...}|, służą do opóźnienia wykonania kodu i traktowania go jako danych\cite{scala3-reflection,epfl-dotty-reflection}.
Splices, oznaczane jako \verb|${...}|, pozwalają na ocenę wyrażenia generującego kod i wstawienie wyniku do otaczającego kontekstu\cite{scala3-reflection,epfl-dotty-reflection,scala3-guides-quotes}.

Formalna semantyka tych konstrukcji została przedstawiona w pracy Stuckiego, Brachthäusera i Odersky'ego\cite{stucki2021multistage}, gdzie quotes i splices są traktowane jako prymitywne formy w typowanych drzewach składni abstrakcyjnej (typed abstract syntax trees).
Autorzy dowodzą, że system zachowuje bezpieczeństwo typów oraz higieniczność, zapewniając, że wygenerowany kod nie może przypadkowo powiązać identyfikatorów z niewłaściwymi zmiennymi\cite{stucki2021multistage}.

\subsection{Bezpieczeństwo międzyetapowe}\label{subsec:bezpieczenstwo-miedzyetapowe}
Scala 3 gwarantuje bezpieczeństwo międzyetapowe (cross-stage safety) poprzez sprawdzanie poziomów etapowania w czasie kompilacji\cite{stucki2020thesis,stucki2021multistage}.
Zmienne lokalne mogą być używane tylko na tym samym poziomie etapowania, na którym zostały zdefiniowane, co zapobiega dostępowi do zmiennych, które jeszcze nie istnieją lub już nie są dostępne\cite{stucki2020thesis}.

System również zapewnia, że typy generyczne używane w wyższym poziomie etapowania niż ich definicja wymagają instancji klasy typu \verb|Type[T]|, która niesie reprezentację typu niepoddaną wymazywaniu (type erasure)\cite{stucki2020thesis}.
To podejście rozwiązuje problem wymazywania typów generycznych w~JVM, zachowując informację o typach potrzebną w kolejnych etapach kompilacji.


\section{Mechanizmy metaprogramowania w Scali 3}\label{sec:mechanizmy-metaprogramowania-w-scali-3}

\subsection{Definicje inline}\label{subsec:definicje-inline}
Najprostszym narzędziem metaprogramowania jest modyfikator \verb|inline|.
Gwarantuje on, że wywołanie oznaczonej nim metody lub wartości zostanie w całości \textbf{wstawione w miejscu wywołania} (ang. \textit{inlining}) podczas kompilacji. Jest to polecenie dla kompilatora, a nie tylko sugestia, jak w niektórych innych językach

\subsection{Makra oparte na wyrażeniach}\label{subsec:makra-oparte-na-wyrazeniach}
Makra w Scali 3 są zdefiniowane jako metody \verb|inline| zawierające splice najwyższego poziomu (top-level splice)\cite{scala3-reference-macros,scala3-guides-macros}, czyli taki, który nie jest zagnieżdżony w żadnym quotes i jest wykonywany w czasie kompilacji\cite{scala3-staging,scala3-reference-macros}.

Typ \texttt{Expr[T]} reprezentuje wyrażenie Scali o typie \texttt{T} jako typowane drzewo składni abstrakcyjnej\cite{scala3-reflection,scala3-guides-macros}.
Makra manipulują wartościami typu \texttt{Expr[T]}, transformując je lub generując nowe wyrażenia\cite{scala3-guides-macros}.
Ta reprezentacja gwarantuje bezpieczeństwo typów na poziomie języka metaprogramowania\cite{scala3-reflection}.

\subsection{Refleksja TASTy}\label{subsec:refleksja-tasty}
Dla przypadków wymagających głębszej analizy kodu, Scala 3 oferuje API refleksji TASTy (Typed Abstract Syntax Tree)\cite{scala3-reflection,epfl-dotty-reflection,scala3-guides-reflection}.
TASTy jest binarnym formatem serializacji typowanych drzew składni abstrakcyjnej używanym przez kompilator Scali 3\cite{stucki2020thesis}.

API refleksji dostarcza szczegółowy widok na strukturę kodu, włączając typy, symbole oraz pozycje w kodzie źródłowym\cite{scala3-reflection,scala3-guides-reflection}.
Jest dostępne poprzez obiekt \texttt{reflect} zdefiniowany w typie \texttt{Quotes}, który jest przekazywany kontekstualnie do makr\cite{scala3-reflection,scala3-guides-reflection}.


\section{Implementacja systemu metaprogramowania}\label{sec:implementacja-systemu-metaprogramowania}

\subsection{Architektura kompilatora}\label{subsec:architektura-kompilatora}
Implementacja systemu metaprogramowania w Scali 3 jest zorganizowana wokół kilku kluczowych komponentów\cite{stucki2020thesis}.
Proces kompilacji obejmuje fazę inliningu (\textit{Inlining} phase), która rozwija definicje \verb|inline| oraz wywołania makr\cite{dotty-compiler-phases}.

Faza \textit{PostInlining} wykonuje czyszczenie po rozwinięciu definicji inline, usuwając pomocnicze struktury i optymalizując kod\cite{dotty-compiler-phases}.
Faza \textit{Staging} zajmuje się sprawdzaniem poziomów etapowania oraz adaptacją typów etapowanych poprzez proces zwany ``type healing''\cite{stucki2020thesis,dotty-compiler-phases}.

\subsection{Dopasowanie wzorców w cytatach}\label{subsec:dopasowanie-wzorcow-w-cytatach}
Scala 3 wspiera analizę kodu poprzez dopasowanie wzorców w quotes (\textit{quote pattern matching})\cite{stucki2020thesis,stucki2021multistage}.
Mechanizm ten pozwala na dekonstrukcję kawałków kodu i ekstrakcję podwyrażeń\cite{stucki2021multistage}.

Stucki, Brachthäuser i Odersky\cite{stucki2021multistage} wprowadzają wzorce wiążące (\textit{bind patterns}) postaci \verb|$x| oraz wzorce HOAS (Higher-Order Abstract Syntax) postaci \verb|$f(y)|, które pozwalają na ekstrakcję podwyrażeń potencjalnie zawierających zmienne z zewnętrznego kontekstu\cite{stucki2021multistage}.
System gwarantuje, że ekstrahowane wyrażenia są zamknięte względem definicji wewnątrz wzorca, zapobiegając wyciekom zakresu\cite{stucki2021multistage}.




    \chapter{Implementacja}
\label{ch:implementacja}


\section{Praktyczna implementacja analizatora leksykalnego z wykorzystaniem makr w Scali 3}\label{sec:praktyczna-implementacja-analizatora-leksykalnego-z-wykorzystaniem-makr-w-scali-3}

\subsection{Wprowadzenie do studium przypadku}\label{subsec:wprowadzenie-do-studium-przypadku}

Niniejszy rozdział prezentuje praktyczną implementację systemu analizy leksykalnej (leksera) wykorzystującego zaawansowane mechanizmy metaprogramowania Scali 3.
Przedstawiony kod stanowi przykład zastosowania technik opisanych w poprzednich rozdziałach do rozwiązania rzeczywistego problemu inżynierskiego: automatycznej generacji wydajnego analizatora leksykalnego z definicji wysokopoziomowej w formie języka dziedzinowego (DSL).

System \texttt{alpaca.lexer} implementuje transformację deklaratywnych reguł tokenizacji zapisanych jako funkcja częściowa (ang. \textit{partial function}) w kod procedualny wykonywany w czasie kompilacji.
Wykorzystuje przy tym pełne spektrum możliwości refleksji TASTy, włączając generację klas w czasie kompilacji, transformację drzew AST oraz wyspecjalizowane typy refinement.

\subsection{Architektura systemu leksera}\label{subsec:architektura-systemu-leksera}

\subsubsection{Interfejs użytkownika}\label{subsubsec:interfejs-uzytkownika}

System oferuje użytkownikowi przejrzysty interfejs DSL oparty na dopasowaniu wzorców:

\lstinputlisting[language=scala,caption={Definicja typu LexerDefinition},label={lst:lexer-01-lexerdefinition}, linerange={21}]{../../src/alpaca/lexer/Lexer.scala}

Definicja \texttt{LexerDefinition} reprezentuje reguły leksera jako funkcję częściową mapującą wzorce wyrażeń regularnych (jako ciągi znaków) na definicje tokenów.
Wykorzystanie funkcji częściowej pozwala na naturalne wyrażenie reguł leksykalnych w idiomatycznej składni Scali.

Główny punkt wejścia systemu stanowi metoda \texttt{lexer}:

\lstinputlisting[language=scala,caption={Punkt wejścia: transparent inline def lexer},label={lst:lexer-02-entrypoint}, linerange={44-52}]{../../src/alpaca/lexer/Lexer.scala}

Modyfikator \texttt{transparent inline} zapewnia, że zwracany typ będzie dokładnie odpowiadał wygenerowanej strukturze, włączając typy refinement dla poszczególnych tokenów.
Użycie parametrów kontekstowych (\texttt{using}) realizuje wzorzec dependency injection na poziomie systemu typów.

\subsubsection{Implementacja makra}\label{subsubsec:implementacja-makra}

Makro przyjmuje wyrażenie reprezentujące reguły leksera jako \texttt{Expr[Ctx ?=> LexerDefinition[Ctx]]} oraz instancje kontekstualnych klas pomocniczych.
Parametr \texttt{using Quotes} dostarcza dostępu do API refleksji TASTy.

\subsection{Analiza drzewa składni abstrakcyjnej}\label{subsec:analiza-drzewa-skadni-abstrakcyjnej}

\subsubsection{Dekonstrukcja funkcji częściowej}\label{subsubsec:dekonstrukcja-funkcji-czesciowej}

Kluczowym krokiem implementacji jest ekstrakcja reguł z definicji funkcji częściowej:

\lstinputlisting[language=scala,caption={Dekonstrukcja funkcji częściowej (dopasowanie AST do CaseDef)},label={lst:lexer-04-extract-cases}, linerange={72}]{../../src/alpaca/lexer/Lexer.scala}

Ten fragment kodu wykorzystuje dopasowanie wzorców w \textit{quotes} do dekonstrukcji typowanego AST funkcji częściowej.
Struktura \texttt{Lambda(\_, Match(\_, cases))} odpowiada wewnętrznej reprezentacji funkcji częściowej, gdzie \texttt{Match} zawiera listę przypadków \texttt{CaseDef}.

\subsection{Transformacja i adaptacja referencji}\label{subsec:transformacja-i-adaptacja-referencji}

\subsubsection{Klasa replacerefs}\label{subsubsec:klasa-replacerefs}

Kluczową techniką jest zastąpienie referencji do starego kontekstu nowymi referencjami:

\lstinputlisting[language=scala,caption={Zastąpienie referencji starego kontekstu nowymi (ReplaceRefs)},label={lst:lexer-06-replace-with-new-ctx}, linerange={75-78}]{../../src/alpaca/lexer/Lexer.scala}

Transformacja ta realizuje proces znany jako \("\)re-owning\("\) w terminologii kompilatorów — zmianę właściciela (owner) symboli w AST. Jest to konieczne, ponieważ kod oryginalnie odnoszący się do parametru makra musi zostać przepisany, aby odnosił się do parametru metody w wygenerowanej klasie.

Klasa \texttt{ReplaceRefs} udostępnia \texttt{TreeMap}, który podczas przejścia po AST podmienia referencje do wskazanych symboli na podane termy.
Umożliwia to tzw. re-owning — przeniesienie fragmentów kodu między różnymi właścicielami symboli bez ręcznego przepisywania drzew (por. \ref{subsubsec:core-replacerefs}).

\subsection{Ekstrakcja i kompilacja wzorców}\label{subsec:ekstrakcja-i-kompilacja-wzorcow}

\subsubsection{Funkcja extractSimple}\label{subsubsec:funkcja-extractsimple}

Funkcja \texttt{extractSimple} implementuje logikę dopasowania różnych typów definicji tokenów:

\lstinputlisting[language=scala,caption={Funkcja extractSimple: dopasowywanie definicji tokenów},label={lst:lexer-08-extract-simple}, linerange={80-85,99-108}]{../../src/alpaca/lexer/Lexer.scala}

Wykorzystuje ona dopasowanie wzorców w \textit{quotes} z ekstraktorem typów, umożliwiając rozróżnienie różnych wariantów definicji tokenów na poziomie typów.
Konstrukcja \texttt{type t <: ValidName} w wzorcu wiąże parametr typu do zmiennej wzorca \texttt{t}, umożliwiając jego późniejsze wykorzystanie.

\subsection{Analiza wzorców: klasa CompileNameAndPattern}
\label{subsec:compile-name-pattern}

Klasa \texttt{CompileNameAndPattern} stanowi kluczowy komponent systemu analizy leksykalnej, odpowiedzialny za ekstrakcję i walidację wzorców tokenów podczas ekspansji makra.
Jej głównym zadaniem jest transformacja różnorodnych form wzorców występujących w definicjach DSL na ujednolicone struktury \texttt{TokenInfo}, które następnie są wykorzystywane do generacji finalnego kodu leksera.

Implementacja wykorzystuje rekurencyjne przetwarzanie drzewa AST z zastosowaniem optymalizacji rekurencji ogonowej (\texttt{@tailrec}), co zapewnia efektywność działania nawet dla złożonych wzorców z wieloma alternatywami.

\subsection{Generacja klasy anonimowej}\label{subsec:generacja-klasy-anonimowej}

Anonimowa klasa implementująca \texttt{Tokenization[Ctx]} jest konstruowana programatycznie: (1) tworzymy symbol klasy przez \texttt{Symbol.newClass} wraz z listą deklaracji pól i typów; (2) budujemy ciało klasy (\texttt{ClassDef}) zawierające \texttt{ValDef} dla każdego zdefiniowanego tokena oraz pola \texttt{tokens} i \texttt{byName}; (3) określamy rodzica przez wywołanie konstruktora \texttt{Tokenization[Ctx]} z wymaganymi zależnościami; (4) instancjonujemy klasę i nadajemy jej typ zrafi­nowany przez kolejne \texttt{Refinement} odpowiadające polom-tokenom.

\lstinputlisting[language=scala,caption={Generacja i instancjowanie anonimowej klasy \texttt{Tokenization[Ctx]} z typem rafinowanym},label={lst:lexer-14-anon-class}, linerange={192-264}]{../../src/alpaca/lexer/Lexer.scala}

\subsubsection{Typy refinement}\label{subsubsec:typy-refinement}

Zwracany typ jest stopniowo rafinowany dla każdego tokena:

\lstinputlisting[language=scala,caption={Rafinowanie typu wynikowego o pola tokenów},label={lst:lexer-15-refinements}, linerange={253-258}]{../../src/alpaca/lexer/Lexer.scala}

\texttt{Refinement(tpe, name, memberType)} tworzy typ refinement dodający członka o podanej nazwie i typie do bazowego typu.
Pozwala to kompilatorowi śledzić, że zwrócony obiekt ma pola odpowiadające poszczególnym tokenom, umożliwiając dostęp do nich z pełnym wsparciem systemu typów.

\subsection{Walidacja i obsługa błędów}\label{subsec:walidacja-i-obsuga-bedow}

\subsubsection{Walidacja wzorców regularnych}\label{subsubsec:walidacja-wzorcow-regularnych}

System wykorzystuje pomocniczą klasę \texttt{RegexChecker} do walidacji wzorców:

Poniższy mechanizm sprawdza poprawność składni wyrażeń regularnych już w czasie kompilacji i raportuje błędy z dokładną lokalizacją wzorca.
Metoda \texttt{report.errorAndAbort} jest częścią API kompilatora do raportowania błędów w czasie kompilacji.
Przerwanie kompilacji w przypadku niepoprawnych wzorców zapewnia, że błędy konfiguracji są wykrywane możliwie wcześnie.

\subsubsection{Obsługa nieobsługiwanych konstrukcji}\label{subsubsec:obsuga-nieobsugiwanych-konstrukcji}

Kod jawnie sygnalizuje nieobsługiwane przypadki:
Obsługiwane są wyłącznie jasno zdefiniowane formy wzorców; w przypadku napotkania innej konstrukcji kompilacja jest przerywana z komunikatem zawierającym szczegóły AST, co upraszcza diagnostykę i utrzymuje zasadę fail-fast.
Ta strategia jest zgodna z zasadą fail-fast - lepiej jest wyraźnie odrzucić nieobsługiwane konstrukcje niż milcząco generować niepoprawny kod.


    \printbibliography
    \listoffigures
    \listoftables
    \listofalgorithmes
    \lstlistoflistings

\end{document}
