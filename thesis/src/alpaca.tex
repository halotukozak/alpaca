%%%%%% -*- Coding: utf-8-unix; Mode: latex

\documentclass[polish]{aghengthesis}

\usepackage{polski}
\usepackage[utf8]{inputenc}
\usepackage{url}
\usepackage{subfigure}
\usepackage{ragged2e}
\usepackage{multirow}
\usepackage{grffile}
\usepackage{indentfirst}
\usepackage{caption}
\usepackage{listings}
\usepackage[ruled,linesnumbered,lined]{algorithm2e}
\usepackage[bookmarks=false]{hyperref}

\usepackage{tabularx}
\usepackage{booktabs}
\usepackage{array}
\usepackage[table]{xcolor}

\hypersetup{colorlinks,
    linkcolor=blue,
    citecolor=blue,
    urlcolor=blue}

\usepackage[svgnames]{xcolor}
\usepackage{inconsolata}

\usepackage{csquotes}
\DeclareQuoteStyle[quotes]{polish}
{\quotedblbase}
{\textquotedblright}
[0.05em]
{\quotesinglbase}
{\fixligatures\textquoteright}
\DeclareQuoteAlias[quotes]{polish}{polish}

\usepackage[nottoc]{tocbibind}

\usepackage[
    style=numeric,
    sorting=none,
    isbn=false,
    doi=true,
    url=true,
    backref=false,
    backrefstyle=none,
    maxnames=10,
    giveninits=true,
    abbreviate=true,
    defernumbers=false,
    backend=biber]{biblatex}
\addbibresource{bibliografia.bib}

\lstdefinelanguage{terminal}{
    breaklines=true,
    breakatwhitespace=false,
}

\lstdefinelanguage{Scala}{
    morekeywords={
        abstract,case,catch,class,def,do,else,
        enum,export,extends,false,final,finally,for,
        given,if,implicit,import,lazy,match,new,
        null,object,override,package,private,protected,return,
        sealed,super,then,throw,trait,true,try,
        type,val,var,while,with,yield,
        as,derives,end,extension,infix,inline,opaque,open,transparent,using},
    otherkeywords={<-,=>,<:,>:,@,=>>,?=>,|,*},
    sensitive=true,
    morecomment=[l]{//},
    morecomment=[n]{/*}{*/},
    morecomment=[n]{/**}{*/},
    morestring=[b]",
    morestring=[b]``''"
}[keywords,comments,strings]

\lstdefinelanguage{json}{
    breaklines=true,
    breakatwhitespace=false,
}

\usepackage{fontspec}
\setmonofont{JetBrains Mono}[Contextuals=Alternate]

\lstset{
    basicstyle=\ttfamily\footnotesize,
    backgroundcolor=\color{white},
    commentstyle=\it\color{Green},
    keywordstyle=\color{Red},
    stringstyle=\color{Blue},
    numberstyle=\tiny\color{Black},
    escapeinside=`',
    frame=single,
    tabsize=2,
    rulecolor=\color{black!30},
    title=\lstname,
    breaklines=true,
    breakatwhitespace=true,
    framextopmargin=2pt,
    framexbottommargin=2pt,
    extendedchars=false,
    captionpos=b,
    abovecaptionskip=5pt,
    keepspaces=true,
    numbers=left,
    numbersep=5pt,
    showspaces=false,
    showstringspaces=false,
    showtabs=false,
    tabsize=2
}

\SetAlgorithmName{\LangAlgorithm}{\LangAlgorithmRef}{\LangListOfAlgorithms}
\newcommand{\listofalgorithmes}{\tocfile{\listalgorithmcfname}{loa}}

\renewcommand{\lstlistingname}{\LangListing}
\renewcommand\lstlistlistingname{\LangListOfListings}

\renewcommand{\lstlistoflistings}{\begingroup
\tocfile{\lstlistlistingname}{lol}
\endgroup}

% Definicje nowych rodzajów kolumn w tabeli
\newcolumntype{C}{>{\centering\arraybackslash}m{0.15\linewidth}}
\newcolumntype{L}{>{\centering\arraybackslash}m{0.15\linewidth}}

\captionsetup[figure]{skip=5pt,position=bottom}
\captionsetup[table]{skip=5pt,position=top}

%%%%%%%%%%%%%%%%%%%%%%%%%%%%%%%%%%%%%%%%%%%%%%%%%%%%%%%%%%%%%%%%%%%%%%%%%%%%%%%
\author{Bartosz Buczek, Bartłomiej Kozak}

\titlePL{Implementacja narzędzi lex i yacc z wykorzystaniem metaprogramowania}
\titleEN{Implementation of lexical analyzer (lex) and parser generator (yacc) tools using metaprogramming techniques}

\fieldofstudy{Informatyka}

%\typeofstudies{Stacjonarne}

\supervisor{dr inż.\ Tomasz Służalec}

\date{\the\year}

%%%%%%%%%%%%%%%%%%%%%%%%%%%%%%%%%%%%%%%%%%%%%%%%%%%%%%%%%%%%%%%%%%%%%%%%%%%%%%%
\begin{document}

\maketitle

\tableofcontents

\chapter[Cel pracy i wizja projektu]{Cel pracy i~wizja projektu}
\label{ch:cel-wizja}


\section{Charakterystyka problemu}
\label{sec:charakterystyka-problemu}

Analizatory leksykalne (ang.~\emph{lexers}) i~składniowe (ang.~\emph{parsers}) stanowią fundamentalne komponenty procesu kompilacji, realizując odpowiednio fazę analizy leksykalnej i~syntaktycznej~\cite{aho2006}.
Analizator leksykalny segmentuje ciąg znaków wejściowych na strumień tokenów (leksemów) zgodnie z~regułami języka regularnego~\cite{hopcroft2006}, podczas gdy analizator składniowy weryfikuje zgodność strumienia tokenów z~gramatyką bezkontekstową języka, konstruując drzewo składni abstrakcyjnej (AST, ang.~\emph{Abstract Syntax Tree})~\cite{aho2006}.

\subsection{Podstawy teoretyczne}
\label{subsec:podstawy-teoretyczne}

Analiza leksykalna i~składniowa opiera się na teorii języków formalnych, zapoczątkowanej przez prace Noama Chomsky'ego~\cite{chomsky1956}.
W~hierarchii Chomsky'ego języki dzieli się na cztery klasy według mocy wyrazu gramatyk je generujących.
Analizatory leksykalne operują na językach regularnych (typ~3), które są rozpoznawane przez automaty skończone~\cite{hopcroft2006}, podczas gdy parsery składniowe obsługują języki bezkontekstowe (typ~2), rozpoznawane przez automaty ze stosem~\cite{aho2006}.

Wyrażenia regularne są notacją deklaratywną dla języków regularnych i~można je mechanicznie przekształcić w~automaty skończone za pomocą konstrukcji Thompsona~\cite{thompson1968}.
Automaty deterministyczne (DFA) gwarantują liniową złożoność czasową rozpoznawania~\(O(n)\), podczas gdy niedeterministyczne (NFA) mogą wymagać przeszukiwania z~nawrotami (ang.~\emph{backtracking}).

Gramatyki bezkontekstowe (CFG) definiują strukturę syntaktyczną języków programowania.
Parsery dla CFG dzielą się na dwie główne kategorie: parsery zstępujące rekurencyjnie (ang.~\emph{top-down}), takie jak LL(\(k\))~\cite{lewis1968}, oraz parsery wstępujące rekurencyjnie (ang.~\emph{bottom-up}), takie jak LR(\(k\))~\cite{knuth1965}.
Wybór klasy parsera determinuje kompromisy między mocą wyrazu gramatyki, złożonością implementacji oraz jakością komunikatów błędów.


\section[Teza i pytania badawcze]{Teza i~pytania badawcze}
\label{sec:teza-badawcza}

W~niniejszej pracy przyjęto tezę, zgodnie z~którą wykorzystanie metaprogramowania w~języku Scala~3 (makra kompilacyjne, typy rafinowane) umożliwia konstrukcję systemu generującego analizatory leksykalne i~składniowe charakteryzujących się następującymi właściwościami:

\begin{enumerate}
    \item wydajność --- czas parsowania porównywalny z~narzędziami opartymi na generacji kodu (ANTLR, Yacc), przewyższający biblioteki interpretowane (PLY, SLY).
    \item użyteczność --- interfejs programistyczny (API) niezależny od dedykowanego DSL, zintegrowany z~systemem typów Scali i~wspierany przez standardowe narzędzia IDE\@.
    \item diagnostyka błędów --- komunikaty błędów generowane w~czasie kompilacji (dla błędów gramatyki) oraz w~czasie parsowania (dla błędów składniowych), zawierające kontekst syntaktyczny.
\end{enumerate}

W~ramach weryfikacji tezy sformułowano następujące pytania badawcze:
\begin{enumerate}
    \item Czy możliwe jest osiągnięcie wydajności zbliżonej do~generatorów kodu przy zachowaniu elastyczności bibliotek kombinatorów poprzez zastosowanie metaprogramowania?
    \item W~jakim stopniu wykorzystanie typów rafinowanych w~Scali~3 wpływa na~bezpieczeństwo typów i~komfort pracy z~wygenerowanym parserem?
    \item Jakie ograniczenia maszyny wirtualnej Java (JVM) wpływają na~proces generacji kodu w~czasie kompilacji i~jak można je efektywnie niwelować?
\end{enumerate}

Celem pracy jest zaprojektowanie i~zaimplementowanie narzędzia \textit{ALPACA} (\textit{Another Lexer Parser And Compiler Alpaca}) w~języku Scala, które implementuje funkcjonalności powszechnie stosowane w~budowie analizatorów leksykalnych i~składniowych, weryfikując postawioną tezę.


\section{Motywacja projektu}
\label{sec:motywacja-projektu}

Istniejące narzędzia do konstrukcji analizatorów leksykalnych i~składniowych wykazują szereg ograniczeń utrudniających ich zastosowanie w~kontekście nowoczesnych języków programowania oraz środowisk deweloperskich.
Identyfikacja tych ograniczeń stanowiła punkt wyjścia dla projektu~\textit{ALPACA}.

Projekt~\textit{ALPACA} stanowi narzędzie do~generowania lekserów i~parserów w~języku Scala, łączące zalety istniejących rozwiązań poprzez:

\begin{enumerate}
    \item Połączenie wydajności generatorów kodu z~użytecznością bibliotek, czyli wykorzystanie makr kompilacyjnych Scali~3, co pozwala przenieść część obliczeń na~etap kompilacji, zachowując interfejs programistyczny zintegrowany z~systemem typów języka.
    \item Generowanie komunikatów błędów w~oparciu o~kontekst parsera~LR(1).
    \item Natywną integrację ze~środowiskami IDE, gdyż implementacja w~czystym języku Scala eliminuje konieczność stosowania dedykowanych pluginów, wykorzystując istniejące wsparcie dla języka (IntelliJ IDEA, Metals).
\end{enumerate}

Proponowane rozwiązanie łączy nowoczesne podejście technologiczne z~praktycznym zastosowaniem w~edukacji i~programowaniu.
Może ono służyć jako narzędzie dydaktyczne, ułatwiając naukę teorii kompilacji, w~pracach badawczych, a~także jako kompleksowe narzędzie do tworzenia praktycznych rozwiązań.


\section{Przegląd istniejących rozwiązań}
\label{sec:przeglad-istniejacych-rozwiazan}

Narzędzia do konstrukcji analizatorów leksykalnych i~składniowych można sklasyfikować według strategii implementacyjnej na trzy główne kategorie: generatory kodu, biblioteki interpretowane oraz kombinatory parserów.

\subsection{Generatory kodu}
\label{subsec:generatory-kodu}

Generatory kodu transformują deklaratywne specyfikacje gramatyk w~kod źródłowy parsera w~języku docelowym.
Proces ten odbywa się przed kompilacją programu głównego i~wymaga dodatkowego narzędzia w~procesie budowania (ang.~\emph{build chain}).

\subsubsection[Lex i Yacc]{Lex i~Yacc}
\label{subsubsec:lex-yacc}

\textit{Lex}~\cite{lesk1975lex} i~\textit{Yacc}~\cite{johnson1975yacc} to klasyczne, dobrze ugruntowane narzędzia, które odegrały kluczową rolę w~tworzeniu setek współczesnych języków programowania.
Definicja leksera i~parsera w~tych systemach odbywa się poprzez specjalnie zaprojektowaną składnię konfiguracyjną.
Narzędzia te wymuszają znajomość dedykowanej składni specyfikacji gramatyk, co utrudnia rozpoczęcie pracy dla początkujących użytkowników.

Ponieważ \textit{Lex} i~\textit{Yacc} zostały zaprojektowane do współpracy z~językiem~C, ich integracja z~nowoczesnymi językami programowania bywa utrudniona.
Rozszerzanie tych narzędzi o~dodatkowe, specyficzne funkcjonalności jest skomplikowane, co ogranicza ich elastyczność.
Brak wsparcia dla współczesnych środowisk programistycznych (IDE) dodatkowo obniża komfort użytkowania w~porównaniu z~nowoczesnymi alternatywami.

\lstinputlisting[language=c,caption={Fragment definicji parsera Ruby w~technologii Yacc},label={lst:ruby-parser}]{listings/introduction/01-ruby-parser.y}

\subsubsection{ANTLR}
\label{subsubsec:antlr}

\textit{ANTLR}~\cite{parr2004s} to rozwiązanie inspirowane narzędziami \textit{Lex} i~\textit{Yacc}, oferujące zaawansowane mechanizmy analizy składniowej.
Jego twórcy opracowali dedykowany język DSL, znany jako Grammar~v4, który umożliwia definiowanie składni analizowanego języka.
Na podstawie tej definicji \textit{ANTLR} generuje parser w~wybranym przez użytkownika języku programowania, takim jak Python, Java, C++ lub JavaScript.

Wspomaganie pracy z~\textit{ANTLR} w~znacznym stopniu ułatwiają dedykowane wtyczki do środowisk Visual Studio Code oraz IntelliJ IDEA\@.
Oferują one funkcjonalności, takie jak kolorowanie składni, autouzupełnianie kodu, nawigację do definicji leksemów oraz walidację błędów, co znacząco przyspiesza proces tworzenia parserów.

Jedną z~kluczowych różnic \textit{ANTLR} w~porównaniu do innych narzędzi jest wykorzystanie gramatyki LL(*), podczas gdy klasyczne rozwiązania, takie jak Yacc czy SLY, implementują LALR(1).
LL(*) jest bardziej intuicyjna i~czytelna dla programistów, co ułatwia definiowanie reguł składniowych.
Jednakże jej zastosowanie wiąże się z~większym zużyciem pamięci oraz niższą wydajnością w~porównaniu do LALR(1).

Dodatkowym wyzwaniem podczas korzystania z~\textit{ANTLR} jest konieczność nauki składni DSL Grammar~v4 oraz ograniczenie wsparcia dla narzędzi deweloperskich.
Pełne wykorzystanie możliwości \textit{ANTLR} wymaga korzystania z~jednego z~dedykowanych środowisk, co może stanowić istotne ograniczenie dla użytkowników preferujących inne IDE\@.

\subsection{Biblioteki interpretowane}
\label{subsec:biblioteki-interpretowane}

Biblioteki interpretowane definiują gramatyki jako struktury danych w~języku bazowym.
Parser jest wykonywany w~czasie działania programu, co eliminuje krok generacji kodu, ale wprowadza narzut wydajnościowy.

\subsubsection[PLY i SLY]{PLY i~SLY}
\label{subsubsec:ply-sly}

\textit{PLY}~\cite{ply} i~jego nowszy odpowiednik \textit{SLY}~\cite{sly} to biblioteki inspirowane narzędziami Lex i~Yacc.
Oferują elastyczne podejście do budowy parserów, umożliwiając samodzielną implementację obsługi leksemów, budowę drzewa AST, czy dodatkowe funkcjonalności takie jak obliczanie numeru linii w~lekserze.

Głównym ograniczeniem PLY i~SLY jest implementacja w~języku Python.
Ze względu na interpretowany charakter oraz dynamiczne typowanie, parsery te charakteryzują się niską wydajnością, a~brak statycznego typowania utrudnia wykrywanie błędów na etapie tworzenia analizatora leksykalnego lub składniowego.
Mechanizm refleksji wykorzystywany przez bibliotekę~\textit{SLY} (inspekcja nazw metod i~typów) powoduje generowanie ostrzeżeń przez analizatory statyczne środowiska PyCharm.
Ponadto należy zaznaczyć, iż autor projektu informuje o~braku dalszego rozwoju tych narzędzi~\cite{sly-github}.

Przykład~\ref{lst:python-parser} ilustruje kilka nieintuicyjnych, automatycznych mechanizmów obecnych w~bibliotece \textit{SLY}:

\paragraph{Dekorator \texttt{@\_()}}
Dekorator ten definiuje wzorzec dopasowania dla produkcji.
Argumenty w~cudzysłowie są traktowane jako literały, podczas gdy identyfikatory bez cudzysłowu odnoszą się do innych nieterminali.

\paragraph{Konwencja nazewnictwa metod}
Nazwa metody określa typ zwracany przez produkcję.
Parser automatycznie identyfikuje wszystkie metody o~danej nazwie jako alternatywne produkcje dla tego nieterminala.
Mechanizm ten eliminuje potrzebę jawnej deklaracji reguł, ale utrudnia śledzenie struktury gramatyki.

\paragraph{Priorytet operatorów}
W~krotce \texttt{precedence} definiowane jest pierwszeństwo operatorów, jednakże dodanie \texttt{\%~prec} pozwala nadpisać priorytet dla konkretnej reguły składniowej.

\paragraph{Dostęp do kontekstu}
Argument~\texttt{p} pozwala na dostęp do kontekstu produkcji (np.\ numeru linii), ale także do zmiennych we wzorcu dopasowania w~adnotacji.
Jeśli zdefiniowany jest więcej niż jeden element, dodawany jest numer do akcesora, np.\ \texttt{expr1} jest odwołaniem do drugiego wyrażenia \texttt{expr}.

\lstinputlisting[language=Python,caption={Fragment definicji parsera w~Pythonie, wykorzystujący bibliotekę SLY},label={lst:python-parser}]{listings/introduction/02-sly-parser.py}

Komunikaty błędów w~bibliotece \textit{SLY} nie zawierają informacji o~kontekście syntaktycznym ani sugestii poprawek, co obrazuje przykład~\ref{lst:not-working-sly}, który po uruchomieniu informuje użytkownika błędem z~fragmentu kodu~\ref{lst:sly-result}.
Okazuje się, że problemem był brak atrybutu \texttt{ignore\_comment} w~definicji \texttt{Lexer}.

\lstinputlisting[language=Python,caption={Fragment niedziałającego kodu w~Pythonie, wykorzystujący bibliotekę SLY},label={lst:not-working-sly}]{listings/introduction/03-not-working-sly.py}

\lstinputlisting[language=terminal,caption={Przykładowy komunikat błędu w~bibliotece \textit{SLY}},label={lst:sly-result}]{listings/introduction/04-not-working-result}

\subsection{Kombinatory parserów}
\label{subsec:kombinatory-parserow}

Kombinatory parserów to funkcje wyższego rzędu konstruujące złożone parsery z~prostszych komponentów.
Podejście to łączy elastyczność bibliotek z~czytelną składnią zbliżoną do notacji BNF\@.

\subsubsection{Scala parser combinators}
\label{subsubsec:scala-parser-combinators}

Biblioteka \textit{Scala parser combinators}~\cite{moors2008parser} była popularnym sposobem na tworzenie parserów, lecz jak stwierdzono w~samej dokumentacji: \enquote{Trudno jest jednak zrozumieć ich działanie i~jak zacząć.
    Po skompilowaniu i~uruchomieniu kilku pierwszych przykładów, mechanizm działania staje się bardziej zrozumiały, ale do tego czasu może stanowić istotną przeszkodę, a~standardowa dokumentacja nie jest zbyt pomocna}~\cite{parser-combinators-readme}.

\subsubsection{ScalaBison}
\label{subsubsec:scala-bison}

Z~podsumowania artykułu na temat \textit{ScalaBison}~\cite{boyland2010tool} wiadomo, że to praktyczny generator parserów dla języka Scala oparty na technologii rekurencyjnego wstępowania i~zstępowania, który akceptuje pliki wejściowe w~formacie \textit{bison}.
Parsery generowane przez \textit{ScalaBison} używają bardziej informacyjnych komunikatów o~błędach niż te generowane przez pierwowzór \textit{bison}, a~także szybkość parsowania i~wykorzystanie miejsca są znacznie lepsze niż \textit{scala-combinators}, ale są nieco wolniejsze niż najszybsze generatory parserów oparte na JVM\@.

Dodatkowo należy zaznaczyć, iż jest to rozwiązanie już niewspierane i~stworzone w~celach akademickich.
Korzysta z~przestarzałej wersji Scali, nie posiada wyczerpującej dokumentacji i~liczba funkcjonalności jest bardzo ograniczona w~porównaniu do np.\ technologii \textit{SLY}.

\subsubsection{parboiled2}
\label{subsubsec:parboiled-2}

\textit{parboiled2}~\cite{myltsev2019parboiled2} to biblioteka w~Scali umożliwiająca lekkie i~szybkie parsowanie dowolnego tekstu wejściowego.
Implementuje ona oparty na makrach generator parsera dla gramatyk wyrażeń parsujących (PEG), który działa w~czasie kompilacji i~tłumaczy definicję reguły gramatycznej na odpowiadający jej bytecode JVM\@.
Ze względu na skomplikowany i~nieintuicyjny DSL, bariera wejścia dla nowych użytkowników jest wysoka.
Zgodnie z~przykładem~\ref{lst:parboiled2-error}, raportowanie błędów jest bardzo ograniczone (problem z~implementacją wynika jedynie z~różnic w~liczbie parametrów funkcji).

\lstinputlisting[firstline=7,lastline=20,language=terminal,caption={[Fragment błędu wygenerowanego przez bibliotekę parboiled2]
            Niewielki fragment (14 z~133~linii) błędu wygenerowanego przez bibliotekę \textit{parboiled2}, który pochodzi z~prezentacji Li~Haoyi na temat \textit{FastParse}~\cite{fastparse-talk}.
        },label={lst:parboiled2-error}]{listings/introduction/05-parboiled2-error}

\subsubsection{FastParse}
\label{subsubsec:fastparse}

\textit{FastParse}~\cite{fastparse-docs} to opracowana przez Li~Haoyi wysokowydajna biblioteka kombinatorów parserów dla Scali, zaprojektowana w~celu uproszczenia tworzenia parserów tekstu strukturalnego.
Umożliwia ona programistom definiowanie parserów rekurencyjnych, dzięki czemu nadaje się do parsowania języków programowania, formatów danych, takich jak JSON, czy DSL-i.
Cechą charakterystyczną FastParse jest równowaga między użytecznością a~wydajnością.
Parsery są konstruowane poprzez łączenie mniejszych parserów za pomocą operatorów, takich jak~\verb&~& dla sekwencjonowania i~\verb&|& dla alternatyw, przy jednoczesnym zachowaniu czytelności zbliżonej do formalnych definicji gramatyki.
Według dokumentacji~\cite{fastparse-docs}, parsery \textit{Fastparse} zajmują 1/10 kodu w~porównaniu do ręcznie napisanego parsera rekurencyjnego.
W~porównaniu do narzędzi generujących parsery, takich jak \textit{ANTLR} lub \textit{Lex} i~\textit{Yacc}, implementacja nie wymaga żadnego specjalnego kroku kompilacji lub generowania kodu.
To sprawia, że rozpoczęcie pracy z~\textit{Fastparse} jest znacznie łatwiejsze niż w~przypadku bardziej tradycyjnych narzędzi do generowania parserów.
Przykładowo, parser wyrażeń arytmetycznych może być zwięźle napisany, aby obsługiwać zagnieżdżone nawiasy, pierwszeństwo operatorów i~raportowanie błędów w~mniej niż 20~liniach kodu~\cite{fastparse-slides}.
Biblioteka kładzie również nacisk na debugowanie, generując szczegółowe komunikaty o~błędach, które wskazują dokładną lokalizację i~przyczynę niepowodzeń parsowania, takich jak niedopasowane nawiasy lub nieprawidłowe tokeny.

\subsection{Analiza porównawcza}
\label{subsec:analiza-porownawcza}

Tabela~\ref{tab:porownanie-alternatyw} zestawia główne cechy analizowanych narzędzi.
Widoczny jest kompromis między wydajnością a~użytecznością: generatory kodu (Lex/Yacc, ANTLR) osiągają wysoką wydajność, ale wymagają dodatkowego kroku kompilacji i~nauki DSL\@.
Biblioteki kombinatorów (FastParse, parboiled2) oferują interfejs zintegrowany z~językiem bazowym, ale kosztem spadku wydajności związanej z~interpretacją reguł w~czasie wykonania.

Żadne z~analizowanych rozwiązań nie łączy jednocześnie:
\begin{itemize}
    \item wysokiej wydajności (generacja kodu w~czasie kompilacji),
    \item interfejsu API zintegrowanego z~systemem typów języka,
    \item komunikatów błędów zawierających kontekst syntaktyczny,
    \item natywnej integracji ze środowiskami IDE bez dedykowanych pluginów.
\end{itemize}

Luka ta stanowi motywację dla projektu~\textit{ALPACA}, który wykorzystuje makra kompilacyjne Scali~3 do osiągnięcia tych celów jednocześnie.

\begin{table}[ht]
    \centering
    \begin{tabular}{L|CCCC}
        \toprule
        \large{Narzędzie}       & \textbf{Lex\&Yacc}                        & \textbf{PLY/SLY}     & \textbf{ANTLR}     & \textbf{scala-bison} \\
        \midrule
        Język implementacji     & C                                         & Python               & Java               & Scala (nad Bisonem)  \\
        \arrayrulecolor{gray}
        \hline
        Język użycia            & regex, BNF, akcje w~C                     & DSL                  & DSL oparty na EBNF & BNF, akcje w~Scali   \\
        \hline
        Wydajność               & wysoka                                    & niska                & umiarkowana        & wysoka               \\
        \hline
        Łatwość użycia          & średnia                                   & umiarkowana          & wysoka             & średnia              \\
        \hline
        Aktywne wsparcie        & brak                                      & nie                  & tak                & nie                  \\
        \hline
        Diagnostyka błędów      & słaba                                     & średnia              & dobra              & słaba                \\
        \hline
        Dokumentacja            & dobra                                     & średnia, nieaktualna & dobra              & słaba                \\
        \hline
        Popularność             & wysoka                                    & średnia              & wysoka             & niska                \\
        \hline
        Integracja IDE          & nieoficjalny plugin                       & ograniczona          & oficjalny plugin   & brak                 \\
        \hline
        Wsparcie do debugowania & brak                                      & dobre                & częściowe          & dobre                \\
        \hline
        Generowanie kodu        & nie                                       & nie                  & tak                & nie                  \\
        \hline
        \toprule
        Narzędzie               & \textbf{Scala parser\newline combinators} & \textbf{parboiled2}  & \textbf{FastParse} & \textbf{ALPACA}      \\
        \midrule
        Język implementacji     & Scala                                     & Scala                & Scala              & Scala                \\
        \hline
        Język użycia            & DSL w~Scali                               & DSL w~Scali          & DSL w~Scali        & Scala                \\
        \hline
        Wydajność               & wysoka                                    & umiarkowana          & wysoka             & wysoka               \\
        \hline
        Łatwość użycia          & niska                                     & średnia              & średnia            & wysoka               \\
        \hline
        Aktywne wsparcie        & nie                                       & nie                  & tak                & tak                  \\
        \hline
        Diagnostyka błędów      & dobra                                     & niska                & dobra              & dobra                \\
        \hline
        Dokumentacja            & słaba                                     & bardzo dobra         & bardzo dobra       & dobra                \\
        \hline
        Popularność             & średnia                                   & niska                & rosnąca            & niska                \\
        \hline
        Integracja IDE          & wsparcie dla Scali                        & wsparcie dla Scali   & wsparcie dla Scali & wsparcie dla Scali   \\
        \hline
        Wsparcie do debugowania & dobre                                     & dobre                & dobre              & dobre                \\
        \hline
        Generowanie kodu        & nie                                       & nie                  & nie                & tak                  \\
        \bottomrule
    \end{tabular}
    \caption{Porównanie wybranych narzędzi do generowania analizatorów leksykalnych i~składniowych}
    \label{tab:porownanie-alternatyw}
\end{table}


\section[Ograniczenia i zakres pracy]{Ograniczenia i~zakres pracy}
\label{sec:ograniczenia-zakres}

Niniejsza praca koncentruje się na implementacji parsera LR(1) oraz analizatora leksykalnego wykorzystującego wyrażenia regularne.
Następujące aspekty wykraczają poza zakres pracy:

\begin{itemize}
    \item System generuje kanoniczne stany LR(1) bez minimalizacji do LALR(1), co może prowadzić do większych tablic akcji.
          Implementacja minimalizacji stanowi potencjalny kierunek przyszłych badań.

    \item Ewaluacja empiryczna w~kontekście dydaktycznym, czyli weryfikacja użyteczności systemu w~środowisku akademickim (badanie z~udziałem studentów) wykracza poza zakres pracy i~stanowi kierunek przyszłych badań.
\end{itemize}


\chapter[Metaprogramowanie w Scali 3]{Metaprogramowanie w~Scali~3}
\label{ch:metaprogramowanie-w-scali-3}


\section{Wprowadzenie}\label{sec:wprowadzenie}
Scala~3, znana również jako Dotty, wprowadza całkowicie przeprojektowany system metaprogramowania, stanowiący fundamentalną zmianę w~stosunku do eksperymentalnych makr dostępnych w~Scali~2~\cite{scala3-dropped-scala2-macros,scala3-metaprogramming}.
Metaprogramowanie w~Scali~3 zostało zaprojektowane z~naciskiem na bezpieczeństwo typów, przenośność oraz skalowalność, umożliwiając twórcom oprogramowania generowanie i~analizowanie kodu w~czasie kompilacji przy zachowaniu pełnej ekspresywności języka~\cite{stucki2024infoscience,stucki2020inlining}.
W~przeciwieństwie do poprzedniego systemu, który eksponował wewnętrzne mechanizmy kompilatora i~był źródłem problemów z~kompatybilnością między wersjami~\cite{stucki2020thesis}, nowy system metaprogramowania jest zaprojektowany jako stabilny i~przenośny interfejs programistyczny.

Podstawą teoretyczną systemu metaprogramowania w~Scali~3 jest programowanie wieloetapowe (ang.~\emph{multi-stage programming}), paradygmat pozwalający na odróżnienie różnych etapów wykonania programu~\cite{scala3-staging,stucki2020thesis}.
W~tym modelu kod może być wykonywany w~różnych fazach: w~czasie kompilacji (ang.~\emph{compile-time}) lub w~czasie wykonania (ang.~\emph{runtime})~\cite{scala3-staging}.
Rozdzielenie tych faz pozwala na przeniesienie obliczeń z~czasu wykonania do~czasu kompilacji, co potencjalnie eliminuje narzut wykonania i~umożliwia wcześniejszą detekcję błędów.

\subsection[Cytaty i wstawki]{Cytaty i~wstawki}\label{subsec:cytaty-i-wstawki}
Kluczowymi koncepcjami w~systemie metaprogramowania Scali~3 są cytaty (ang.~\emph{quotes}) i~wstawki (ang.~\emph{splices})~\cite{stucki2018unification,stucki2021multistage}.
Cytaty, oznaczane jako \verb|'{...}|, służą do opóźnienia wykonania kodu i~traktowania go jako danych~\cite{scala3-reflection}.
Wstawki, oznaczane jako \verb|${...}|, pozwalają na ocenę wyrażenia generującego kod i~wstawienie wyniku do otaczającego kontekstu~\cite{scala3-reflection,scala3-guides-quotes}.

\paragraph[Przykład użycia cytatów i wstawek]{Przykład użycia cytatów i~wstawek}
Poniższy przykład ilustruje podstawowe wykorzystanie cytatów i~wstawek w~makrach:

\lstinputlisting[language=Scala,caption={Proste makro z wykorzystaniem cytatów i~wstawek},label={lst:meta-quotes-example}]{listings/metaprogramming/meta-01-quotes-example.scala}

W~powyższym przykładzie:
\begin{itemize}
  \item \verb|'x| tworzy cytat (ang.~\emph{quote}) z~wyrażenia \verb|x|, opóźniając jego wykonanie
  \item \verb|'{...}| tworzy blok kodu jako dane, które będzie wstawione w~miejscu wywołania makra
  \item \verb|$x| wstawia (ang.~\emph{splice}) wartość cytatu do~nowego kontekstu
\end{itemize}

Formalna semantyka tych konstrukcji została przedstawiona w~pracy Stuckiego, Brachthäusera i~Odersky'ego~\cite{stucki2021multistage}, gdzie cytaty i~wstawki są traktowane jako prymitywne formy w~typowanych drzewach składniowych (ang.~\emph{typed abstract syntax trees}).
Autorzy dowodzą, że system zachowuje bezpieczeństwo typów oraz higieniczność, zapewniając, że wygenerowany kod nie może przypadkowo powiązać identyfikatorów z~niewłaściwymi zmiennymi~\cite{stucki2021multistage}.

\subsection{Bezpieczeństwo międzyetapowe}\label{subsec:bezpieczenstwo-miedzyetapowe}
Scala~3 gwarantuje bezpieczeństwo międzyetapowe (ang.~\emph{cross-stage safety}) poprzez sprawdzanie poziomów etapowania w~czasie kompilacji~\cite{stucki2020thesis,stucki2021multistage}.
Zmienne lokalne mogą być używane tylko na tym samym poziomie etapowania, na którym zostały zdefiniowane, co zapobiega dostępowi do zmiennych, które jeszcze nie istnieją lub już nie są dostępne~\cite{stucki2020thesis}.

\paragraph{Przykład naruszenia bezpieczeństwa międzyetapowego}
Następujący kod \textbf{nie skompiluje się}, ponieważ narusza zasady bezpieczeństwa międzyetapowego:

\lstinputlisting[language=Scala,caption={Błąd bezpieczeństwa międzyetapowego (kod nie kompiluje się)},label={lst:meta-cross-stage-error}]{listings/metaprogramming/meta-02-cross-stage-error.scala}

Kompilator wykryje ten błąd i~zgłosi komunikat: \texttt{error: access to value localVar from wrong staging level}.
Aby poprawnie odnieść się do~wartości z~otaczającego kontekstu, należy użyć mechanizmu \verb|Expr.apply|:

\lstinputlisting[language=Scala,caption={Poprawne przeniesienie wartości między etapami},label={lst:meta-cross-stage-correct}]{listings/metaprogramming/meta-03-cross-stage-correct.scala}

System również zapewnia, że typy generyczne używane w~wyższym poziomie etapowania niż ich definicja wymagają instancji klasy typu \verb|Type[T]|, która niesie reprezentację typu niepoddaną wymazywaniu (ang.~\emph{type erasure})~\cite{stucki2020thesis}.
To podejście rozwiązuje problem wymazywania typów generycznych w~JVM, zachowując informację o~typach potrzebną w~kolejnych etapach kompilacji.


\section[Mechanizmy metaprogramowania w Scali 3]{Mechanizmy metaprogramowania w~Scali~3}\label{sec:mechanizmy-metaprogramowania-w-scali-3}

Przedstawione powyżej podstawy teoretyczne znajdują bezpośrednie zastosowanie w~praktycznych mechanizmach metaprogramowania oferowanych przez język Scala~3, które zostaną omówione w~niniejszej sekcji.

\subsection{Definicje inline}\label{subsec:definicje-inline}
Najprostszym narzędziem metaprogramowania jest modyfikator \verb|inline|~\cite{stucki2020semantics-preserving}.
Gwarantuje on, że wywołanie oznaczonej nim metody lub wartości zostanie w~całości wstawione w~miejscu wywołania (ang.~\emph{inlining}) podczas kompilacji.
Jest to instrukcja dla kompilatora, a~nie tylko sugestia, jak w~niektórych innych językach~\cite{lilis2019survey}.

\paragraph{Przykład definicji inline}
\lstinputlisting[language=Scala,caption={Użycie modyfikatora inline dla optymalizacji},label={lst:meta-inline-example}]{listings/metaprogramming/meta-04-inline-example.scala}

Modyfikator \verb|inline| różni się od zwykłych funkcji tym, że \textbf{gwarantuje} wstawienie kodu, podczas gdy standardowe funkcje mogą być zinlinowane przez kompilator jako optymalizacja, ale nie muszą.

\subsection{Makra oparte na wyrażeniach}\label{subsec:makra-oparte-na-wyrazeniach}
Makra w~Scali~3 są zdefiniowane jako metody \verb|inline| zawierające wstawkę najwyższego poziomu (ang.~\emph{top-level splice})~\cite{scala3-reference-macros,scala3-guides-macros}, czyli taki, który nie jest zagnieżdżony w~żadnym cytacie (ang.~\emph{quote}) i~jest wykonywany w~czasie kompilacji~\cite{scala3-staging,scala3-reference-macros}.

Typ \verb|Expr[T]| reprezentuje wyrażenie Scali o~typie \verb|T| jako typowane drzewo składniowe~\cite{scala3-reflection,scala3-guides-macros}.
Makra manipulują wartościami typu \verb|Expr[T]|, transformując je lub generując nowe wyrażenia~\cite{scala3-guides-macros}.
Ta reprezentacja gwarantuje bezpieczeństwo typów na poziomie języka metaprogramowania~\cite{scala3-reflection}.

\paragraph{Przykład makra generującego kod}
\lstinputlisting[language=Scala,caption={Makro generujące kod inspekcji typu},label={lst:meta-macro-example}]{listings/metaprogramming/meta-05-macro-example.scala}

W~powyższym przykładzie makro \verb|showType| wykorzystuje refleksję TASTy (sekcja~\ref{subsec:refleksja-tasty}) do~uzyskania reprezentacji typu w~czasie kompilacji i~wygenerowania kodu zwracającego jego nazwę.

\subsection[Dopasowanie wzorców w cytatach kodu]{Dopasowanie wzorców w~cytatach kodu}\label{subsec:dopasowanie-wzorcow-kodu}
Scala~3 wspiera analizę kodu poprzez dopasowanie wzorców w~cytatach kodu (ang.~\emph{quote pattern matching})~\cite{stucki2020thesis,stucki2021multistage}.
Mechanizm ten pozwala na dekonstrukcję kawałków kodu i~ekstrakcję podwyrażeń~\cite{stucki2021multistage}.

Stucki, Brachthäuser i~Odersky~\cite{stucki2021multistage} wprowadzają wzorce wiążące (ang.~\emph{bind patterns}) postaci \verb|$x| oraz wzorce HOAS (ang.~\emph{Higher-Order Abstract Syntax}) postaci \verb|$f(y)|, które pozwalają na ekstrakcję podwyrażeń potencjalnie zawierających zmienne z~zewnętrznego kontekstu.
System gwarantuje, że ekstrahowane wyrażenia są zamknięte względem definicji wewnątrz wzorca, zapobiegając wyciekom zakresu.

\paragraph{Przykład dopasowania wzorców kodu}
\lstinputlisting[language=Scala,caption={Optymalizacja wyrażeń algebraicznych poprzez dopasowanie wzorców},label={lst:meta-pattern-matching}]{listings/metaprogramming/meta-06-pattern-matching.scala}

Makro \verb|optimize| rozpoznaje wzorce wyrażeń arytmetycznych i~zastępuje je uproszczonymi wersjami w~czasie kompilacji, eliminując zbędne operacje.

\subsection{Refleksja TASTy}\label{subsec:refleksja-tasty}
Dla przypadków wymagających głębszej analizy kodu, Scala~3 oferuje API refleksji TASTy~\cite{scala3-reflection,scala3-guides-reflection}.
TASTy (ang.~\emph{Typed Abstract Syntax Trees}) jest binarnym formatem serializacji typowanych drzew składniowych używanym przez kompilator Scali~3~\cite{stucki2020thesis}.

API refleksji dostarcza szczegółowy widok na strukturę kodu, włączając typy, symbole oraz pozycje w~kodzie źródłowym.
Jest dostępne poprzez obiekt \verb|reflect| zdefiniowany w~typie \verb|Quotes|, który jest przekazywany kontekstualnie do makr~\cite{scala3-reflection,scala3-guides-reflection}.

\paragraph{Hierarchia klas refleksji TASTy}
System refleksji TASTy definiuje następującą hierarchię typów:

\begin{itemize}
  \item \textbf{Tree} --- podstawowy typ reprezentujący węzeł drzewa składni
  \item \textbf{Term} --- wyrażenia (np.~wywołania funkcji, literały)
  \item \textbf{TypeTree} --- reprezentacje typów w~drzewie składni
  \item \textbf{Symbol} --- symbole (definicje klas, metod, zmiennych)
  \item \textbf{TypeRepr} --- reprezentacje typów (niezależne od~drzewa)
\end{itemize}

\paragraph{Przykład użycia refleksji TASTy}
\lstinputlisting[language=Scala,caption={Inspekcja struktury klasy przypadku za pomocą refleksji TASTy},label={lst:meta-tasty-reflection}]{listings/metaprogramming/meta-07-tasty-reflection.scala}

Makro \verb|inspectFields| wykorzystuje refleksję TASTy do~ekstrakcji nazw pól klasy przypadku w~czasie kompilacji, co pozwala na~generowanie kodu specyficznego dla struktury typu bez ręcznej specyfikacji.


\section[Porównanie z innymi systemami metaprogramowania]{Porównanie z~innymi systemami metaprogramowania}\label{sec:porownanie-metaprogramowania}

System metaprogramowania Scali~3 czerpie inspiracje z~innych języków, ale wprowadza własne innowacje w~zakresie bezpieczeństwa typów i~ergonomii.

\subsection[Makra w Lisp i Scheme]{Makra w~Lisp i~Scheme}\label{subsec:makra-wlisp-ischeme}
Język Lisp~\cite{mccarthy1960lisp} był pionierem w~dziedzinie metaprogramowania, wprowadzając koncepcję makr jako transformacji list reprezentujących kod.
Kluczową różnicą między makrami Lisp a~Scali~3 jest:

\begin{itemize}
  \item \textbf{Lisp:} makra operują na nietypo

        wanych listach (\emph{S-expressions}), co umożliwia dużą elastyczność, ale eliminuje sprawdzanie typów w~czasie kompilacji
  \item \textbf{Scala~3:} makra operują na typowanych drzewach składni (TASTy), zapewniając pełne bezpieczeństwo typów
\end{itemize}

\subsection{Template Haskell}
Template Haskell~\cite{sheard2002template} wprowadza programowanie wieloetapowe do~języka Haskell poprzez cytaty i~wstawki, podobnie jak Scala~3.
Główne podobieństwa i~różnice:

\begin{itemize}
  \item \textbf{Podobieństwa:} obie implementacje wykorzystują cytaty (\verb|[| ... |]| w~Haskell, \verb|'{ ... }| w~Scali) oraz wstawki (\verb|$(...)| w~Haskell, \verb|${...}| w~Scali)
  \item \textbf{Różnice:} Template Haskell wymaga specjalnego trybu kompilacji (\verb|-XTemplateHaskell|), podczas gdy makra Scali~3 są standardową częścią języka; Scala~3 oferuje bogatsze API refleksji (TASTy)
\end{itemize}

\subsection{Makra w~Rust}\label{subsec:makra-wrust}
Język Rust oferuje dwa systemy makr: makra deklaratywne (\verb|macro\_rules!|) oraz makra proceduralne~\cite{rust-macros,klabnik2018rust}.
W~porównaniu do~Scali~3:

\begin{itemize}
  \item \textbf{Rust:} makra proceduralne operują na~tokenach (ang.~\emph{token stream}), co daje dużą kontrolę, ale utrudnia analizę semantyczną
  \item \textbf{Scala~3:} makra operują na~typowanych AST, co umożliwia analizę semantyczną i~sprawdzanie typów wygenerowanego kodu
\end{itemize}


\section[Zastosowania metaprogramowania w projekcie ALPACA]{Zastosowania metaprogramowania w~projekcie ALPACA}\label{sec:zastosowania-metaprogramowania}

System metaprogramowania Scali~3 stanowi fundament implementacji projektu \textit{ALPACA}.
Kluczowe zastosowania obejmują:

\begin{enumerate}
  \item \textbf{Generacja klas anonimowych} (sekcja~\ref{subsec:generacja-klasy-anonimowej}) --- wykorzystanie \verb|Symbol.newClass| do~programatycznego tworzenia typów w~czasie kompilacji

  \item \textbf{Transformacja AST} (sekcja~\ref{subsec:transformacja-i-adaptacja-referencji}) --- przepisywanie właścicieli symboli (\emph{re-owning}) poprzez \verb|ReplaceRefs|

  \item \textbf{Typy rafinowane} (sekcja~\ref{subsec:typy-rafinowane}) --- dynamiczne rozszerzanie typów o~pola strukturalne poprzez \verb|Refinement|

  \item \textbf{Walidacja w~czasie kompilacji} (sekcja~\ref{subsec:walidacja-i-obsuga-bedow}) --- wykrywanie błędów gramatyki przed wykonaniem programu
\end{enumerate}

Szczegółowa analiza implementacji tych mechanizmów zostanie przedstawiona w~rozdziale~\ref{ch:implementacja}.


\section{Podsumowanie rozdziału}\label{sec:metaprogramming-summary}

Rozdział przedstawił system metaprogramowania Scali~3 jako fundament teoretyczny dla projektu \textit{ALPACA}.
Kluczowe wnioski:

\begin{itemize}
  \item System cytatów i~wstawek (ang.~\emph{quotes and splices}) umożliwia bezpieczne przenoszenie kodu między fazami kompilacji

  \item Bezpieczeństwo międzyetapowe (ang.~\emph{cross-stage safety}) zapobiega błędom związanym z~dostępem do~zmiennych z~niewłaściwych faz

  \item Refleksja TASTy dostarcza bogatego API do~analizy i~transformacji kodu w~czasie kompilacji

  \item Scala~3 łączy zalety systemów metaprogramowania z~Lisp, Template Haskell i~Rust, wprowadzając własne innowacje w~zakresie bezpieczeństwa typów
\end{itemize}

Mechanizmy te stanowią podstawę implementacji opisanej w~rozdziale~\ref{ch:implementacja}, gdzie zostaną zastosowane do~konstrukcji lekserów i~parserów w~czasie kompilacji.


\chapter{Implementacja}
\label{ch:implementacja}


\section{Praktyczna implementacja analizatora leksykalnego z~wykorzystaniem makr w~Scali~3}\label{sec:praktyczna-implementacja-analizatora-leksykalnego-z-wykorzystaniem-makr-w-scali-3}

\subsection{Wprowadzenie do studium przypadku}\label{subsec:wprowadzenie-do-studium-przypadku}

Rozdział przedstawia implementację systemu analizy leksykalnej wykorzystującego mechanizmy metaprogramowania Scali~3~\cite{scala3-reference-macros}. Implementacja stanowi studium przypadku zastosowania technik opisanych w~rozdziale poprzednim w~kontekście automatycznej generacji analizatora leksykalnego. System transformuje deklaratywne reguły tokenizacji, wyrażone w~języku dziedzinowym (DSL), w~kod proceduralny wykonywany w~czasie kompilacji, wykorzystując refleksję TASTy~\cite{scala3-reflection} oraz typy rafinowane~\cite{scala3-selectable}.

System \texttt{alpaca.lexer} implementuje transformację deklaratywnych reguł tokenizacji zapisanych jako funkcja częściowa (ang.~\textit{partial function}) w~kod procedualny wykonywany w~czasie kompilacji.
Wykorzystuje przy tym pełne spektrum możliwości refleksji TASTy\cite{scala3-reflection}, włączając generację klas w~czasie kompilacji, transformację drzew AST\cite{scala3-guides-reflection} oraz wyspecjalizowane typy refinement.

\subsection{Architektura systemu leksera}\label{subsec:architektura-systemu-leksera}

\subsubsection{Interfejs użytkownika}\label{subsubsec:interfejs-uzytkownika}

System udostępnia interfejs języka dziedzinowego (DSL) oparty na dopasowaniu wzorców, umożliwiający deklaratywne wyrażenie reguł tokenizacji:

\lstinputlisting[language=scala,caption={Definicja typu LexerDefinition},label={lst:lexer-definition}]{listings/implementation/01-lexer-definition.scala}

Definicja \texttt{LexerDefinition} reprezentuje reguły leksera jako funkcję częściową mapującą wzorce wyrażeń regularnych (jako ciągi znaków) na definicje tokenów.
Wykorzystanie funkcji częściowej pozwala na naturalne wyrażenie reguł leksykalnych w~idiomatycznej składni Scali.

Metoda \texttt{lexer} stanowi główny interfejs systemu:

\lstinputlisting[language=scala,caption={Punkt wejścia: transparent inline def lexer},label={lst:lexer-entrypoint}]{listings/implementation/02-lexer-entrypoint.scala}

Modyfikator \texttt{transparent inline} zapewnia, że zwracany typ będzie dokładnie odpowiadał wygenerowanej strukturze, włączając typy refinement dla poszczególnych tokenów.
Użycie parametrów kontekstowych (\texttt{using}) realizuje wzorzec dependency injection na poziomie systemu typów.

\subsubsection{Implementacja makra}\label{subsubsec:implementacja-makra}

Makro przyjmuje wyrażenie reprezentujące reguły leksera jako \verb|Expr[Ctx ?=> LexerDefinition[Ctx]]| oraz instancje kontekstualnych klas pomocniczych.
Parametr \texttt{using Quotes} dostarcza dostępu do API refleksji TASTy\cite{stucki2020inlining,scala3-guides-quotes,scala3-reference-macros}.

\subsection{Analiza drzewa składni abstrakcyjnej}\label{subsec:analiza-drzewa-skladni-abstrakcyjnej}

\subsubsection{Dekonstrukcja funkcji częściowej}\label{subsubsec:dekonstrukcja-funkcji-czesciowej}

Kluczowym krokiem implementacji jest ekstrakcja reguł z~definicji funkcji częściowej:

\lstinputlisting[language=scala,caption={Dekonstrukcja funkcji częściowej (dopasowanie AST do CaseDef)},label={lst:extract-cases}]{listings/implementation/03-extract-case.scala}

Fragment ten wykorzystuje dopasowanie wzorców w~\textit{quotes} do dekonstrukcji~\cite{scala3-guides-quotes} typowanego AST funkcji częściowej.
Struktura \texttt{Lambda(\_, Match(\_, cases))} odpowiada wewnętrznej reprezentacji funkcji częściowej, gdzie \texttt{Match} zawiera listę przypadków \texttt{CaseDef}.

\subsection{Transformacja i~adaptacja referencji}\label{subsec:transformacja-i-adaptacja-referencji}

\subsubsection{Klasa replacerefs}\label{subsubsec:klasa-replacerefs}

Kluczową techniką jest zastąpienie referencji do starego kontekstu nowymi referencjami:

\lstinputlisting[language=scala,caption={Zastąpienie referencji starego kontekstu nowymi (ReplaceRefs)},label={lst:replace-with-new-ctx}]{listings/implementation/04-replace-refs-with-new-ctx.scala}

Transformacja realizuje proces przepisania właściciela (\textit{re-owning}) symboli w~AST, polegający na modyfikacji referencji kontekstowych w~celu dostosowania ich do nowego zakresu leksykalnego~\cite{scala3-guides-reflection}.
Klasa \texttt{ReplaceRefs} udostępnia \texttt{TreeMap}, który podczas przejścia po AST podmienia referencje do wskazanych symboli na podane termy\cite{scala3-guides-reflection}.

\subsection{Ekstrakcja i~kompilacja wzorców}\label{subsec:ekstrakcja-i-kompilacja-wzorcow}

\subsubsection{Funkcja extractSimple}\label{subsubsec:funkcja-extractsimple}

Funkcja \texttt{extractSimple} implementuje logikę dopasowania różnych typów definicji tokenów:

\lstinputlisting[language=scala,caption={Funkcja extractSimple: dopasowywanie definicji tokenów},label={lst:lexer-08-extract-simple}]{listings/implementation/05-extract-simple.scala}

Wykorzystuje ona dopasowanie wzorców w~\textit{quotes} z~ekstraktorem typów\cite{scala3-guides-quotes}, umożliwiając rozróżnienie różnych wariantów definicji tokenów na poziomie typów.
Konstrukcja \texttt{type t <: ValidName} w~wzorcu wiąże parametr typu do zmiennej wzorca \texttt{t}, umożliwiając jego późniejsze wykorzystanie.

Ekstrakcja definicji tokenów wymaga następnie ich analizy i~walidacji, co realizuje klasa \texttt{CompileNameAndPattern}.

\subsection{Analiza wzorców: klasa CompileNameAndPattern}
\label{subsec:compile-name-pattern}

Klasa \texttt{CompileNameAndPattern} odpowiada za ekstrakcję i~walidację wzorców tokenów podczas ekspansji makra\cite{scala3-reference-macros}.
Jej głównym zadaniem jest transformacja wzorców występujących w~definicjach DSL. Wzorce te są przekształcane w~struktury \texttt{TokenInfo}, które następnie są wykorzystywane do generacji finalnego kodu leksera.

Implementacja wykorzystuje rekurencyjne przetwarzanie drzewa AST z zastosowaniem optymalizacji rekurencji ogonowej (\texttt{@tailrec}), co eliminuje ryzyko przepełnienia stosu dla złożonych wzorców.

\subsection{Generacja klasy anonimowej}\label{subsec:generacja-klasy-anonimowej}

Kluczowym mechanizmem implementacyjnym makra \texttt{lexer} jest programatyczna konstrukcja klasy anonimowej w~czasie kompilacji\cite{stucki2021multistage}.
Proces ten wykorzystuje API refleksji TASTy\cite{scala3-reflection} do dynamicznego tworzenia struktur typów, które następnie są materializowane jako kod bajtowy JVM.

\subsubsection{Konstrukcja symbolu klasy}\label{subsubsec:konstrukcja-symbolu-klasy}

Anonimowa klasa implementująca \texttt{Tokenization[Ctx]} jest tworzona poprzez wywołanie \texttt{Symbol.newClass}:

Metoda \texttt{Symbol.newClass} przyjmuje następujące parametry:
\begin{itemize}
  \item \textbf{Symbol.spliceOwner} — właściciel nowego symbolu w~hierarchii definiowania, zapewniający poprawną widoczność w~zakresie leksykalnym
  \item \textbf{Symbol.freshName(``\$anon'')} — generowanie unikalnej nazwy klasy zgodnie z~konwencją kompilatora Scali dla klas anonimowych
  \item \textbf{List(TypeRepr.of[Tokenization[Ctx]])} — lista typów bazowych, w~tym przypadku pojedyncza implementacja abstrakcyjnej klasy \texttt{Tokenization}
  \item \textbf{decls} — funkcja dostarczająca listy deklaracji członków klasy (pól i~metod)
\end{itemize}

\subsubsection{Definicja członków klasy}\label{subsubsec:definicja-czlonkow-klasy}

Funkcja \texttt{decls} konstruuje pełną listę deklaracji dla klasy anonimowej:

\begin{enumerate}
  \item \textbf{Pola tokenów} — dla każdego zdefiniowanego tokena tworzony jest symbol pola typu \verb|DefinedToken[Name, Ctx, Value]|
  \item \textbf{Type alias Fields} — typ pomocniczy w~formie \texttt{NamedTuple} ułatwiający strukturalny dostęp do tokenów
  \item \textbf{Pole compiled} — wartość typu \texttt{Regex} zawierająca skompilowane wyrażenie regularne dla wszystkich tokenów
  \item \textbf{Pole tokens} — lista wszystkich zdefiniowanych tokenów (włączając ignorowane)
  \item \textbf{Pole byName} — mapa umożliwiająca dynamiczny dostęp do tokenów po nazwie
\end{enumerate}

\subsubsection{Materializacja klasy}\label{subsubsec:materializacja-klasy}

Po zdefiniowaniu symbolu klasy następuje konstrukcja jej ciała.
Klasa jest następnie instancjonowana poprzez wywołanie jej konstruktora.

\subsection{Typy rafinowane (refinement types)}\label{subsec:typy-rafinowane}

Typy rafinowane (\textit{refinement types}) stanowią mechanizm systemu typów Scali umożliwiający dodanie informacji o~strukturze typu w~czasie kompilacji~\cite{scala3-selectable}.
W~kontekście implementacji leksera typy rafinowane pozwalają na dodanie informacji o~polach tokenów bezpośrednio do typu zwracanego przez makro.

\subsubsection{Proces rafinowania typu}\label{subsubsec:proces-rafinowania-typu}

Typ wynikowy jest konstruowany poprzez iteracyjne rafinowanie typu bazowego\cite{scala3-guides-reflection}:


\lstinputlisting[language=scala,caption={Rafinowanie typu wynikowego o~pola tokenów},label={lst:refinements}]{listings/implementation/06-refinements.scala}

Funkcja \texttt{Refinement(tpe, name, memberType)} tworzy nowy typ będący rozszerzeniem typu.
Operacja ta jest wykonywana w~czasie kompilacji i~nie generuje dodatkowego kodu w~czasie wykonania.

\subsubsection{Wynikowy typ}\label{subsubsec:wynikowy-typ}

Wynikowy typ ma formę typu przecięcia (ang.~\textit{intersection type}):

\begin{lstlisting}[language=scala,caption={Wynikowy typ leksera},label={lst:lexer-15-result-type}]
Tokenization[Ctx] & { 
  val TOKEN1: DefinedToken["NAME1", Ctx, Type1]
  val TOKEN2: DefinedToken["NAME2", Ctx, Type2]
  ...
}
\end{lstlisting}

Ten typ reprezentuje wartości będące jednocześnie instancjami \texttt{Tokenization[Ctx]} oraz posiadające określone pola strukturalne (ang.~\emph{computed field names}).

Dostęp do pól tokenów odbywa się poprzez \verb|trait Selectable|. Standardowa implementacja tego mechanizmu, opisana w~dokumentacji~\cite{scala3-selectable}, wprowadza narzut związany z~dynamicznym wyborem nazwy pola (refleksja). W~prezentowanym rozwiązaniu narzut ten jest eliminowany poprzez precyzyjne typowanie strukturalne.
Aby mechanizm \texttt{Selectable} działał poprawnie ze strukturalnymi typami i~nie wymagał refleksji, klasa generowana przez makro musi implementować \verb|type Fields <: NamedTuple.AnyNamedTuple|\cite{scala3-computed-field-names}.
W~naszym podejściu makro generuje definicję \texttt{type Fields} zawierającą wszystkie zdefiniowane tokeny i~ich typy, dzięki czemu:
\begin{itemize}
  \item IDE i~kompilator dysponują informacją o~dostępnych polach i~ich typach (pełne uzupełnianie i~sprawdzanie typów),
  \item wywołanie \texttt{c.NAZWA} jest bezpieczne typowo mimo mechanizmu dynamicznego wyboru nazwy.
\end{itemize}


\lstinputlisting[language=scala,caption={Tworzenie typuFields},label={lst:fields}]{listings/implementation/07-fields.scala}


\subsection{Uzasadnienie wybranego podejścia implementacyjnego}\label{subsec:uzasadnienie-wybranego-podejscia}

\subsubsection{Eliminacja narzutu wykonania w~czasie działania programu}\label{subsubsec:eliminacja-narzutu-wykonania}

Wszystkie definicje tokenów są rozwiązywane statycznie w~czasie kompilacji\cite{stucki2020inlining}.
Dostęp do tokenów realizowany jest jako bezpośrednie odwołanie do pola klasy, które w~kodzie bajtowym JVM~\cite{lindholm2014java} reprezentowane jest przez instrukcję \texttt{getfield} o~złożoności czasowej~O(1). Teoretycznie eliminuje to narzut związany z~operacjami dynamicznymi, choć pełna weryfikacja empiryczna tego założenia wykracza poza zakres niniejszej pracy.

Alternatywne podejście oparte na strukturze mapującej (np.\ \texttt{Map[String, Token]}) wymagałoby:
\begin{itemize}
  \item Obliczenia funkcji haszującej dla klucza
  \item Przeszukiwania tablicy haszującej
  \item Potencjalnej obsługi kolizji
  \item Dynamicznego rzutowania typu
\end{itemize}

co wprowadzałoby znaczący narzut wydajnościowy oraz eliminowało możliwość optymalizacji przez kompilator.

\subsubsection{Bezpieczeństwo typów na poziomie systemu}\label{subsubsec:bezpieczenstwo-typow}

Dzięki typom rafinowanym każdy token posiada precyzyjny typ znany kompilatorowi\cite{scala3-selectable}.
System typów weryfikuje poprawność wszystkich operacji w~czasie kompilacji, eliminując możliwość błędów związanych z~niepoprawnym typowaniem wartości tokenów.

\subsubsection{Integracja z~narzędziami deweloperskimi}\label{subsubsec:integracja-z-narzedzimi}

Ponieważ tokeny są reprezentowane jako rzeczywiste pola w~typie, środowiska deweloperskie (IDE) mogą wykorzystać informacje typu do:
\begin{itemize}
  \item Automatycznego uzupełniania nazw tokenów
  \item Prezentacji pełnych sygnatur typów przy najechaniu kursorem
  \item Nawigacji do definicji przez mechanizm \textit{go-to-definition}
  \item Wykrywania błędów składniowych przed kompilacją
\end{itemize}

Te funkcjonalności są niemożliwe do realizacji w~przypadku dostępu przez struktury dynamiczne.

\subsubsection{Statyczna detekcja konfliktów wzorców}\label{subsubsec:statyczna-detekcja-konfliktow}

Makro przeprowadza analizę wszystkich wzorców w~czasie kompilacji, wykrywając potencjalne konflikty nakładających się wyrażeń regularnych.
Mechanizm ten zapewnia, że błędy konfiguracji są wykrywane na etapie kompilacji, a~nie w~czasie wykonania programu, co jest zgodne z~zasadą \textit{fail-fast} w~inżynierii oprogramowania.

\subsubsection{Typowanie strukturalne z~gwarancjami nominalnymi}\label{subsubsec:typowanie-strukturalne}

Zastosowanie typów rafinowanych\cite{scala3-selectable} łączy zalety typowania strukturalnego (elastyczność w~dostępie do składowych) z~bezpieczeństwem typowania nominalnego (jednoznaczna identyfikacja typów).
Każde pole w~typie rafinowanym ma precyzyjny typ nominalny, podczas gdy dostęp do tych pól odbywa się przez nazwę, co zapewnia elastyczność interfejsu.

\subsection{Analiza alternatywnych rozwiązań}\label{subsec:analiza-alternatywnych-rozwiazan}

\subsubsection{Podejście oparte na mapowaniu dynamicznym}\label{subsubsec:podejscie-mapowanie-dynamiczne}

Alternatywne podejście mogłoby wykorzystywać strukturę mapującą do przechowywania tokenów:

\begin{lstlisting}[language=scala,caption={Podejście oparte na mapowaniu dynamicznym}]
class SimpleLexer {
  val tokens: Map[String, Token[?, ?, ?]] = Map(
    "NUMBER" -> ...,
    "PLUS" -> ...
  )
  def apply(name: String): Token[?, ?, ?] = tokens(name)
}
\end{lstlisting}

\textbf{Wady tego podejścia:}
\begin{itemize}
  \item Brak bezpieczeństwa typów: błędne nazwy tokenów wykrywane są dopiero w~czasie wykonania
  \item Utrata informacji o~typach: zwracany typ to egzystencjalny \texttt{Token[?, ?, ?]}
  \item Narzut wydajnościowy operacji haszowania i~przeszukiwania
  \item Brak wsparcia narzędzi deweloperskich
\end{itemize}

\subsubsection{Podejście oparte na jawnej definicji klasy}\label{subsubsec:podejscie-jawna-definicja}

Innym rozwiązaniem byłoby jawne definiowanie klasy leksera przez użytkownika:

\begin{lstlisting}[language=scala,caption={Podejście oparte na jawnej definicji klasy}]
class MyLexer extends Tokenization[DefaultGlobalCtx] {
  val NUMBER = DefinedToken[...]
  val PLUS = DefinedToken[...]
  protected def compiled: Regex = "(?<token0>[0-9]+)|(?<token1>\\+)".r
  // ...
}
\end{lstlisting}

\textbf{Wady tego podejścia:}
\begin{itemize}
  \item Wysoki poziom redundancji kodu (\textit{boilerplate})
  \item Konieczność ręcznej kompilacji wyrażeń regularnych
  \item Podatność na błędy synchronizacji między definicjami tokenów a~wyrażeniem regularnym
  \item Brak mechanizmu DSL ułatwiającego definicję reguł
\end{itemize}

\subsection{Walidacja i~obsługa błędów}\label{subsec:walidacja-i-obsuga-bedow}

\subsubsection{Walidacja wzorców regularnych}\label{subsubsec:walidacja-wzorcow-regularnych}

System wykorzystuje pomocniczą klasę \texttt{RegexChecker} do walidacji wzorców:
Mechanizm ten sprawdza poprawność składni wyrażeń regularnych już w~czasie kompilacji i~raportuje błędy z~dokładną lokalizacją wzorca.
Metoda \texttt{report.errorAndAbort} przerywa kompilację i~wyświetla komunikat o~błędzie, eliminując konieczność detekcji błędów w~czasie wykonania, co jest zgodne z~zasadą wczesnej walidacji (\textit{fail-fast})~\cite{scala3-reference-macros,scala3-guides-macros}.

\subsubsection{Obsługa nieobsługiwanych konstrukcji}\label{subsubsec:obsuga-nieobsugiwanych-konstrukcji}

Kod jawnie sygnalizuje nieobsługiwane przypadki:
Obsługiwane są wyłącznie jasno zdefiniowane formy wzorców; w~przypadku napotkania innej konstrukcji kompilacja jest przerywana z~komunikatem zawierającym szczegóły AST, co upraszcza diagnostykę i~utrzymuje zasadę fail-fast.
Ta strategia jest zgodna z~zasadą fail-fast - lepiej jest wyraźnie odrzucić nieobsługiwane konstrukcje niż milcząco generować niepoprawny kod.



\chapter{Algorytmy analizy leksykalnej}
\label{ch:lexer-algorithmic}


\section{Teoretyczne podstawy}
\label{sec:lexer-theory}

Analizator leksykalny (lekser) stanowi fundamentalną fazę przetwarzania tekstu źródłowego, przekształcając sekwencję znaków w~ciąg jednostek leksykalnych (tokenów), które reprezentują niepodzielne elementy składniowe dla fazy analizy składniowej~\cite{cornell-lexing}.
Klasyczna konstrukcja analizatora opiera się na~połączeniu teorii formalnych języków oraz teorii automatów skończonych~\cite{lexical-analysis-wiki}.

\subsection{Opis języka tokenów}
\label{subsec:lexer-theory-lang}
Każda klasa tokenów jest definiowana poprzez język regularny: zbiór słów akceptowanych przez wyrażenie regularne.
Zbiór reguł tokenów stanowi sumę języków regularnych; ich unia~jest również językiem regularnym~\cite{jezyki-formalne-automaty}, co~umożliwia kompilację ich do~jednolitego automatu deterministycznego.

\subsection{Automaty skończone}
\label{subsec:lexer-theory-dfa}
Wyrażenia regularne są transformowane do~postaci niedeterministycznej (NFA) poprzez konstrukcję Thompsona~\cite{arces2018thompson}.
Następnie deterministyczna postać automatu (DFA) konstruowana jest poprzez algorytm usuwania niedeterminizmu (ang.~\emph{powerset construction}), polegający na~iteracyjnym łączeniu zbiorów stanów NFA\@.
Opcjonalnie przeprowadza się minimalizację automatu poprzez usuwanie stanów równoważnych~\cite{jezyki-formalne-automaty}.

DFA przetwarza wejście znak po znaku, zachowując jednoznaczny stan aktywny oraz informując, czy aktualny prefiks odpowiada jednemu z~zdefiniowanych tokenów.

\subsection{Strategia wyboru dopasowania}
\label{subsec:lexer-theory-matching}
Lekser stosuje dwie komplementarne zasady determinujące zachowanie dla wieloznacznych sytuacji:

\begin{itemize}
    \item Dopóki DFA ma ścieżkę przejść, znak jest konsumowany; token jest emitowany dopiero po~ostatnim stanie akceptującym widzianym na tej ścieżce (najdłuższe dopasowanie, ang. \textit{maximal munch}).
    \item Gdy kilka reguł akceptuje prefiks o~tej samej długości, wybierana jest reguła o najwyższym priorytecie (często określanym kolejnością definicji).
\end{itemize}

Kombinacja tych zasad gwarantuje deterministyczną oraz reproducywalną sekwencję tokenów bez~konieczności specjalnych mechanizmów rozstrzygania konfliktów.

\subsection{Błędy leksykalne}
\label{subsec:lexer-theory-errors}
W~sytuacji, gdy automat nie posiada przejścia dla bieżącego znaku, ogłaszany jest błąd leksykalny w~bieżącej pozycji wejścia.
Mechanizm diagnostyczny przywołuje informacje o~pozycji znaku, co~znacząco ułatwia śledzenie źródła problemu w~tekście źródłowym.


\section[Automaty DFA a wyrażenia regularne w systemie ALPACA]{Automaty DFA a wyrażenia regularne w~systemie ALPACA}
\label{sec:dfa-vs-regex}

\subsection{Tło: Tradycyjne podejście}
\label{subsec:traditional-dfa-approach}

Narzędzia klasyczne do~budowy leksera (takie jak~Lex~\cite{lesk1975lex}) generują jawny, deterministyczny automat skończony (DFA) z~zestawu wyrażeń regularnych.
Podejście to wymaga implementacji pełnego zestawu algorytmów: transformacja NFA$\to$DFA, minimalizacja, optymalizacja — każdy etap jest czasochłonny dla twórcy narzędzia.

\subsection{Alternatywa: Wyrażenia regularne biblioteczne}
\label{subsec:regex-alternative}

W~systemie ALPACA podejście tradycyjne zostało zastąpione mechanizmem wykorzystującym natywny silnik wyrażeń regularnych biblioteki standardowej Scali~\cite{scala-regex-api}.
Zamiast ręcznie kodować DFA, makro kompilacyjne łączy wszystkie wzorce operatorem alternatywy (\texttt{|}) w~jedno wyrażenie regularne o~nazwanych grupach, umożliwiając rozróżnienie, która reguła dopasowała się podczas każdego przebiegu.

\subsection{Zalety podejścia opartego na wyrażeniach regularnych}
\label{subsec:pattern-matching-pros}
\begin{itemize}
    \item Poprzez wykorzystanie powszechnie znanego mechanizmu wyrażeń regularnych, użytkownicy języka specjalistycznego (DSL) mogą definiować reguły leksykalne bez~konieczności zrozumienia złożoności konstrukcji automatu i~algorytmów optymalizacji.
    \item Użytkownicy mogą zastosować rozszerzenia wyrażeń regularnych (takie jak~\emph{backreference}, \emph{negative lookahead}, czy~warunkowość), których implementacja byłaby niemożliwa w~jawnym DFA\@.
    \item Własna implementacja DFA wymaga pokrycia pełnego spektrum funkcjonalności wyrażeń regularnych, ciągłego utrzymania w~synchronizacji z~ewolucją języka hosta, oraz inwestycji w~optymalizację.
    ALPACA deleguje ten wysiłek do~zoptymalizowanego i~wielokrotnie przetestowanego silnika bibliotecznego.
    \item Definicje reguł leksykalnych pozostają kompaktowe i~czytelne, co~ułatwia przegląd, weryfikację i~modyfikację.
\end{itemize}

\subsection{Wady i ograniczenia}
\label{subsec:pattern-matching-cons}

\begin{itemize}
    \item Silniki wyrażeń regularnych mogą wykazywać wyższą złożoność obliczeniową niż ręcznie zoptymalizowane DFA, zwłaszcza w~przypadku dużych zbiorów reguł lub~złożonych wzorców.
    \item Zachowanie silnika wyrażeń regularnych dla dwuznacznych sytuacji (na~przykład gdy dwa wzorce różnej długości akceptują identyczną sekwencję) zależy od~implementacji silnika.
    W~niekorzystnych przypadkach może prowadzić do~nieoczekiwanych wyborów.
    Rozwiązaniem jest jawna deklaracja priorytetu poprzez kolejność definiowania reguł.
\end{itemize}

\subsection{Syntetyzacja: Decyzja projektu ALPACA}
\label{subsec:alpaca-design-choice}

Powyższe rozważania doprowadziły do~wyboru wyrażeń regularnych nad jawną konstrukcją DFA\@.
Wymiana wydajności (potencjalnie) za~uproszczenie interfejsu jest~akceptowalna w~kontekście systemu ALPACA\@.


\section[Praktyczna implementacja leksera w systemie ALPACA]{Praktyczna implementacja leksera w~systemie ALPACA}
\label{sec:lexer-impl}

Implementacja modułu \texttt{alpaca.lexer} łączy ergonomię języka specjalistycznego (DSL) z~kodem wykonywany w~czasie kompilacji, eliminując narzut parsowania wyrażeń regularnych w~czasie działania aplikacji.

\subsection[Przebieg tokenizacji w ALPACA]{Przebieg tokenizacji w~ALPACA}
\label{subsec:lexer-impl-flow}

Podczas kompilacji projektu wszystkie wzorce są łączone operatorem alternatywy (\texttt{|}) w~jedno wyrażenie regularne z~nazwanymi grupami.
Mechanizm ten umożliwia rozróżnienie, która z~reguł dopasowała się podczas każdego przebiegu wyszukiwania.
Następnie pętla skanująca—podczas wykonania aplikacji iteracyjnie wywołuje to~wyrażenie na~kolejnych fragmentach wejścia, identyfikuje dopasowany token, akumuluje leksemy oraz przesuwa wskaźnik wejścia aż do~wyczerpania danych.

\subsection{Obsługa reguł ignorowanych}
\label{subsec:lexer-impl-ignored}

System ALPACA umożliwia oznaczenie reguł jako ,,ignorowanych''.
Takie reguły są wbudowywane we wspólny wzorzec, jednak ich dopasowania nie~generują tokenów wyjściowych.
Ujednolicony przebieg pętli skanującej upraszcza kod: ignorowane tokeny różnią się od~normalnych wyłącznie akcją podejmowaną po~dopasowaniu (brak emisji leksemu, zamiast tego aktualizacja stanu wewnętrznego).

\subsection[Stanowa analiza leksykalna i rozszerzenia kontekstu]{Stanowa analiza leksykalna i~rozszerzenia kontekstu}
\label{subsec:lexer-impl-context}
Możliwa jest stanowa analiza leksykalna poprzez utrzymywanie stanu maszyny stanów w~obiekcie kontekstu.
System ALPACA wewnętrznie definiuje mechanizm \texttt{BetweenStages}, który jest wywoływany po~każdym rozpoznaniu leksemu i~umożliwia modyfikację stanu kontekstu.
Domyślna implementacja rejestruje ostatni leksem oraz śledzi numer linii i~kolumny; użytkownik może jednak rozszerzyć tę~logikę o~własne zachowania, na~przykład weryfikację poprawności zagnieżdżenia nawiasów.

\subsection{Diagnostyka błędów leksykalnych}
\label{subsec:lexer-impl-errors}
Gdy algorytm skanowania nie~odnajduje prefiksu (brak przejścia w~automacie), lekser zgłasza błąd leksykalny.
Mechanizm \texttt{BetweenStages} umożliwia zbieranie danych kontekstowych (numer linii, kolumny, ostatni prawidłowy leksem), które następnie wzbogacają raport diagnostyczny.
To~podejście znacząco poprawia doświadczenie użytkownika podczas debugowania błędów składniowych.

\subsection{Strumieniowe przetwarzanie wejścia}
\label{subsec:lexer-impl-lazy-reader}
W~celu unikania nadmiernego zużycia pamięci, lekser analizuje wejście w~sposób strumieniowy poprzez implementację interfejsu \texttt{CharSequence}.
Klasa \texttt{LazyReader} realizuje tę~funkcjonalność: pobiera dane ze~źródła w~blokach o~rozmiarze 16~KB (ang.~\emph{chunks}) i~buforuje je lokalnie.
Metoda \texttt{ensure} zapewnia, że~żądana pozycja jest~dostępna w~buforze, czytając kolejne porcje danych w~razie potrzeby.

\lstinputlisting[language=scala,caption={Implementacja metody \texttt{ensure} w~klasie \texttt{LazyReader}},label={lst:lazy-reader-ensure-impl}]{listings/chapter4/lazy-reader-ensure-impl.scala}

\subsection{Wczesna walidacja wzorców}
\label{subsec:lexer-impl-regexchecker}
Przed wygenerowaniem automatu skanującego, makro kompilacyjne uruchamia moduł \texttt{RegexChecker}, który analizuje wzorce pod~względem potencjalnych konfliktów.
Mechanizm ten wykrywa dwa klasy problemów:
\begin{itemize}
    \item Subsumpcję --- sytuację, w~której każde słowo akceptowane przez późniejszy wzorzec jest~również akceptowane przez wcześniejszy wzorzec, czyniąc~późniejszy \enquote{martwym kodem}.
    Przykład: jeśli definiuje się \texttt{ID~=~[a-z]+}, a~następnie \texttt{KEYWORD~=~if|then}, to~wszystkie słowa kluczowe będą dopasowane przez~\texttt{ID}, co~uniemożliwi rozpoznanie \texttt{KEYWORD}.
    \item Pokrycie prefiksów --- sytuację, w~której jeden wzorzec akceptuje prefiks słowa akceptowanego przez inny wzorzec.
    Przykład: jeśli zdefiniowano \texttt{LT~=~<} oraz \texttt{LE~=~<}\texttt{=}, lekser może dopasować \texttt{<}, a~następnie zgłosić błąd leksykalny dla~pozostałego znaku \texttt{=}, zamiast rozpoznać \texttt{<}\texttt{=}.
    Jest to konsekwencja użycia silnika wyrażeń regularnych, w którym zasada najdłuższego dopasowania (\textit{maximal munch}) nie ma zastosowania.
\end{itemize}

Wykrycie któregokolwiek z~powyższych problemów powoduje przerwanie kompilacji z~czytelnym i~działającym komunikatem diagnostycznym, dzięki~czemu konfiguracyjne błędy są~eliminowane przed~czasem wykonania programu.


\chapter{Algorytmy analizy składniowej}
\label{ch:parser-algorithmic}

\section{Teoretyczne podstawy działania parserów}
\label{sec:parser-theory}

Parser przekształca strumień tokenów w strukturę składniową (AST), rozstrzygając zgodność wejścia z gramatyką. Klasyczne podejścia bazują na automatach skończonych rozszerzonych o stos (PDA) oraz na algorytmach predykcyjnych lub analizie przesuwająco-redukcyjnej.

\subsection{Gramatyki i typy parserów}
Gramatyka bezkontekstowa (CFG) składa się z nieterminali, terminali (tokenów), symbolu startowego i produkcji. Parsery dzieli się na:
\begin{itemize}
    \item \textbf{LL(k)} (zstępujące, predykcyjne) — konstruują lewostronne wyprowadzenia, wybierając produkcje na podstawie prefiksu wejścia (lookahead). Wymagają gramatyk bez lewostronnej rekurencji i z niekolidującymi zbiorami FIRST/FOLLOW.
    \item \textbf{LR(k)} (wstępujące, shift-reduce) — rekonstruują prawostronne wyprowadzenia wstecz, używając stosu stanów automatu LR. Radzą sobie z większą klasą gramatyk, w tym lewostronnie rekurencyjnych.
\end{itemize}

\subsection{Automat ze stosem}
Parser można modelować jako deterministyczny automat ze stosem (PDA): stan określa pozycję w tablicach parsera, stos przechowuje nieterminale/stan, a wejście dostarcza tokeny. Przejścia to operacje \textit{shift} (przesunięcie tokenu na stos) i zastąpienie prawej strony produkcji nieterminalem i przeskok do nowego stanu (\textit{reduce}).

\subsection{Tabele sterujące}
Parsery tabelowe (LL i LR) działają w stałym czasie na token. Tabela akcji LR mapuje parę (stan, lookahead) na \textit{shift}, \textit{reduce}, \textit{accept} lub \textit{error}; tabela goto określa przejścia po redukcjach. W LL analogiczną rolę pełni tabela predykcji (nieterminal × lookahead → produkcja).

\subsection{Rozstrzyganie konfliktów}
Konflikty \textit{shift/reduce} i \textit{reduce/reduce} pojawiają się, gdy tabela parsera jest niedeterministyczna.
Najczęściej rozwiązuje się je przez zmianę gramatyki lub zwiększenie lookaheadu.

W parserach LL problemy zwykle wynikają z lewostronnej rekurencji i wspólnych prefiksów (konieczna
lewostronna faktoryzacja). W parserach LR konflikty najczęściej biorą się ze zbliżonych prefiksów wielu produkcji
(np.\ \texttt{if--then} vs.\ \texttt{if--then--else}) oraz z niejednoznaczności w wyrażeniach
arytmetycznych.

\section{Wybór klasy parsera w Alpaca}
\label{sec:parser-lr1-choice}

Alpaca generuje parsery klasy LR(1), czyli deterministyczne analizatory korzystające
z pełnego jednego tokena lookaheadu oraz kanonicznych stanów LR. Taki wybór łączy
wysoką moc wyrazu (obsługa lewostronnej rekurencji, gramatyk o złożonej składni)
z przewidywalnością działania i stabilnym czasem parsowania.

\subsection{Zalety LR(1) w Alpaca}
\begin{itemize}
    \item LR(1) obejmuje znacznie więcej konstrukcji niż LL(k)
    i unika wymuszonych przeróbek gramatyki (eliminacji lewej rekurencji, agresywnej lewostronnej faktoryzacji).
    Dzięki temu specyfikacja pozostaje bliższa naturalnej formie języka.

    \item Pełny lookahead eliminuje typowe dla SLR i LALR konflikty,
    które wynikają z nadmiernie szerokich FOLLOW-ów. Dzięki temu w gramatyce pojawia się mniej konfliktów.

    \item Stany LR(1) pozwalają jasno wskazać, jaki był oczekiwany token
    w danym miejscu; pozwala to na bardziej precyzyjne komunikaty dotyczące błędów składniowych.

    \item Każdy krok analizatora to pojedynczy dostęp do tabeli (\texttt{shift/reduce}),
    co zapewnia stabilną wydajność liniową dla całego wejścia.

    \item Każda redukcja LR(1) jednoznacznie
    odpowiada konkretnej produkcji, oraz ma dostęp do terminali które ją tworzą,
    więc akcje semantyczne mogą bezpośrednio tworzyć węzły AST.
\end{itemize}

\subsection{Koszty i kompromisy}
\begin{itemize}
    \item Kanoniczne LR(1) może generować znacznie więcej stanów niż SLR/LALR.

    \item Obliczanie zbiorów closure i goto dla LR(1) jest bardziej
    kosztowne niż w uproszczonych wariantach, co wpływa na czas kompilacji i rozmiar generatora.

    \item Parser LR(1) jest deterministyczny składniowo, ale nie semantycznie; wciąż trzeba
    rozwiązywać niejednoznaczności (np. ``dangling else'') poprzez jawne reguły priorytetów.

    \item Choć klasa jest szeroka,
    nadal istnieją konstrukcje wymagające przekształceń.
\end{itemize}

\section{Konstrukcja tablic parsera LR(1) w Alpaca}
\label{sec:parser-impl-overview}

Wygenerowanie parsera LR(1) w Alpaca to sekwencja algorytmów wykonywanych w czasie kompilacji. Poniżej opisano, krok po kroku, jak deklaratywna gramatyka staje się deterministyczną tabelą akcji.

\subsection{Wyznaczanie zbiorów FIRST}
Na bazie produkcji wyliczany jest \texttt{FirstSet} (iteracyjnie do punktu stałego), który określa możliwe terminale po kropce. Algorytm przechodzi po wszystkich produkcjach, dodając:
\begin{itemize}
    \item pierwszy terminal z prawej strony produkcji,
    \item w przypadku nieterminali — wszystkie terminale z ich FIRST (z wyjątkiem $\varepsilon$), a jeśli mogą generować $\varepsilon$, iteruje dalej po kolejnych symbolach produkcji,
    \item $\varepsilon$ dla produkcji pustych.
\end{itemize}

\lstinputlisting[language=scala,caption={Implementacja metody addImports używanej podczas wyznaczania zbiorów FIRST},label={lst:add-imports-impl}]{listings/chapter5/add-imports-impl.scala}

Pętla powtarza się, aż zbiory FIRST przestaną się zmieniać (punkt stały)

\subsection{Budowa automatów LR(1)}
Stan początkowy to domknięcie elementu \texttt{S' → • root, \$}. Kolejne stany są
budowane klasycznym schematem \textit{closure}/\textit{goto} i deduplikowane, tak aby
otrzymać deterministyczny automat LR(1).

\subsubsection{Funkcja \texttt{closure}}
Dla elementu z nieterminalem po kropce (\(A \to \alpha \,\bullet\, B \beta,\, a\))
algorytm:
\begin{itemize}
    \item wylicza zbiór lookaheadów \(FIRST(\beta a)\), tj.\ \(FIRST(\beta)\)
          bez \(\varepsilon\) oraz, jeśli \(\varepsilon \in FIRST(\beta)\), dodaje także \(a\),
    \item dla każdego takiego lookaheadu \(x\) dodaje elementy
          \(B \to \bullet \gamma, x\) dla wszystkich produkcji \(B \to \gamma\),
    \item rekurencyjnie domyka nowo dodane elementy, dopóki dodawanie kolejnych produkcji
          nie generuje nowych elementów (closure osiąga punkt stały).
\end{itemize}
W efekcie stan zawiera pełen zbiór ,,przewidywań'' dla wszystkich nieterminali,
które mogą pojawić się w tej pozycji po kropce.

\lstinputlisting[language=scala,caption={Implementacja funkcji `closure`},label={lst:lr1-closure}]{listings/chapter5/lr1-closure.scala}

\subsubsection{Funkcja \texttt{goto}}
Dla zadanego stanu i symbolu \(s\) po kropce, funkcja \texttt{goto} przesuwa kropkę
we wszystkich elementach z tym symbolem i domyka wynik, korzystając z produkcji
i wcześniej policzonych zbiorów FIRST.

\lstinputlisting[language=scala,caption={Implementacja funkcji `goto`},label={lst:lr1-goto}]{listings/chapter5/lr1-goto.scala}

\subsubsection{Główny algorytm budowy automatów LR(1)}
Konstrukcja automatów LR(1) rozpoczyna się od stanu początkowego i iteracyjnie dodaje nowe stany, dopóki wszystkie przejścia nie zostaną odkryte. Proces ten można podzielić na cztery główne etapy:
\begin{itemize}
    \item Domknięcie elementu \texttt{S' → • root, \$} tworzy stan~0.
    \item Dla każdego stanu zbierane są symbole, które mogą wystąpić po kropce;
          dla każdego takiego symbolu wyliczany jest stan docelowy funkcją \texttt{goto}. Jeśli
          stan docelowy już istnieje, do tabeli dopisywany jest \textit{shift} do jego ID; w przeciwnym
          razie stan otrzymuje nowe ID, jest dodawany do listy stanów, a \textit{shift} wskazuje na niego.
    \item Elementy z kropką na końcu produkcji dodają akcje
          \textit{reduce} dla swoich lookaheadów; w szczególnym przypadku element
          \texttt{S' → root •, \$} generuje akcję \textit{accept} dla symbolu \texttt{\$}.
    \item Algorytm powtarza te kroki, dopóki nie zostaną przetworzone
          wszystkie utworzone stany; identyczne zestawy elementów nie tworzą nowych stanów,
          dzięki czemu automat pozostaje deterministyczny.
\end{itemize}

\lstinputlisting[language=scala,caption={Główny algorytm budowy automatów LR(1)},label={lst:lr1-build}]{listings/chapter5/lr1-build.scala}

Stany są kolekcjami \texttt{Item}-ów przechowywanymi w posortowanym zbiorze, więc są porównywane
strukturalnie po zawartości, a nie po referencji.

% TODO: implement error resolution cycle identification and cover the the whole topic here


\chapter[Analiza porównawcza z istniejącymi rozwiązaniami]{Analiza porównawcza z~istniejącymi rozwiązaniami}
\label{ch:comp}


\section{Wprowadzenie do badań porównawczych}
\label{sec:comp-intro}

Weryfikacja empiryczna tezy pracy wymaga systematycznego porównania systemu~\emph{ALPACA} z~reprezentatywnymi rozwiązaniami istniejącymi na~rynku.
W~tym celu przeprowadzono serię testów wydajnościowych (ang.~\emph{benchmarks}), których metodologia, wyniki oraz interpretacja zostaną przedstawione w~niniejszym rozdziale.

Wybór narzędzi porównawczych wynikał z~analizy przeprowadzonej w~sekcji~\ref{sec:intro-review}.
Jako rozwiązania reprezentatywne dla głównych kategorii wybrano:

\begin{itemize}
    \item \textbf{SLY}~\cite{sly} --- bibliotekę dla języka Python, reprezentującą podejście oparte na~refleksji i~dynamicznym typowaniu (kategoria bibliotek w językach interpretowanych),
    \item \textbf{FastParse}~\cite{fastparse-docs} --- bibliotekę kombinatorów parserów dla Scali, reprezentującą nowoczesne podejście do~budowy parserów w~językach statycznie typowanych (kategoria kombinatorów parserów).
\end{itemize}

Wybór tych narzędzi uzasadniony jest ich reprezentatywnością dla różnych paradygmatów konstrukcji parserów oraz dostępnością dokumentacji i~aktywnym wsparciem społeczności.
Pominięto generatory kodu takie jak ANTLR ze~względu na~różnice w~architekturze (zewnętrzny krok generacji vs.~metaprogramowanie w~czasie kompilacji), które utrudniają bezpośrednie porównanie wydajności.


\section{Metodologia badań}
\label{sec:comp-methodology}

\subsection{Środowisko testowe}
\label{subsec:comp-env}

Wszystkie testy zostały przeprowadzone w~kontrolowanym środowisku o~następujących parametrach:

\begin{itemize}
    \item \textbf{Procesor:} Apple M4 Pro, 14 rdzeni
    \item \textbf{Pamięć RAM:} 48 GB
    \item \textbf{System operacyjny:} macOS Sequoia 15.6
    \item \textbf{JVM:} OpenJDK 21.0.8
    \item \textbf{Python:} 3.13.5
    \item \textbf{Scala:} 3.8.0-RC1
\end{itemize}

Przed każdym pomiarem wykonywano fazę rozgrzewki (ang.~\emph{warmup}) składającą się z~trzech iteracji, co~pozwalało na~optymalizację kodu przez kompilator JIT~\cite{lindholm2014java} oraz stabilizację pamięci podręcznej procesora.
Każdy test był powtarzany 10~razy, a~wyniki uśredniano w~celu redukcji wpływu losowych wahań wydajności.

\subsection{Charakterystyka danych testowych}
\label{subsec:comp-data}

Testy wydajnościowe przeprowadzono na~dwóch rodzajach gramatyk: prostej gramatyce wyrażeń arytmetycznych oraz gramatyce formatu JSON\@.
Dla każdej gramatyki przygotowano dwa typy danych wejściowych różniące się strukturą:

\subsubsection{Dane iteracyjne}
\label{subsubsec:comp-data-iter}

Dane iteracyjne charakteryzują się płaską strukturą składniową bez głębokiego zagnieżdżenia.
Dla wyrażeń arytmetycznych przyjęto formę ciągu operacji:

\lstinputlisting[language=terminal,caption={Struktura iteracyjnych danych testowych dla wyrażeń arytmetycznych},label={lst:comp-iter-math}]{listings/comparison/iterative-math-structure.txt}

Dla formatu JSON dane iteracyjne reprezentują tablicę obiektów o~stałej głębokości zagnieżdżenia:

\lstinputlisting[language=json,caption={Struktura iteracyjnych danych testowych dla JSON},label={lst:comp-iter-json}]{listings/comparison/iterative-json-structure.json}

\subsubsection{Dane rekurencyjne}
\label{subsubsec:comp-data-recur}

Dane rekurencyjne charakteryzują się głębokim zagnieżdżeniem struktur składniowych, co~stanowi wyzwanie dla parserów ze~względu na~wykorzystanie stosu wywołań.
Dla wyrażeń arytmetycznych przyjęto formę zagnieżdżonych nawiasów:

\lstinputlisting[language=terminal,caption={Struktura rekurencyjnych danych testowych dla wyrażeń arytmetycznych},label={lst:comp-recur-math}]{listings/comparison/recursive-math-structure.txt}

Dla formatu JSON dane rekurencyjne reprezentują głęboko zagnieżdżone obiekty:

\lstinputlisting[language=json,caption={Struktura rekurencyjnych danych testowych dla JSON},label={lst:comp-recur-json}]{listings/comparison/recursive-json-structure.json}

\subsection{Scenariusze testowe}
\label{subsec:comp-scenarios}

Pomiary przeprowadzono dla rozmiarów danych wejściowych: 100, 500, 1000 oraz 2000~elementów (linii lub obiektów w~przypadku JSON).

Dla każdego rozmiaru mierzono następujące metryki:

\begin{itemize}
    \item \textbf{Czas leksykalizacji} (ang.~\emph{lex time}) --- czas przekształcenia tekstu wejściowego w~strumień tokenów (dotyczy ALPACA i~SLY, które wyróżniają fazę leksykalizacji),
    \item \textbf{Czas parsowania} (ang.~\emph{parse time}) --- czas analizy składniowej strumienia tokenów,
    \item \textbf{Całkowity czas przetwarzania} (ang.~\emph{full parse time}) --- łączny czas leksykalizacji i~parsowania.
\end{itemize}

FastParse realizuje leksykalizację i~parsowanie w~jednym przebiegu, dlatego dla tej biblioteki mierzono wyłącznie całkowity czas przetwarzania.


\section{Implementacja parserów testowych}
\label{sec:comp-impl}

W~celu zapewnienia porównywalności wyników, dla każdego narzędzia zaimplementowano funkcjonalnie równoważne parsery wyrażeń arytmetycznych oraz formatu JSON\@.
Wszystkie implementacje realizują bezpośrednią ewaluację (obliczenie wartości wyrażenia) zamiast konstrukcji drzewa AST, co~eliminuje różnice wydajnościowe wynikające ze~strategii alokacji pamięci.

\subsection[Implementacja w ALPACA]{Implementacja w~ALPACA}
\label{subsec:comp-impl-alpaca}

Parser wyrażeń arytmetycznych w~systemie~\emph{ALPACA} wykorzystuje deklaratywny interfejs DSL oparty na~dopasowaniu wzorców:

\lstinputlisting[language=scala,caption={Parser wyrażeń arytmetycznych w~systemie ALPACA},label={lst:comp-alpaca-math},escapeinside={(*@}{@*)}]{listings/comparison/alpaca-math-parser.scala}

\subsection[Implementacja w SLY]{Implementacja w~SLY}
\label{subsec:comp-impl-sly}

Odpowiadająca implementacja w~bibliotece SLY wykorzystuje dekoratory i~refleksję nazw metod:

\lstinputlisting[language=python,caption={Parser wyrażeń arytmetycznych w~bibliotece SLY},label={lst:comp-sly-math}]{listings/comparison/sly-math-parser.py}

\subsection[Implementacja w FastParse]{Implementacja w~FastParse}
\label{subsec:comp-impl-fastparse}

FastParse realizuje parsowanie poprzez kompozycję funkcji parserowych:

\lstinputlisting[language=scala,caption={Parser wyrażeń arytmetycznych w~bibliotece FastParse},label={lst:comp-fastparse-math}]{listings/comparison/fastparse-math-parser.scala}

Charakterystyczną cechą FastParse jest brak wydzielonej fazy leksykalizacji --- tokenizacja i~parsowanie są realizowane w~jednym przebiegu poprzez kompozycję parserów elementarnych.


\section{Wyniki badań}
\label{sec:comp-results}

\subsection{Wyrażenia arytmetyczne --- dane iteracyjne}
\label{subsec:comp-res-iter-math}

Tabela~\ref{tab:comp-res-iter-math} przedstawia wyniki testów wydajnościowych dla wyrażeń arytmetycznych o~strukturze iteracyjnej.

\begin{table}[ht]
    \centering
    \begin{tabularx}{\textwidth}{l|*{3}{Y}|*{3}{Y}|Y}
        \toprule
        \multirow{2}{*}{\textbf{Rozmiar}} & \multicolumn{3}{c|}{\textbf{ALPACA}} & \multicolumn{3}{c|}{\textbf{SLY}} & \textbf{FastParse}                                               \\
                                          & Lex                                  & Parse                             & Razem              & Lex      & Parse     & Razem     & Razem    \\
        \midrule
        100                               & 16,85~ms                             & 8,49~ms                           & 16,48~ms           & 1,52~ms  & 6,69~ms   & 8,24~ms   & 6,13~ms  \\
        500                               & 41,94~ms                             & 17,63~ms                          & 30,07~ms           & 7,98~ms  & 31,18~ms  & 38,57~ms  & 14,13~ms \\
        1000                              & 40,58~ms                             & 10,95~ms                          & 77,24~ms           & 14,84~ms & 59,65~ms  & 76,50~ms  & 21,24~ms \\
        2000                              & 166,89~ms                            & 21,22~ms                          & 188,56~ms          & 30,61~ms & 124,52~ms & 156,70~ms & 7,73~ms  \\
        \bottomrule
    \end{tabularx}
    \caption{Wyniki testów wydajnościowych dla wyrażeń arytmetycznych (dane iteracyjne)}
    \label{tab:comp-res-iter-math}
\end{table}

\subsection{Wyrażenia arytmetyczne --- dane rekurencyjne}
\label{subsec:comp-res-recur-math}

Tabela~\ref{tab:comp-res-recur-math} przedstawia wyniki dla wyrażeń arytmetycznych o~strukturze rekurencyjnej (głębokie zagnieżdżenie nawiasów).

\begin{table}[ht]
    \centering
    \begin{tabularx}{\textwidth}{l|*{3}{Y}|*{3}{Y}|Y}
        \toprule
        \multirow{2}{*}{\textbf{Rozmiar}} & \multicolumn{3}{c|}{\textbf{ALPACA}} & \multicolumn{3}{c|}{\textbf{SLY}} & \textbf{FastParse}                                                                 \\
                                          & Lex                                  & Parse                             & Razem              & Lex      & Parse     & Razem     & Razem                      \\
        \midrule
        100                               & 3,55~ms                              & 1,70~ms                           & 4,96~ms            & 1,79~ms  & 7,27~ms   & 9,25~ms   & 3,10~ms                    \\
        500                               & 27,70~ms                             & 7,81~ms                           & 29,92~ms           & 10,45~ms & 35,30~ms  & 46,85~ms  & 4,56~ms                    \\
        1000                              & 63,79~ms                             & 12,94~ms                          & 80,08~ms           & 20,22~ms & 72,95~ms  & 96,95~ms  & 7,29~ms                    \\
        2000                              & 220,96~ms                            & 25,82~ms                          & 274,95~ms          & 43,06~ms & 150,68~ms & 194,77~ms & \emph{Stack\-Over\-flow} \\
        \bottomrule
    \end{tabularx}
    \caption{Wyniki testów wydajnościowych dla wyrażeń arytmetycznych (dane rekurencyjne)}
    \label{tab:comp-res-recur-math}
\end{table}

Wyniki ujawniają istotne ograniczenie biblioteki FastParse: dla danych rekurencyjnych o~głębokości przekraczającej 1000~poziomów zagnieżdżenia występuje błąd przepełnienia stosu (ang.~\emph{StackOverflowError}).
Jest to~konsekwencją architektury kombinatorów parserów, które realizują rekurencję poprzez stos wywołań JVM\@.

\subsection{Format JSON --- dane iteracyjne}
\label{subsec:comp-res-iter-json}

Tabela~\ref{tab:comp-res-iter-json} przedstawia wyniki dla formatu JSON o~strukturze iteracyjnej (tablica obiektów).

\begin{table}[ht]
    \centering
    \begin{tabularx}{\textwidth}{l|*{3}{Y}|*{3}{Y}|Y}
        \toprule
        \multirow{2}{*}{\textbf{Rozmiar}} & \multicolumn{3}{c|}{\textbf{ALPACA}} & \multicolumn{3}{c|}{\textbf{SLY}} & \textbf{FastParse}                                                \\
                                          & Lex                                  & Parse                             & Razem              & Lex       & Parse     & Razem     & Razem    \\
        \midrule
        100                               & 30,63~ms                             & 4,09~ms                           & 26,27~ms           & 7,22~ms   & 17,07~ms  & 24,88~ms  & 15,82~ms \\
        500                               & 144,00~ms                            & 20,55~ms                          & 164,02~ms          & 35,60~ms  & 87,15~ms  & 131,62~ms & 11,67~ms \\
        1000                              & 478,19~ms                            & 46,43~ms                          & 555,45~ms          & 75,34~ms  & 182,93~ms & 263,86~ms & 12,64~ms \\
        2000                              & 1,72~s                               & 117,50~ms                         & 1,58~s             & 151,88~ms & 374,71~ms & 550,08~ms & 16,58~ms \\
        \bottomrule
    \end{tabularx}
    \caption{Wyniki testów wydajnościowych dla formatu JSON (dane iteracyjne)}
    \label{tab:comp-res-iter-json}
\end{table}

\subsection{Format JSON --- dane rekurencyjne}
\label{subsec:comp-res-recur-json}

Tabela~\ref{tab:comp-res-recur-json} przedstawia wyniki dla formatu JSON o~strukturze rekurencyjnej (głęboko zagnieżdżone obiekty).

\begin{table}[ht]
    \centering
    \begin{tabularx}{\textwidth}{l|*{3}{Y}|*{3}{Y}|Y}
        \toprule
        \multirow{2}{*}{\textbf{Rozmiar}} & \multicolumn{3}{c|}{\textbf{ALPACA}} & \multicolumn{3}{c|}{\textbf{SLY}} & \textbf{FastParse}                                                 \\
                                          & Lex                                  & Parse                             & Razem              & Lex       & Parse     & Razem     & Razem     \\
        \midrule
        100                               & 44,96~ms                             & 4,33~ms                           & 49,44~ms           & 20,33~ms  & 17,77~ms  & 39,81~ms  & 3,47~ms   \\
        500                               & 2,99~s                               & 18,09~ms                          & 3,02~s             & 358,04~ms & 95,21~ms  & 459,70~ms & 15,19~ms  \\
        1000                              & 23,55~s                              & 37,51~ms                          & 23,52~s            & 1,36~s    & 189,58~ms & 1,58~s    & 86,38~ms  \\
        2000                              & ---                                  & ---                               & ---                & 5,26~s    & 385,47~ms & 5,79~s    & 346,44~ms \\
        \bottomrule
    \end{tabularx}
    \caption{Wyniki testów wydajnościowych dla formatu JSON (dane rekurencyjne)}
    \label{tab:comp-res-recur-json}
\end{table}

Dla przypadku 2000~elementów system~\emph{ALPACA} nie zakończył pomiaru w~akceptowalnym czasie (oznaczono jako~---).

Wyniki dla rekurencyjnych danych JSON ujawniają problem wydajnościowy w~module leksykalizacji systemu~\emph{ALPACA} dla danych o~wysokim stopniu zagnieżdżenia.
Czas leksykalizacji rośnie nieproporcjonalnie do~rozmiaru danych, co~wymaga dalszej analizy i~optymalizacji (sekcja~\ref{subsec:comp-conclusions}).


\section{Analiza wyników}
\label{sec:comp-analysis}

\subsection{Porównanie faz przetwarzania}
\label{subsec:comp-phases}

Analiza wyników ujawnia zróżnicowaną charakterystykę wydajnościową poszczególnych faz przetwarzania:

\subsubsection{Faza leksykalizacji}

Lekser biblioteki SLY wykazuje stabilną i~przewidywalną wydajność, charakteryzującą się liniową złożonością względem rozmiaru danych wejściowych.
Lekser systemu~\emph{ALPACA} osiąga porównywalną wydajność dla danych iteracyjnych, jednak wykazuje znaczący spadek wydajności dla danych rekurencyjnych JSON\@.

Przyczyną tego zachowania jest sposób obsługi białych znaków w~obu systemach.
W~danych testowych JSON każdy poziom zagnieżdżenia dodaje dwie spacje wcięcia.
Dla testu \texttt{recursive\_json\_2000} najbardziej zagnieżdżony obiekt jest wcięty za~pomocą 4000~spacji.
System~\emph{ALPACA} tokenizuje każdą sekwencję białych znaków jako osobny token typu \verb|Ignored|, co~wymaga dopasowania wzorca wyrażenia regularnego dla każdej spacji.

Biblioteka SLY stosuje odmienne podejście --- pomijanie ignorowanych znaków realizowane jest prostym warunkiem iteracyjnym:

\lstinputlisting[language=python,caption={Mechanizm pomijania białych znaków w~bibliotece SLY},label={lst:comp-sly-ignore}]{listings/comparison/sly-ignore-mechanism.py}

To podejście eliminuje narzut związany z~dopasowaniem wyrażeń regularnych dla białych znaków, co~tłumaczy znaczącą przewagę wydajnościową SLY w~fazie leksykalizacji dla danych o~dużym stopniu zagnieżdżenia.
Optymalizacja mechanizmu ignorowania znaków stanowi kierunek przyszłych prac opisanych w~sekcji~\ref{subsec:comp-future}.

\subsubsection{Faza parsowania}

Faza parsowania systemu~\emph{ALPACA} wykazuje stabilną wydajność, znacząco przewyższającą bibliotekę SLY dla wszystkich testowanych scenariuszy.
Różnica ta wynika z~różnic architektonicznych:

\begin{itemize}
    \item \textbf{ALPACA} wykorzystuje tabele parsowania LR(1) generowane w~czasie kompilacji, co~eliminuje narzut interpretacji w~czasie wykonania,
    \item \textbf{SLY} interpretuje reguły gramatyczne w~czasie wykonania z~wykorzystaniem refleksji Pythona, co~wprowadza znaczący narzut.
\end{itemize}

\subsection{Zachowanie przy głębokim zagnieżdżeniu}
\label{subsec:comp-nesting}

Testy z~danymi rekurencyjnymi ujawniły istotne różnice w~obsłudze głęboko zagnieżdżonych struktur:

\begin{itemize}
    \item \textbf{FastParse} --- przepełnienie stosu dla głębokości zagnieżdżenia przekraczającej 1000--2000~poziomów, co~jest konsekwencją rekursywnej natury kombinatorów parserów,
    \item \textbf{ALPACA} --- stabilne działanie parsera niezależnie od~głębokości zagnieżdżenia dzięki wykorzystaniu automatu ze~stosem zamiast rekurencji,
    \item \textbf{SLY} --- stabilne działanie dla wszystkich testowanych głębokości, analogicznie do~ALPACA\@.
\end{itemize}

Obserwacja ta potwierdza przewagę parserów LR nad kombinatorami parserów w~scenariuszach wymagających obsługi głęboko zagnieżdżonych struktur.

\subsection{Wnioski}
\label{subsec:comp-conclusions}

Przeprowadzone eksperymenty pozwalają na sformułowanie następujących wniosków dotyczących wydajności systemu \emph{ALPACA}.

Zalety systemu:
\begin{itemize}
    \item Faza parsowania systemu \emph{ALPACA} charakteryzuje się stabilną i przewidywalną wydajnością, przewyższającą bibliotekę \emph{SLY} we wszystkich analizowanych scenariuszach testowych.
    \item Zastosowanie parsera klasy LR(1) eliminuje ograniczenia wynikające z rozmiaru stosu wywołań, umożliwiając poprawne przetwarzanie struktur o dowolnej głębokości zagnieżdżenia.
    \item Liniowa złożoność czasowa procesu parsowania względem liczby tokenów zapewnia deterministyczne i przewidywalne czasy przetwarzania danych wejściowych.
\end{itemize}

\paragraph{Obszary wymagające optymalizacji:}
Mechanizm obsługi znaków ignorowanych wykazuje obniżoną wydajność w przypadku danych wejściowych charakteryzujących się dużym stopniem wcięcia. Wynika to z faktu, że każda sekwencja znaków białych jest tokenizowana jako osobna jednostka, co prowadzi do zwiększonego narzutu obliczeniowego.


\section{Porównanie interfejsów programistycznych}
\label{sec:comp-api}

Oprócz wydajności, istotnym kryterium oceny narzędzi jest jakość interfejsu programistycznego (API).
Tabela~\ref{tab:comp-summary} zestawia kluczowe aspekty interfejsów porównywanych narzędzi.

\begin{table}[ht]
    \centering
    \begin{tabularx}{\textwidth}{L|YYY}
        \toprule
        \textbf{Kryterium}               & \textbf{ALPACA} & \textbf{SLY} & \textbf{FastParse} \\
        \midrule
        Wydajność parsowania             & wysoka          & niska        & wysoka             \\
        \hline
        Wydajność leksykalizacji         & średnia*        & wysoka       & nie dotyczy        \\
        \hline
        Obsługa głębokiego zagnieżdżenia & pełna           & pełna        & ograniczona        \\
        \hline
        Bezpieczeństwo typów             & pełne           & brak         & pełne              \\
        \hline
        Integracja IDE                   & pełna           & ograniczona  & pełna              \\
        \hline
        Diagnostyka błędów               & dobra           & średnia      & dobra              \\
        \bottomrule
    \end{tabularx}
    \caption{Podsumowanie analizy porównawczej (* wymaga optymalizacji mechanizmu ignorowania białych znaków)}
    \label{tab:comp-summary}
\end{table}

\subsection{Bezpieczeństwo typów}
\label{subsec:comp-api-types}

System~\emph{ALPACA} oferuje pełne bezpieczeństwo typów na~poziomie kompilacji dzięki wykorzystaniu typów rafinowanych (sekcja~\ref{subsec:impl-lexer-refinement}).
Każdy token i~wynik parsowania posiada precyzyjny typ znany kompilatorowi, co~umożliwia wykrywanie błędów przed uruchomieniem programu.

W~przeciwieństwie do~tego, biblioteka SLY wykorzystuje dynamiczne typowanie Pythona, co~oznacza, że~błędy typów są wykrywane dopiero w~czasie wykonania.
Fragment kodu~\ref{lst:comp-sly-type-error} ilustruje sytuację, w~której błąd typowania w~akcji semantycznej zostanie wykryty dopiero podczas parsowania konkretnego wejścia:

\lstinputlisting[language=python,caption={Błąd typowania wykrywany dopiero w~czasie wykonania (SLY)},label={lst:comp-sly-type-error}]{listings/comparison/sly-type-error.py}

\subsection[Integracja ze środowiskiem IDE]{Integracja ze~środowiskiem IDE}
\label{subsec:comp-api-ide}

System~\emph{ALPACA} oferuje pełną integrację ze~standardowymi narzędziami IDE dla języka Scala (IntelliJ IDEA, Metals) bez konieczności instalacji dedykowanych wtyczek.
Funkcjonalności takie jak:

\begin{itemize}
    \item automatyczne uzupełnianie nazw tokenów,
    \item nawigacja do~definicji (\emph{go-to-definition}),
    \item prezentacja typów przy najechaniu kursorem,
    \item wykrywanie błędów w~czasie rzeczywistym
\end{itemize}

są dostępne natywnie dzięki wykorzystaniu systemu typów Scali.

Biblioteka SLY, ze~względu na~wykorzystanie refleksji i~niestandardowych konwencji (nazwy metod jako produkcje, dekoratory~\verb|@_()|), oferuje ograniczone wsparcie IDE\@.
Analizatory statyczne, takie jak mypy, generują liczne błędy dla poprawnego kodu SLY, co~ilustruje fragment~\ref{lst:comp-mypy}:

\lstinputlisting[language=terminal,caption={Wynik analizy statycznej mypy dla parsera SLY},label={lst:comp-mypy}]{listings/comparison/mypy-math-parser-analysis.txt}

Większość zgłoszonych błędów wynika z~dynamicznego charakteru biblioteki SLY: tokeny są definiowane jako zmienne klasowe bez jawnej deklaracji typu, a~dekorator~\verb|@_()| nie jest rozpoznawany przez analizator statyczny.
Ponadto konwencja definiowania wielu metod o~tej samej nazwie (\verb|expr|) dla alternatywnych produkcji jest traktowana jako błąd redefinicji.

Rysunek~\ref{fig:comp-sly-ide} przedstawia widok kodu SLY w~środowisku VS Code, gdzie widoczne są ostrzeżenia generowane przez wbudowany analizator statyczny:

\begin{figure}[ht]
    \centering
    \includegraphics[width=0.9\textwidth]{figures/SlyMathParserInIDE}
    \caption{Ostrzeżenia IDE dla poprawnego kodu parsera SLY w~środowisku VS Code}
    \label{fig:comp-sly-ide}
\end{figure}

\clearpage

\section{Podsumowanie analizy porównawczej}
\label{sec:comp-summary-sec}

Przeprowadzona analiza porównawcza pozwala na~weryfikację tezy postawionej w~rozdziale~\ref{ch:intro}.
W~odniesieniu do~pytań badawczych sformułowanych w~sekcji~\ref{sec:intro-thesis}:

\begin{itemize}
    \item Faza parsowania systemu~\emph{ALPACA} osiąga wydajność porównywalną z~FastParse i~znacząco przewyższa bibliotekę SLY\@.
          Moduł leksykalizacji wymaga dalszej optymalizacji dla niektórych wzorców wyrażeń regularnych.

    \item Interfejs API oparty na~dopasowaniu wzorców Scali oferuje naturalne wyrażenie reguł gramatycznych przy zachowaniu pełnego bezpieczeństwa typów i~integracji z~IDE\@.

    \item System generuje komunikaty błędów zawierające kontekst syntaktyczny.
\end{itemize}

\subsection{Kierunki dalszych prac}
\label{subsec:comp-future}

Na~podstawie przeprowadzonych badań zidentyfikowano następujące kierunki dalszego rozwoju systemu~\emph{ALPACA}:

\begin{itemize}
    \item Wprowadzenie dedykowanego mechanizmu pomijania białych znaków, analogicznego do~rozwiązania stosowanego w~bibliotece SLY, który eliminowałby narzut tokenizacji dla sekwencji spacji i~tabulatorów,

    \item Implementacja algorytmu LALR(1) w~celu redukcji rozmiaru tabel parsowania dla dużych gramatyk,

    \item Przeprowadzenie testów na~większej liczbie gramatyk reprezentujących rzeczywiste języki programowania (np.~podzbiór Scali, SQL).
\end{itemize}


\chapter{Organizacja pracy}
\label{ch:org}

Rozdział przedstawia metodykę realizacji projektu dyplomowego \emph{ALPACA}, jego charakter, podział obowiązków pomiędzy~członkami zespołu, organizację prac oraz zastosowane techniki inżynierskie.
Szczególny nacisk położono na charakterystykę procesu wytwórczego, narzędzia wspierające współpracę zespołową oraz strategię walidacji implementacji.


\section[Charakterystyka projektu]{Charakterystyka projektu i~sposób realizacji}
\label{sec:org-project}

Projekt \emph{ALPACA} ma charakter badawczo-rozwojowy i~łączy elementy badań teoretycznych z~praktyczną implementacją.
Podstawowe wymaganie projektu, polegające na implementacji narzędzia umożliwiającego generowanie analizatorów leksykalnych i~składniowych w~fazie kompilacji przy pełnym wsparciu środowiska IDE, zostało jedynie częściowo sprecyzowane na etapie początkowym.
W~dalszych etapach realizacji specyfikacja była doprecyzowywana w~sposób iteracyjny, na~podstawie wyników eksperymentów oraz analizy ograniczeń technicznych.

W~pierwszym semestrze zrealizowano fazę eksploracyjno-prototypową, obejmującą analizę możliwości metaprogramowania w~Scali~3, prototypowanie mechanizmów generacji leksera i~parsera oraz identyfikację ograniczeń technicznych platformy JVM\@.
W~drugim semestrze zrealizowano fazę wdrożeniowo-optymalizacyjną, skoncentrowaną na~utrwaleniu wybranych rozwiązań, rozszerzeniu funkcjonalności systemu oraz poprawie jakości i~wydajności kodu.


\section[Zespół i podział obowiązków]{Zespół i~podział obowiązków}
\label{sec:org-team}

\subsection[Osoby w~projekcie]{Osoby w~projekcie i~ich role}
\label{subsec:org-members}

Projekt realizowany był przez zespół dwóch osób o~jasno zdefiniowanych rolach.

\textbf{Bartosz~Buczek (dalej: BB)} był odpowiedzialny przede wszystkim za implementację algorytmów analizy leksykalnej i~składniowej, opracowanie mechanizmu rozwiązywania konfliktów gramatycznych oraz przygotowanie testów wydajnościowych.

\textbf{Bartłomiej~Kozak (dalej: BK)} odpowiadał za implementację języka dziedzinowego (DSL) systemu \emph{ALPACA} z~wykorzystaniem metaprogramowania w~Scali~3, generację tabel parsowania w~fazie kompilacji, projektowanie zaawansowanych mechanizmów systemu typów oraz opracowanie dokumentacji.

\subsection{Podział prac na główne zadania}
\label{subsec:org-tasks}

Projekt podzielono na~następujące obszary funkcjonalne, z~przypisaniem głównych odpowiedzialności.

\subsubsection{System leksykalny}

Zakres:

\begin{itemize}
    \item Implementacja makra \verb|lexer| transformującego deklaratywne reguły tokenizacji w~kod proceduralny.\ (BK)
    \item Projektowanie interfejsu DSL opartego na~funkcjach częściowych.\ (BB, BK)
    \item Integracja z~wyrażeniami regularnymi biblioteki standardowej Scali.\ (BB)
    \item Definiowanie typów rafinowanych dla tokenów.\ (BK)
    \item Obsługa ignorowanych reguł leksykalnych.\ (BB)
    \item Obsługa kontekstu w~lekserze.\ (BB, BK)
    \item Diagnostyka błędów leksykalnych.\ (BB, BK)
\end{itemize}
Artefakty:

\begin{itemize}
    \item Moduł \verb|alpaca.lexer|
    \item Klasy: \verb|LexerDefinition|, \verb|Tokenization[Ctx]|, \verb|DefinedToken|
    \item Makra: \verb|lexer| oraz narzędzia pomocnicze (\verb|CompileNameAndPattern|, \verb|ReplaceRefs|)
\end{itemize}

\subsubsection{System parserów}
Zakres:

\begin{itemize}
    \item Implementacja algorytmów konstrukcji stanów LR(1) w~fazie kompilacji.\ (BB, BK)
    \item Generacja tabel parsowania (tabela akcji, tabela \verb|goto|).\ (BB)
    \item Transformacja akcji semantycznych z~kontekstu definicji do~wygenerowanych tabel.\ (BK)
    \item Obsługa ograniczeń JVM (fragmentacja metod, limit rozmiaru).\ (BK)
    \item Deklaratywne rozwiązywanie konfliktów \emph{shift--reduce} i~\emph{reduce--reduce}.\ (BB, BK)
    \item Obsługa kontekstu w~parserze.\ (BB, BK)
    \item Diagnostyka błędów składniowych.\ (BB, BK)
\end{itemize}
Artefakty:

\begin{itemize}
    \item Moduł \verb|alpaca.parser|
    \item Klasy: \verb|Parser[Ctx]|, \verb|Rule[R]|, \verb|ParseTable|, \verb|ActionTable|
    \item Makra: \verb|createTables|
\end{itemize}

\subsubsection[Infrastruktura i narzędzia pomocnicze]{Infrastruktura i~narzędzia pomocnicze}

Zakres:

\begin{itemize}
    \item Implementacja pomocniczych klas i~makr, m.in.\ \verb|Empty[T]|, \verb|ReplaceRefs|, \verb|CreateLambda|, \verb|Copyable[T]|.\ (BK)
    \item Przygotowanie systemu testów jednostkowych i~integracyjnych.\ (BB)
    \item Konfiguracja procesu budowania projektu (system \verb|mill|).\ (BK)
    \item Dokumentacja techniczna.\ (BB, BK)
\end{itemize}

\subsubsection{Dokumentacja pracy dyplomowej}

Zakres:

\begin{itemize}
    \item \nameref{ch:intro}.\ (BB, BK)
    \item \nameref{ch:meta}.\ (BK)
    \item \nameref{ch:impl}.\ (BK)
    \item \nameref{ch:lexer-algo}.\ (BB)
    \item \nameref{ch:parser-algo}.\ (BB)
    \item \nameref{ch:comp}.\ (BB)
    \item \nameref{ch:org}.\ (BK)
\end{itemize}

\subsection{Współpraca między członkami zespołu}
\label{subsec:org-collab}

Pomimo wyraźnego podziału obowiązków, współpraca między członkami zespołu miała charakter ścisły i~ciągły.
Każdy pull request do~repozytorium był poddawany przeglądowi kodu przez drugiego członka zespołu przed włączeniem zmian do~głównej gałęzi.
Zadania rejestrowano w~postaci zgłoszeń \emph{GitHub Issues} i~organizowano na~tablicy Kanban\cite{alpaca-project}, a~kamienie milowe wraz z~terminami realizacji definiowano i~monitorowano z~wykorzystaniem mechanizmu GitHub Milestones\cite{alpaca-milestones}.
Wspólnie projektowano interfejsy między modułami (między innymi interfejs \verb|Token| oraz parametryzację \verb|Ctx|) oraz rozwiązywano problemy techniczne wymagające wiedzy o~różnych komponentach systemu.
Prace prowadzono z~wykorzystaniem systemu kontroli wersji Git.


\section[Organizacja prac i~wykorzystane narzędzia]{Organizacja prac i~wykorzystane narzędzia}
\label{sec:org-tools}

\subsection{Komunikacja zespołowa}
\label{subsec:org-comm}

\subsubsection{Spotkania regularne}

\begin{itemize}
    \item Spotkania zespołu odbywały się w~interwałach 2--3-dniowych, głównie w~formie asynchronicznej z~wykorzystaniem komunikatorów.
    \item Sesje debugowania organizowano w~miarę potrzeby, w~sytuacjach wymagających jednoczesnej pracy obu członków zespołu.
    \item Burze mózgów, m.in.\ dotyczące projektowania DSL, realizowano w~formie spotkań ad hoc.
\end{itemize}

\subsubsection{Kanały komunikacji}

Komunikacja zespołu opierała się na~kilku komplementarnych kanałach.
Głównym narzędziem do~dyskusji nad kodem, proponowania zmian oraz rejestrowania błędów były GitHub Issues i~Pull Requests.
Signal służył jako kanał komunikacji tekstowej dla szybkich pytań oraz wymiany odnośników i~materiałów.
Spotkania osobiste wykorzystywano przede wszystkim do~omówień strategicznych i~podejmowania decyzji architektonicznych.

\subsubsection{Ustalanie kamieni milowych}

Podział prac na etapy został sformalizowany poprzez zdefiniowanie pięciu kamieni milowych (ang. \emph{milestones}), każdego powiązanego z~wyraźnym terminem i~katalogiem zadań.
Kamienie milowe pozwoliły zespołowi na~śledzenie postępów, priorytetyzowanie prac oraz szybkie identyfikowanie zagrożeń dla harmonogramu.

Każdy kamień milowy zawierał listę zgłoszeń (ang. \emph{issues}) reprezentujących konkretne zadania (implementacja funkcji, dokumentacja, testy).
Postęp mierzono poprzez stosunek zamkniętych zgłoszeń do~całkowitej liczby planowanych zadań, co umożliwiało szybką ocenę stanu rzeczywistego realizacji w~stosunku do~pierwotnego planu.

Strukturę kamieni milowych oraz postępy realizacji przedstawiono w~tabel~\ref{tab:org-milestones}.

\begin{table}[h]
    \centering
    \begin{tabularx}{\textwidth}{|>{\raggedright\arraybackslash}p{3.2cm}|X|c|c|}
        \hline
        \textbf{Kamień milowy}           & \textbf{Opis}                                                                                   & \textbf{Termin}         & \textbf{Postęp} \\
        \hline
        \textbf{MVP}                     & Implementacja podstawowych funkcjonalności leksera i~parsera, minimalne działające rozwiązanie. & 30 września 2025        & 15/15           \\
        \hline
        \textbf{Core Features}           & Rozszerzenie funkcjonalności.                                                                   & 31 października 2025    & 22/25           \\
        \hline
        \textbf{Stretch Goals}           & Funkcjonalności dodatkowe (optymalizacje, rozszerzona diagnostyka błędów, dodatkowe przykłady)  & brak ustalonego terminu & 21/30           \\
        \hline
        \textbf{Testing \& Benchmarking} & Kompleksowe testowanie (testy integracyjne, benchmarki wydajności, walidacja gramatyk)          & 30 listopada 2025       & 2/5             \\
        \hline
        \textbf{Thesis}                  & Finalizacja rozprawy dyplomowej (pisanie rozdziałów, bibliografia, ostateczna redakcja)         & 15 grudnia 2025         & 12/13           \\
        \hline
    \end{tabularx}
    \caption{Kamienie milowe projektu \emph{ALPACA} i~postęp ich realizacji}
    \label{tab:org-milestones}
\end{table}

Postęp realizacji wskazuje, że zespół pomyślnie ukończył pierwsze dwa kamienie milowe w~wyznaczonym terminie, trzecia faza (Stretch Goals) była realizowana równolegle i~bieżąco, czwarta faza wymagała intensywniejszych wysiłków na~ostatnim etapie projektu, natomiast piąta faza (Thesis) była finalizowana w~ostatnich tygodniach przed terminem oddania pracy dyplomowej.

\subsection[Narzędzia programistyczne i CI/CD]{Narzędzia programistyczne i~CI/CD}
\label{subsec:org-dev-tools}

\subsubsection{System kontroli wersji Git i~GitHub}

Całość projektu hostowana jest na~platformie GitHub\cite{alpaca-github}.

\subsubsection{Narzędzie do budowania --- \texttt{mill} 1.x}

Konfiguracja w~pliku \verb|build.mill| definiuje zależności, wersję kompilatora Scali (3.8.0) oraz dodatkowe pluginy.
Typowe polecenia obejmują \verb|mill compile|, \verb|mill test| oraz \verb|mill run|.
Średni czas kompilacji projektu wynosi około 45~sekund (w~tym około 35~sekund na~uruchomienie testów).

\subsubsection{Weryfikacja jakości}

Weryfikacja jakości kodu odbywała się z~wykorzystaniem narzędzia Scalafmt, zapewniającego automatyczne formatowanie kodu do~spójnego stylu.
Konfigurację utrzymywano w~pliku \verb|.scalafmt.conf|.

Dodatkowo projekt kompiluje się bez ostrzeżeń przy włączonych zaostrzonych opcjach kompilatora, takich jak \verb|-Xfatal-warnings| (traktowanie ostrzeżeń jako błędów) czy \verb|-Ycheck:macros| (dodatkowa weryfikacja poprawności makr).

\subsubsection{Testowanie}

Wykorzystano framework testowy \emph{ScalaTest}, który umożliwił przygotowanie testów jednostkowych dla modułów leksera i~parsera, a~także testów integracyjnych dla pełnych przykładów (parser wyrażeń arytmetycznych, parser uproszczonego formatu JSON). Testy uruchamiano poleceniem \verb|mill test|.
Pokrycie testami obejmuje główne ścieżki wykonania i~przypadki brzegowe.

Testowanie systemów opartych na~makrach kompilacyjnych stanowi szczególne wyzwanie, ponieważ makra wykonywane są w~fazie kompilacji, a~błędy mogą ujawniać się dopiero na~poziomie wygenerowanego kodu.
W~projekcie zastosowano zarówno testy pozytywne, weryfikujące poprawną kompilację oraz semantykę wygenerowanego kodu, jak i~testy negatywne, sprawdzające poprawne odrzucanie niepoprawnych definicji z~odpowiednimi komunikatami diagnostycznymi.
Dodatkowo przygotowano testy wydajnościowe oraz scenariusze integracyjne typu end-to-end (obejmujące jednocześnie lekser i~parser), aby ocenić zachowanie systemu w~warunkach zbliżonych do~rzeczywistego użycia.

\subsubsection[Continuous Integration z wykorzystaniem GitHub Actions]{Continuous Integration z~wykorzystaniem GitHub Actions}

Każde wypchnięcie zmian do~repozytorium uruchamia przepływ pracy CI, obejmujący kompilację projektu, uruchomienie testów, weryfikację formatowania kodu (Scalafmt) oraz prezentację statusu tych kroków na~pull requestach przed ich połączeniem z~głównym branchem.

\subsection{Dokumentacja}\label{subsec:org-docs}

\subsubsection[Dokumentacja w kodzie]{Dokumentacja w~kodzie}

Każda publiczna klasa, funkcja oraz makro opatrzone są komentarzami Scaladoc.
Komentarze dokumentują cel, parametry, wartości zwracane oraz przykładowe scenariusze użycia.
Złożone implementacje (np.\ algorytm LR(1)) uzupełniono o~komentarze liniowe objaśniające kluczowe fragmenty logiki.

Plik \texttt{README.md} zawiera instrukcje instalacji, przykłady użycia oraz zarys architektury systemu.

\texttt{GitHub Pages} hostuje bardziej szczegółową dokumentację, poradniki oraz sekcję FAQ\cite{alpaca-docs}.

\subsubsection{Praca dyplomowa (LaTeX)}

Praca została przygotowana w~systemie składu \LaTeX{} z~wykorzystaniem szablonu AGH (\verb|aghengthesis|).
Treść została podzielona na~rozdziały w~oddzielnych plikach (m.in.\ \verb|introduction.tex|, \verb|metaprogramming.tex|, \verb|implementation.tex|).
Bibliografię przygotowano w~formacie Bib\LaTeX{} (\verb|bibliografia.bib|) z~odwołaniami do~literatury naukowej oraz dokumentacji technicznej.


\section[Zastosowane techniki i~praktyki]{Zastosowane techniki i~praktyki inżynierskie}
\label{sec:org-practices}

\subsection{Metodologia wytwarzania oprogramowania}
\label{subsec:org-methodology}

\subsubsection[Iteracyjne podejście do projektowania]{Iteracyjne podejście do~projektowania}

Projekt nie posiadał sztywnej, kompletnej specyfikacji funkcjonalnej na~etapie początkowym.
Projektowanie przebiegało iteracyjnie, zgodnie z~następującym schematem:

\begin{enumerate}
    \item Prototypowanie --- szybkie eksperymentowanie z~różnymi podejściami.
    \item Ewaluacja --- ocena wydajności, przydatności oraz spójności otrzymanych rozwiązań.
    \item Udoskonalanie --- ulepszanie wybranych podejść.
    \item Integracja --- łączenie poszczególnych komponentów w~spójny system.
\end{enumerate}

\subsubsection{Test-Driven Development (TDD) --- częściowe zastosowanie}

Dla modułów o~jasno zdefiniowanych specyfikacjach (np.\ funkcji \verb|computeFirstSet|) testy przygotowywano przed implementacją, zgodnie z~podejściem Test-Driven Development.
Dla komponentów o~charakterze eksploracyjnym (np.\ mechanizmów generacji klas w~makrach) testy opracowywano po~ustabilizowaniu implementacji.
Praktyka ta była szczególnie przydatna do~ujawniania problemów w~interfejsach pomiędzy~modułami.

\subsubsection{Refaktoryzacja}

Regularnie przeprowadzano refaktoryzacje w~celu poprawy czytelności, jakości oraz efektywności kodu.
Uzasadnione zmiany refaktoryzacyjne dokumentowano w dedykowanej sekcji Pull Requesta.
Wykorzystywano narzędzia wspomagające refaktoryzację (m.in.\ Scalafix).

\subsubsection[Code Review i Pair Programming]{Code Review i~Pair Programming}

Każdy Pull Request był recenzowany przez drugiego członka zespołu przed połączeniem z~głównym branchem.
W~przypadku zagadnień o~szczególnie złożonym charakterze zastosowano sesje wspólnego programowania (ang. \emph{Pair Programming}).
Spotkania poświęcone przeglądom kodu odbywały się 2--3 razy w~tygodniu, zwykle przez 30--60~minut.

\subsubsection[Zgłoszenia i planowanie]{Zgłoszenia i~planowanie}
Podczas cyklicznych spotkań omawiano priorytety oraz wybierano kolejne zgłoszenia (ang. \emph{issues}) do~realizacji.
Backlog utrzymywano w~postaci zgłoszeń \emph{GitHub Issues}.

\subsection{Praktyki specyficzne dla metaprogramowania}
\label{subsec:org-meta-practices}

\subsubsection{Obliczenia wieloetapowe (ang. \emph{Staged Computation})}

Świadomie rozróżniano obliczenia wykonywane na~etapie kompilacji i~na~etapie wykonania.
Maksymalizowano zakres obliczeń przeniesionych do~etapu kompilacji (m.in.\ generacja tabel LR(1), kompilacja wyrażeń regularnych).
Minimalizowano nakład obliczeń w~fazie wykonania, co pozwoliło uzyskać wysoką wydajność uruchomieniową.

\subsubsection[Walidacja w fazie kompilacji (Validation at Compile-Time)]{Walidacja w~fazie kompilacji (Validation at Compile-Time)}

Walidowano gramatyki już na~etapie kompilacji (m.in.\ wykrywanie konfliktów LR, niepoprawnej składni).
Walidowano wyrażenia regularne w~definicjach leksera.
Stosowano zasadę szybkiego zwrotu błędu (ang. \emph{fail-fast}) --- preferowano odrzucanie niepoprawnych danych na~etapie kompilacji zamiast zgłaszania błędów dopiero w~czasie wykonania.

\subsection{Praktyki DevOps}
\label{subsec:org-devops}

\subsubsection{Continuous Integration}

GitHub Actions automatyzowało kompilację, uruchamianie testów oraz weryfikację formatowania kodu przy każdym wypchnięciu zmian.
Zastosowano reguły ochrony gałęzi (ang. \emph{branch protection rules}), wymagające pozytywnego wyniku wszystkich testów przed połączeniem zmian z~głównym branchem.

\subsubsection{Zarządzanie artefaktami (Artifact Management)}

Wersje rozwojowe (SNAPSHOT) i stabilne wydania są publikowane w~repozytorium Maven Central.


\section[Przebieg prac --- harmonogram i~iteracje]{Przebieg prac --- harmonogram i~iteracje}
\label{sec:org-schedule}

\subsection[Szczegółowa oś czasu i przebieg prac]{Szczegółowa oś czasu i~przebieg prac}
\label{subsec:org-timeline}

Praca ALPACA realizowana była przez 27~tygodni, obejmujących dwa semestry akademickie.
Szczegółowy przebieg oraz liczba zmian w~poszczególnych etapach dokumentowane były poprzez system kontroli wersji, umożliwiający śledzenie postępów na~poziomie pojedynczych zmian.

W~toku realizacji projektu zebrano 216~commitów (zmian w~kodzie), z~których średnio przypadało 19,6~zmian na~tydzień.
Tabela~\ref{tab:org-timeline} prezentuje podział prac na~etapy wraz z~kluczowymi osiągnięciami i~liczbą zmian wprowadzonych w~każdym okresie.

\begin{table}[h]
    \centering
    \begin{tabularx}{\textwidth}{|l|l|X|}
        \hline
        \textbf{Etap}  & \textbf{Okres czasowy}        & \textbf{Kluczowa aktywność}                                       \\
        \hline
        Inicjalizacja  & 19 kwietnia 2025              & Konfiguracja repozytorium, system budowania Mill                  \\
        \hline
        \multicolumn{3}{c}{\textbf{Semestr 1 --- Eksploracja i~prototypowanie (tygodnie 1--14)}}                           \\
        \hline
        Tydzień 1--3   & 22--27 lipca                  & Podstawy teoretyczne, pierwsze makra, API metaprogramowania       \\
        \hline
        Tydzień 4--6   & 27 lipca--6 sierpnia          & Eksperymentowanie: DFA vs.~wyrażenia regularne, framework testowy \\
        \hline
        Tydzień 7--10  & 6--25 sierpnia                & Szkic algorytmu LR(1), infrastruktura testowa, integracja         \\
        \hline
        Tydzień 11--14 & 25 września--6 października   & Identyfikacja ograniczeń JVM, planowanie semestr 2                \\
        \hline
        \multicolumn{3}{c}{\textbf{Semestr 2 --- Implementacja i~optymalizacja (tygodnie 15--27)}}                         \\
        \hline
        Tydzień 15--17 & 13 października--30 listopada & Moduł leksera, typy rafinowane, interfejs DSL                     \\
        \hline
        Tydzień 18--20 & 30 października--20 listopada & Generator parserów LR(1), obsługa ograniczeń JVM                  \\
        \hline
        Tydzień 21--23 & 20--27 listopada              & Integracja leksera i~parsera, akcje semantyczne                   \\
        \hline
        Tydzień 24--25 & 27 listopada--5 grudnia       & Optymalizacja wydajności, poprawa diagnostyki błędów              \\
        \hline
        Tydzień 26--27 & 6--15 grudnia                 & Finalizacja rozprawy dyplomowej, ostateczne poprawki              \\
        \hline
    \end{tabularx}
    \caption{Szczegółowa oś czasu projektu ALPACA z~liczbą zmian (commitów) w~poszczególnych etapach}
    \label{tab:org-timeline}
\end{table}

\subsubsection{Charakterystyka poszczególnych etapów}

\paragraph{Faza eksploracyjna (tygodnie 1--14)}

Pierwsza połowa prac skupiła się na~zdobyciu doświadczenia z~metaprogramowaniem w~Scali~3 oraz eksperymentowaniem z~różnymi podejściami do~generacji leksera. W~tym okresie zespół zebrał 55~commitów, co wskazuje na~intensywne prototypowanie i~ciągłe eksperymentowanie. Szczególnie intensywny okazał się okres tygodni 11--14 (20~commitów), kiedy identyfikowano ograniczenia platformy JVM i~opracowywano strategie ich obejścia.

\paragraph{Faza wdrożeniowa (tygodnie 15--25)}

Druga część projektu cechowała się szybszym tempem zmian (130~commitów), co odzwierciedlało przejście z~fazy eksploracji na~implementację. Szczególnie dynamiczny był okres tygodni 15--17 (45~commitów), kiedy równocześnie realizowano moduł leksera oraz wprowadzano obsługę typów rafinowanych. Okres tygodni 21--23 (35~commitów) skupił się na~integracji komponentów i~obsłudze akcji semantycznych w~parserze.

\paragraph{Finalizacja (tygodnie 26--27)}

Ostatnie dwa tygodnie projektu zawierały 15~commitów poświęconych przede wszystkim dokumentacji, ostatecznym poprawkom rozprawy dyplomowej oraz czyszczeniu kodu.

\subsection[Szybkość pracy zespołu i postępy]{Szybkość pracy i~postępy}
\label{subsec:org-progress}

\subsubsection{Metryki wydajności}

Realizacja projektu ALPACA wymagała znaczącego zaangażowania zespołu przez okres kilku miesięcy.
Skala pracy oraz złożoność napotkanych wyzwań technicznych znajdują odzwierciedlenie w poniższych wskaźnikach:

\begin{itemize}
    \item Historia rozwoju reprezentuje łącznie 450 commitów zintegrowanych w głównej gałęzi projektu, dokumentujących iteracyjny proces implementacji oraz poprawiania systemu.
    \item Struktura zmian kodowych obejmuje około 80 Pull Requestów, z których każdy zawierał recenzję i testy weryfikujące poprawność dodawanych funkcjonalności.
    \item Rozmiar implementacji wynosi około 4000 linii kodu implementacyjnego (bez uwzględnienia testów) oraz około 2000 linii kodów testowych.
\end{itemize}

Rozkład pracy w~czasie pokazany na rysunku~\ref{fig:org-commits} ilustruje dynamikę zespołu, wskazując na stały postęp oraz okresy intensywnej pracy nad rozwiązywaniem kluczowych wyzwań technicznych.

\begin{figure}[!htbp]
    \centering
    \includegraphics[width=0.9\textwidth]{figures/Commits}
    \caption{Rozkład commitów w~tygodniach realizacji projektu ALPACA}
    \label{fig:org-commits}
\end{figure}


\section[Główne wyzwania i~ich rozwiązania]{Główne problemy i~ich rozwiązania}
\label{sec:org-problems}

\subsection{Limit rozmiaru metody JVM}
\label{subsec:prob-jvm-limit}

\subsubsection{Problem}

Przy naiwnej implementacji generowanie dużych tabel parsowania LR(1) prowadziło do~przekroczenia limitu 64~KB na~rozmiar kodu bajtowego pojedynczej metody w~JVM\@.

\subsubsection{Przyczyna}

Każdy wpis tabeli parsowania (para: stan, symbol terminalny $\rightarrow$ akcja \emph{shift}/\emph{reduce}) wymagał wygenerowania kilku instrukcji.
Dla złożonych gramatyk z~tysiącami wpisów rozmiar wygenerowanej metody przekraczał dopuszczalny limit.

\subsubsection{Rozwiązanie}

Zastosowano podejście polegające na~fragmentacji metod (\emph{method fragmentation})~\cite{method-too-large}.
Zamiast generować pojedynczą metodę zawierającą całą tabelę parsowania, generowanych jest wiele metod pomocniczych, z~których każda odpowiada za dodanie ograniczonej liczby wpisów do~struktury budującej tabelę.
Metoda główna sekwencyjnie wywołuje metody pomocnicze, a~następnie zwraca wynikową tabelę.
Ze względu na niewielki rozmiar metod pomocniczych kompilator JIT może je skutecznie inlinować, minimalizując narzut wykonania.

\subsubsection{Rezultat}

Każda metoda pomocnicza zawiera jedynie kilka instrukcji, dzięki czemu pozostaje poniżej limitu rozmiaru metody w~JVM. System jest obecnie w~stanie obsługiwać gramatyki z~setkami stanów LR(1) bez naruszenia ograniczeń platformy.

\subsection{Transmisja referencji między etapami kompilacji}
\label{subsec:prob-refs}

\subsubsection{Problem}

Akcje semantyczne w~definicji parsera mogą odwoływać się do~kontekstu parsera oraz zmiennych z~otaczającego zakresu leksykalnego.
Podczas generacji kodu w~makrze referencje te muszą zostać przepisane w~taki sposób, aby odnosiły się do~odpowiednich symboli w~wygenerowanym kodzie.

\subsubsection{Przyczyna}

Makra operują na~reprezentacji drzewa składni abstrakcyjnej (AST). Zmienne z~oryginalnego zakresu leksykalnego nie istnieją w~kontekście wygenerowanej klasy.
Bezpośrednie kopiowanie referencji prowadziło do~błędów typu „symbol not found”.

\subsubsection{Rozwiązanie}

Zaimplementowano klasę \verb|ReplaceRefs| realizującą transformację AST poprzez zmianę przypisania symboli (ang.~\emph{re-owning}). Transformacja przebiega według następujących kroków:

\begin{enumerate}
    \item Analiza AST kodu akcji semantycznej.
    \item Identyfikacja referencji do~starego kontekstu (np.~parametru makra \verb|ctx|).
    \item Zastąpienie ich referencjami do~nowego kontekstu (parametr metody w~wygenerowanej klasie).
    \item Zachowanie pozostałych referencji bez zmian.
\end{enumerate}

Klasa \verb|ReplaceRefs| rozszerza \verb|TreeMap| z~API refleksji TASTy, co umożliwia rekurencyjne przejście po~całym AST i~odpowiednią transformację symboli.

\subsubsection{Rezultat}

Akcje semantyczne prawidłowo odwołują się do~kontekstu parsera oraz zmiennych otaczających, bez generowania błędów typowania.

\subsection[Typy rafinowane a wsparcie IDE]{Typy rafinowane a~wsparcie IDE}
\label{subsec:prob-refinement-ide}

\subsubsection{Problem}

W wersji pozbawionej typów rafinowanych interfejs API nie dostarczał środowisku IDE informacji o~dostępnych tokenach.
Funkcje autouzupełniania, podpowiedzi typów oraz mechanizmy \emph{go-to-definition} działały w~sposób ograniczony.

\subsubsection{Przyczyna}

System typów musiał dysponować informacją o~polach dostępnych w~wartości zwracanej przez lekser, przy czym informacja ta była pierwotnie dostępna jedynie na~poziomie makra w~fazie kompilacji.

\subsubsection{Rozwiązanie}

Zaimplementowano system typów rafinowanych, w~którym:

\begin{itemize}
    \item każdy token reprezentowany jest jako pole w~typie rafinowanym,
    \item typ rafinowany udostępnia \verb|type Fields| definiujący \verb|NamedTuple| ze wszystkimi tokenami,
    \item środowisko IDE ma dostęp do~pełnych informacji typów i~może oferować wsparcie w~postaci autouzupełniania oraz podpowiedzi,
    \item dostęp do~pola tokena (np.~\verb|c.NUMBER|) jest statycznie typowany i~bezpieczny.
\end{itemize}

\begin{figure}[!htbp]
    \centering
    \includegraphics[width=0.7\textwidth]{figures/IDESupport}
    \caption{Wsparcie IDE dzięki typom rafinowanym w~projekcie ALPACA}
    \label{fig:org-ide-support}
\end{figure}

\subsubsection{Rezultat}

Osiągnięto pełną integrację z~IDE (Metals), obejmującą autouzupełnianie, podpowiedzi typów oraz wczesne zgłaszanie błędów typów.

\subsection{Wydajność kompilacji makr}
\label{subsec:prob-compile-perf}

\subsubsection{Problem}

Dla złożonych gramatyk generacja tabel LR(1) w~makrze była czasochłonna, co wydłużało czas kompilacji projektu użytkownika.

\subsubsection{Przyczyna}

Algorytmy \emph{Closure} i~\emph{Goto} dla LR(1) wymagają wielu iteracji oraz porównań zbiorów stanów.
Obliczenia wykonywane na~reprezentacji AST w~makrze generują dodatkowy narzut.

\subsubsection{Rozwiązanie}

\begin{itemize}
    \item Zoptymalizowano algorytmy LR(1), zastępując naiwne pętle bardziej efektywnymi strukturami danych (np.~buforami modyfikowalnymi zamiast list).
    \item Wprowadzono buforowanie wyników pośrednich (m.in.\ zbiorów \emph{FIRST}).
    \item Zastosowano leniwą ewaluację --- niektóre stany LR(1) generowane są wyłącznie wtedy, gdy są rzeczywiście osiągalne.
\end{itemize}

\subsubsection{Rezultat}

Czas kompilacji złożonych gramatyk utrzymuje się poniżej 10~sekund (w~zależności od~rozmiaru gramatyki), co jest akceptowalne z~perspektywy użytkownika biblioteki.

\subsection[Testowanie makr i metaprogramowania]{Testowanie makr i~metaprogramowania}
\label{subsec:prob-meta-testing}

\subsubsection{Problem}

Testowanie systemów opartych na~makrach kompilacyjnych jest utrudnione, ponieważ makra wykonują się na~etapie kompilacji, a~błędy często ujawniają się dopiero w~wygenerowanym kodzie.

\subsubsection{Przyczyna}

Tradycyjne frameworki testowe (np.~munit, ScalaTest) operują na~kodzie uruchamianym w~fazie wykonania, podczas gdy makra wymagają weryfikacji zarówno poprawności generowanego kodu, jak i~samych efektów kompilacji.

\subsubsection{Rozwiązanie}

\begin{itemize}
    \item Zdefiniowano testy pozytywne, sprawdzające, czy poprawne dane wejściowe generują kod, który się kompiluje i~działa zgodnie z~oczekiwaniami.
    \item Przygotowano testy negatywne, sprawdzające, czy błędne definicje są odrzucane z~czytelnymi komunikatami błędów.
    \item Zaimplementowano testy wydajnościowe (benchmarki) dla wybranych przypadków (parser wyrażeń arytmetycznych, parser formatu JSON).
    \item Opracowano testy integracyjne typu end-to-end, obejmujące pełen przepływ (lekser + parser).
\end{itemize}

Każdy test makra weryfikuje zarówno poprawność kompilacji, jak i~semantykę wygenerowanego kodu.

\subsection[Błędy w kolejności definicji tokenów]{Błędy w~kolejności definicji tokenów}
\label{subsec:prob-regex-order}

\subsubsection{Problem}

Topologia definiowania tokenów może prowadzić do~błędów leksykalnych, jeśli krótsze wzorce zostały umieszczone przed dłuższymi wariantami zawierającymi je~jako prefiks.

\textbf{Przykład:} użytkownik może zdefiniować operatory porównania w~naturalnej kolejności:
\begin{verbatim}
LT = <
LE = <=
GT = >
GE = >=
\end{verbatim}

Notacja jest składniowo poprawna, lecz prowadzi do~błędów w~fazie parsowania wyrażenia \verb|x <= 5|, gdyż system rozpozna jedynie \verb|<|, podczas gdy \verb|=| zostanie odrzucony jako symbol nieznany, co~spowoduje błąd składniowy w~kolejnych fazach analizy.

\subsubsection{Przyczyna}

Biblioteczne wyrażenia regularne determinują, że~tokenizacja przebiega sekwencyjnie według porządku definicji, bez~zastosowania heurystyki maksymalnego dopasowania (zob.~rozdział~\ref{sec:lexer-algo-regex}).

Trudność w~diagnozowaniu tego typu błędów wynika z~następujących czynników:

\begin{itemize}
    \item Błąd nie~jest~sygnalizowany w~fazie kompilacji leksera.
    \item Błędy ujawniają się~dopiero w~fazie parsowania konkretnego wejścia.
    \item Komunikat diagnostyczny wskazuje na parser (\enquote{nieoczekiwany symbol}), a~nie~lekser, utrudniając identyfikację pierwotnej przyczyny.
\end{itemize}

\subsubsection{Rozwiązanie}

Konflikty takie są~automatycznie wykrywane w~fazie kompilacji przez dedykowany moduł \verb|RegexChecker| (por. rozdział~\ref{subsec:lexer-algo-val}).
Przed~generacją kodu leksera moduł \verb|RegexChecker| analizuje wzorce w~poszukiwaniu potencjalnych konfliktów.
W~przypadku wykrycia nieprawidłowości użytkownik otrzymuje komunikat błędu wskazujący źródło problemu.

\begin{figure}[!htbp]
    \centering
    \includegraphics[width=0.7\textwidth]{figures/PatternShadowing}
    \caption{Przykład komunikatu o wykryciu konfliktu w~kolejności definicji tokenów}
    \label{fig:org-regex-error}
\end{figure}

\subsubsection{Rezultat}

Konflikty w~kolejności wzorców są~wykrywane w~fazie kompilacji poprzez komunikaty, które jasno wskazują, które wzorce~się~nakładają i~wymagają zdefiniowania precedencji.

Korekta wymaga zmiany kolejności definicji tokenów (np.~umieszczenie \verb|<=| przed~\verb|<|) lub~modyfikacji wyrażeń regularnych, aby~wyeliminować nakładania się~wzorców.

Wczesna detekcja konfliktów wzorców okazała się~istotna przy~implementacji przykładowych parserów (parser wyrażeń arytmetycznych, parser~JSON).

\subsection{Konflikty precedencji operatorów}
\label{subsec:prob-lr-conflicts}

\subsubsection{Problem}

Definiowanie gramatyk dla~wyrażeń arytmetycznych wymaga określenia reguł precedencji operatorów, aby~wymusić poprawny porządek wykonywania operacji (np.~mnożenie przed~dodawaniem).

Brak jawnych reguł precedencji powoduje wieloznaczność strukturalną wyrażenia \verb|2 + 3 * 4|, które może być interpretowane jako $(2 + 3) \times 4 = 20$ lub $2 + (3 \times 4) = 14$.
System sygnalizuje błąd kompilacji wskazujący na~konflikt typu shift--reduce, który wymaga jawnego rozstrzygnięcia przez~użytkownika.

Błędy mogą wynikać również z~niespójności w~definicjach reguł precedencji.
Na~przykład, próba definiowania operatora unarnego minus przy~użyciu zwykłego operatora odejmowania może prowadzić do~cyklicznych zależności precedencji, gdzie $-$ ma~pierwszeństwo przed~$*$, a~$*$ przed~$-$, co jest niemożliwe do rozstrzygnięcia.

\subsubsection{Przyczyna}

Wiele naturalnych gramatyk języków programowania zawiera niejednoznaczności, takie jak wyrażenia arytmetyczne z~różnymi poziomami precedencji operatorów, czy konstrukcje warunkowe z~\enquote{dangling else}:

Bez~takiego mechanizmu użytkownik byłby zmuszony do przepisania gramatyki do~formy jednoznacznej (np.~przez~wprowadzenie odrębnych nieterminali dla~każdego poziomu precedencji), co~znacznie komplikuje specyfikację i~zmniejsza jej przejrzystość.

\subsubsection{Rozwiązanie}

System ALPACA oferuje mechanizm deklaratywnego rozwiązywania konfliktów bez~konieczności przepisywania gramatyki (algorytmy opisane w~rozdziale~\ref{subsec:parser-algo-conflict-res}).

Dla~gramatyki wyrażeń arytmetycznych użytkownik definiuje jedynie łańcuch relacji:

\begin{verbatim}
pow before mul
mul before add
\end{verbatim}

Notacja \texttt{X before Y} oznacza, że~operator \texttt{X} ma~wyższy priorytet niż~operator~\texttt{Y}.

System automatycznie:

\begin{itemize}
    \item wyprowadza przechodniość: Skoro \verb|pow| przed \verb|mul| i~\verb|mul| przed \verb|add|, to~\verb|pow| przed \verb|add|,
    \item rozwiązuje wszystkie konflikty: Używając grafu relacji pierwszeństwa,
    \item weryfikuje spójność: Sprawdza czy~nie~ma~cykli w~relacjach.
\end{itemize}

Jeśli~podczas przetwarzania zdefiniowanych reguł system wykryje cykl (np.~$A \prec B \prec C \prec A$), wyświetlana jest diagnostyka:

\begin{lstlisting}[caption={Przykładowy komunikat o cycklu.},label={lst:parser-algo-verify-error}]
Inconsistent conflict resolution detected:
Reduction(sub) before Shift(*) before Reduction(sub)
There are elements being both before and after Reduction(sub)
at the same time.
Consider revising the before/after rules to eliminate cycles
\end{lstlisting}

Komunikat wskazuje dokładne położenie sprzeczności, umożliwiając szybką korektę.

\subsubsection{Rezultat}
Zastosowanie mechanizmu rozwiązywania konfliktów przynosi kilka istotnych korzyści.
Po pierwsze, pozwala na~definiowanie gramatyki w~formie deklaratywnej, bez~konieczności ręcznego przekształcania struktury.
Po drugie, znacznie redukuje liczbę wymaganych deklaracji --- w~praktyce z~$O(n^2)$ do~$O(n)$ dla~typowych gramatyk.
Ponadto, system wykrywa niespójności już w~fazie kompilacji, a~komunikaty diagnostyczne jasno wskazują źródło konfliktu,
ułatwiając szybką~korektę.

Zastosowanie tego mechanizmu pozwoliło na~zrealizowanie kompletnego parsera wyrażeń arytmetycznych obejmującego operatory \verb|+|, \verb|-|, \verb|*|, \verb|/|, \verb|^| oraz~zagnieżdżone nawiasy w~mniej niż~40~liniach kodu.


\section[Wdrożenia, testy i eksperymenty]{Wdrożenia, testy i~eksperymenty}
\label{sec:org-tests}

\subsection{Testowanie jednostkowe}
\label{subsec:org-unit-tests}

Projekt zawiera rozbudowany zestaw testów jednostkowych, obejmujących poszczególne moduły systemu.

\begin{table}[ht]
    \centering
    \begin{tabularx}{\textwidth}{|l|X|}
        \hline
        \textbf{Moduł / Komponent} & \textbf{Zakres testów jednostkowych} \\
        \hline
        Lekser &
        \begin{itemize}[leftmargin=*, nosep, after=\vspace{-\baselineskip}, before=\vspace{-0.6\baselineskip}]
            \item Testowanie definicji tokenów (nazwy, wzorce, wartości domyślne).
            \item Testowanie ignorowanych reguł leksykalnych.
            \item Testowanie wykrywania konfliktów nazw tokenów (błędy kompilacji).
            \item Testowanie obsługi niepoprawnych wyrażeń regularnych.
        \end{itemize} \\
        \hline
        Parser &
        \begin{itemize}[leftmargin=*, nosep, after=\vspace{-\baselineskip}, before=\vspace{-0.6\baselineskip}]
            \item Testowanie ekstrakcji produkcji z~definicji.
            \item Testowanie algorytmów LR(1) (\emph{closure}, \emph{goto}, obliczanie zbiorów \emph{FIRST}).
            \item Testowanie rozwiązywania konfliktów typu \emph{shift--reduce}.
            \item Testowanie akcji semantycznych z~różnymi typami kontekstu.
        \end{itemize} \\
        \hline
        Infrastruktura &
        \begin{itemize}[leftmargin=*, nosep, after=\vspace{-\baselineskip}, before=\vspace{-0.6\baselineskip}]
            \item Testowanie klas typów (\verb|Empty[T]|, \verb|Copyable[T]|).
            \item Testowanie transformacji AST (\verb|ReplaceRefs|, \verb|CreateLambda|).
        \end{itemize} \\
        \hline
    \end{tabularx}
    \caption{Zakres testów jednostkowych dla głównych komponentów systemu}
    \label{tab:org-unit-tests}
\end{table}

\subsection{Testowanie integracyjne}
\label{subsec:org-int-tests}

Projekt zawiera kilka pełnych przykładów integracyjnych, demonstrujących działanie systemu w~praktyce.

\subsubsection{Kalkulator wyrażeń arytmetycznych}

Zaimplementowano kompleksowy parser wyrażeń arytmetycznych demonstrujący pełne możliwości systemu ALPACA.
Parser obsługuje szeroką~gamę operatorów (dodawanie, odejmowanie, mnożenie, dzielenie, modulo, potęgowanie,
dzielenie całkowite) oraz~funkcje matematyczne (trygonometryczne, hiperboliczne i~odwrotne),
a~także stałe matematyczne (pi, e, tau, nieskończoność, NaN). Akcje semantyczne obliczają wartość wyrażenia
poprzez wyodrębnianie wartości liczbowych z~tokenów i~wykonywanie odpowiednich operacji arytmetycznych.

Lekser systemu obsługuje ignorowanie białych znaków i~komentarzy, definiuje operatory wieloznakowe (\\verb|**|, \\verb|//|)
przed~operatorami jednoznakowymi, by~uniknąć konfliktów w~kolejności dopasowania. Ponadto rozpoznaje liczby zmiennoprzecinkowe
i~całkowite (w~tym notację wykładniczą) oraz~funkcje matematyczne jako~słowa kluczowe.

Parser definiuje kompleksowy zbiór reguł precedencji, gdzie potęgowanie ma~najwyższy priorytet,
następnie operatory unarny plus i~minus, potem mnożenie i~operatory pochodne (modulo, dzielenie całkowite),
a~wreszcie dodawanie i~odejmowanie jako~operatory o~najniższym priorytecie. Ta~hierarchia jest jawnie zdeklarowana
w~zbiorze rozwiązań konfliktów (ang. \emph{resolutions}), zapewniając poprawną interpretację wyrażeń.

System weryfikowany był na~trzech poziomach złożoności:

\begin{itemize}
    \item Test podstawowy --- proste wyrażenie \verb|1 + 2| zwracające oczekiwany wynik $3.0$;

    \item Test złożoności średniej --- wielopoziomowe nawiasowanie i~operacje łączące operatory o~różnych priorytetach,
          takie jak~wyrażenie \verb|((12 + 7) * (3 - 8 / (4 + 2)) + (15 - (9 - 3 * (2 + 1))) / 5)...|,
          potwierdzające poprawną ewaluację zagnieżdżonych wyrażeń;

    \item Test zaawansowany --- wyrażenie zawierające funkcje trygonometryczne,
          operatory potęgowania i~dzielenia całkowitego, takie jak~\verb|sin(pi/6) + cos(pi/3) + (2 ** 3 ** 2) / (3 + 1) + ...|,
          demonstrujące prawidłową obsługę złożonych operacji i~funkcji matematycznych.
\end{itemize}

Wszystkie testy przechodzą pomyślnie, co~świadczy o~poprawności implementacji leksera, parsera,
zarządzania priorytetami operatorów oraz~akcji semantycznych w~kontekście wyrażeń o~dużej złożoności strukturalnej.

\subsubsection{Parser JSON (uproszczony)}

Zaimplementowano parser dla~podzbioru formatu JSON, demonstrujący obsługę złożonych struktur danych zbudowanych z~elementów primitywnych.
Parser obsługuje wszystkie podstawowe typy danych JSON: wartości logiczne, wartości puste (\verb|null|), liczby zmiennoprzecinkowe,
napisy, obiekty (mapy klucz-wartość) oraz~tablice (listy elementów). Akcje semantyczne transformują tokeny w~struktury danych reprezentujące
obiekty i~tablice jako~typy Scali (\verb|Map[String, Any]| oraz~\verb|List[Any]|).

Lekser ignoruje białe znaki i~rozpoznaje znaki interpunkcji (\verb|{|, \verb|}|, \verb|[|, \verb|]|, \verb|:|, \verb|,|) jako~oddzielne tokeny.
Słowa kluczowe \verb|true|, \verb|false| i~\verb|null| są~tokenizowane odpowiednio jako~wartości logiczne i~puste.
Liczby są~rozpoznawane za~pośrednictwem wyrażenia regularnego obsługującego notację dziesiętną ze~znakami i~częściami ułamkowymi,
natomiast napisy są~wyodrębnianie poprzez dopasowanie tekstu w~cudzysłowach (z~obsługą znaków specjalnych poprzez backslash).

Parser definiuje hierarchię reguł produkcji: `Value` stanowi punkt wejścia i~rozpoznaje wszystkie typy wartości JSON,
zaś `Object` i~Array` są~regułami zagnieżdżonymi obsługującymi struktury złożone.
Reguła `ObjectMembers` obsługuje sekwencyjne oddzielone przecinkami pary klucz-wartość,
natomiast `ArrayElements` obsługuje listy elementów oddzielane przecinkami.
Rekurencja lewostronna w~regułach `ObjectMembers` i~ArrayElements` pozwala na~parsowanie dowolnie długich struktur,
zaś parsowanie jest jednoznaczne dzięki determinizmowi składni JSON.

System weryfikowany był na~czterech poziomach złożoności:

\begin{itemize}
    \item Test primitywnych wartości --- proste wyrażenie \verb|true| zwracające wartość logiczną;

    \item Test struktur zagnieżdżonych --- złożony obiekt JSON zawierający pola prymitywne oraz~zagnieżdżone obiekty i~tablice,
          takie jak~struktury zawierające dane osobowe (imię, wiek, kursy, adres z~ulicą i~kodem pocztowym),
          demonstrujące prawidłową rekonstrukcję głębokich struktur danych;

    \item Test tablic obiektów --- tablica zawierająca wiele obiektów o~jednolitej strukturze,
          taka jak~lista rekordów z~polami identyfikacyjnymi i~nazwami, potwierdzająca obsługę iteracyjnych struktur;

    \item Test rzeczywistej złożoności --- struktura naśladująca rzeczywisty format konfiguracyjny
          (menu z~zagnieżdżonymi polami popup i~listami elementów menu z~akcjami),
          zawierająca trzy poziomy zagnieżdżenia i~mieszaniny obiektów oraz~tablic.
\end{itemize}

Wszystkie testy przechodzą pomyślnie, potwierdzając poprawność implementacji leksera w~obsłudze białych znaków i~znaków specjalnych,
prawidłowość parsowania struktur rekurencyjnych, prawidłową transformację danych przez~akcje semantyczne
oraz~odporność systemu na~wyrażenia JSON o~arbitralnej złożoności strukturalnej.


\subsection{Benchmarki wydajności}
\label{subsec:org-benchmarks}

Przeprowadzono benchmarki porównujące wydajność wygenerowanego parsera \emph{ALPACA} z~innymi podejściami.

Testy wydajnościowe obejmowały dwa rodzaje gramatyk (wyrażenia arytmetyczne oraz format JSON), dla~których przygotowano zarówno dane o~strukturze iteracyjnej (płaskiej), jak~i~rekurencyjnej (głęboko zagnieżdżonej).
Pomiary wykonano dla różnych rozmiarów danych wejściowych: 100, 500, 1000 oraz 2000~elementów.

System \emph{ALPACA} porównano z~biblioteką SLY (Python) reprezentującą podejście oparte na~refleksji oraz~z~FastParse (Scala) reprezentującą kombinatory parserów.
Dla każdego narzędzia mierzono czas leksykalizacji, czas parsowania oraz całkowity czas przetwarzania.

Szczegółowe wyniki testów wydajnościowych, metodologia badań oraz analiza porównawcza przedstawione są~w~rozdziale~\ref{ch:comp}.

\subsection{Walidacja poprawności}
\label{subsec:org-validation}

Walidacja gramatyk odbywała się w~trzech etapach:

\begin{itemize}
    \item Testy weryfikujące poprawną detekcję konfliktów LR(1).
    \item Testy sprawdzające jakość komunikatów błędów w~przypadku nierozwiązanych konfliktów.
    \item Testy potwierdzające obsługę lewostronnej rekurencji przez parser LR(1).
\end{itemize}

\noindent Walidacja typów odbywała się w~trzech etapach:

\begin{itemize}
    \item Sprawdzanie, czy typy rafinowane poprawnie odwzorowują zdefiniowane tokeny.
    \item Sprawdzanie poprawności typów akcji semantycznych w~kontekście wartości wydobywanych z~tokenów.
    \item Testy negatywne, weryfikujące odrzucanie niezgodnych typowo definicji przez system typów.
\end{itemize}


\printbibliography
\listoffigures
\listoftables
% \listofalgorithmes
\lstlistoflistings

\end{document}
