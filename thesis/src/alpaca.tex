%%%%%% -*- Coding: utf-8-unix; Mode: latex

\documentclass[polish]{aghengthesis}

\usepackage{polski}
\usepackage[utf8]{inputenc}
\usepackage{url}
\usepackage{subfigure}
\usepackage{ragged2e}
\usepackage{multirow}
\usepackage{grffile}
\usepackage{indentfirst}
\usepackage{caption}
\usepackage{listings}
\usepackage[ruled,linesnumbered,lined]{algorithm2e}
\usepackage[bookmarks=false]{hyperref}

\usepackage{tabularx}
\usepackage{booktabs}
\usepackage{array}
\usepackage[table]{xcolor}

\hypersetup{colorlinks,
    linkcolor=blue,
    citecolor=blue,
    urlcolor=blue}

\usepackage[svgnames]{xcolor}
\usepackage{inconsolata}

\usepackage{csquotes}
\DeclareQuoteStyle[quotes]{polish}
{\quotedblbase}
{\textquotedblright}
[0.05em]
{\quotesinglbase}
{\fixligatures\textquoteright}
\DeclareQuoteAlias[quotes]{polish}{polish}

\usepackage[nottoc]{tocbibind}

\usepackage[
    style=numeric,
    sorting=none,
    isbn=false,
    doi=true,
    url=true,
    backref=false,
    backrefstyle=none,
    maxnames=10,
    giveninits=true,
    abbreviate=true,
    defernumbers=false,
    backend=biber]{biblatex}
\addbibresource{bibliografia.bib}
% \bibliographystyle{plain}

\lstdefinelanguage{terminal}{
    breaklines=true,
    breakatwhitespace=false,
}

\lstdefinelanguage{Scala}{
    morekeywords={
        abstract,case,catch,class,def,do,else,
        enum,export,extends,false,final,finally,for,
        given,if,implicit,import,lazy,match,new,
        null,object,override,package,private,protected,return,
        sealed,super,then,throw,trait,true,try,
        type,val,var,while,with,yield,
        as,derives,end,extension,infix,inline,opaque,open,transparent,using},
    otherkeywords={<-,=>,<:,>:,@,=>>,?=>,|,*},
    sensitive=true,
    morecomment=[l]{//},
    morecomment=[n]{/*}{*/},
    morecomment=[n]{/**}{*/},
    morestring=[b]",
    morestring=[b]"""
}[keywords,comments,strings]

\usepackage{fontspec}
\setmonofont{JetBrains Mono}[Contextuals=Alternate]

\lstset{
    basicstyle=\ttfamily\footnotesize,
    backgroundcolor=\color{white},
    commentstyle=\it\color{Green},
    keywordstyle=\color{Red},
    stringstyle=\color{Blue},
    numberstyle=\tiny\color{Black},
    escapeinside=`',
    frame=single,
    tabsize=2,
    rulecolor=\color{black!30},
    title=\lstname,
    breaklines=true,
    breakatwhitespace=true,
    framextopmargin=2pt,
    framexbottommargin=2pt,
    extendedchars=false,
    captionpos=b,
    abovecaptionskip=5pt,
    keepspaces=true,
    numbers=left,
    numbersep=5pt,
    showspaces=false,
    showstringspaces=false,
    showtabs=false,
    tabsize=2
}

\SetAlgorithmName{\LangAlgorithm}{\LangAlgorithmRef}{\LangListOfAlgorithms}
\newcommand{\listofalgorithmes}{\tocfile{\listalgorithmcfname}{loa}}

\renewcommand{\lstlistingname}{\LangListing}
\renewcommand\lstlistlistingname{\LangListOfListings}

\renewcommand{\lstlistoflistings}{\begingroup
\tocfile{\lstlistlistingname}{lol}
\endgroup}

% Definicje nowych rodzajów kolumn w tabeli
\newcolumntype{C}{>{\centering\arraybackslash}m{0.15\linewidth}}
\newcolumntype{L}{>{\centering\arraybackslash}m{0.15\linewidth}}

\captionsetup[figure]{skip=5pt,position=bottom}
\captionsetup[table]{skip=5pt,position=top}

%%%%%%%%%%%%%%%%%%%%%%%%%%%%%%%%%%%%%%%%%%%%%%%%%%%%%%%%%%%%%%%%%%%%%%%%%%%%%%%
\author{Bartosz Buczek, Bartłomiej Kozak}

\titlePL{Implementacja narzędzi lex i yacc z wykorzystaniem metaprogramowania}
\titleEN{Implementation of lexical analyzer (lex) and parser generator (yacc) tools using metaprogramming techniques}

\fieldofstudy{Informatyka}

%\typeofstudies{Stacjonarne}

\supervisor{dr inż.\ Tomasz Służalec}

\date{\the\year}

%%%%%%%%%%%%%%%%%%%%%%%%%%%%%%%%%%%%%%%%%%%%%%%%%%%%%%%%%%%%%%%%%%%%%%%%%%%%%%%
\begin{document}

    \maketitle

    \tableofcontents

    \chapter{Cel Wizja}
\label{ch:cel-wizja}


\section{Charakterystyka problemu}
\label{sec:charakterystyka-problemu}
Leksery i parsery są kluczowymi elementami w procesie tworzenia interpreterów i kompilatorów języków programowania.
Pozwalają one przekształcić kod źródłowy napisany przez programistę na reprezentację wewnętrzną, wykorzystywaną później przez dalsze etapy przetwarzania kodu.

Analiza leksykalna wykonywana przez lekser polega na rozdzieleniu kodu źródłowego na jednostki logiczne, zwane leksemami.
Parser natomiast wykonuje analizę składniową w celu ustalenia struktury gramatycznej tekstu i jej zgodności z gramatyką języka.

Celem pracy inżynierskiej jest stworzenie narzędzia \textit{ALPACA} (Another Lexer Parser And Compiler Alpaca) w języku Scala, które implementuje funkcjonalności powszechnie stosowane w budowie lekserów i parserów.


\section{Motywacja projektu}
\label{sec:motywacja-projektu}

Projekt ma na celu stworzenie nowoczesnego narzędzia do generowania lekserów i parserów w języku Scala, łączącego zalety istniejących rozwiązań z nowoczesnym podejściem technologicznym.
Jego główne cele to:
\begin{enumerate}
    \item Stworzenie intuicyjnego API\@.
    \item Opracowanie obszernej dokumentacji.
    \item Rozbudowana diagnostyka błędów.
    \item Poprawa wydajności względem rozwiązań w języku Python.
    \item Integracja z popularnymi środowiskami programistycznymi (IDE).
\end{enumerate}

Proponowane rozwiązanie łączy nowoczesne podejście technologiczne z praktycznym zastosowaniem w edukacji i programowaniu.
Może on służyć jako narzędzie dydaktyczne, ułatwiając naukę teorii kompilacji, w pracach badawczych, a także jako kompleksowe narzędzie do tworzenia praktycznych rozwiązań.


\section{Przegląd istniejących rozwiązań}
\label{sec:przeglad-istniejacych-rozwiazan}

Dostępne na rynku rozwiązania umożliwiają tworzenie analizatorów, jednak charakteryzują się ograniczeniami związanymi z wydajnością, wysokim progiem wejścia i diagnostyką błędów.

\subsection{Lex, Yacc}
\label{subsec:lex-yacc}

\textit{Lex}\cite{lesk1975lex} i \textit{Yacc}\cite{johnson1975yacc} to klasyczne, dobrze ugruntowane narzędzia, które odegrały kluczową rolę w tworzeniu setek współczesnych języków programowania.
Definicja leksera i parsera w tych systemach odbywa się poprzez specjalnie zaprojektowaną składnię konfiguracyjną.
Mimo pewnych zalet, jego złożoność i wysoki próg wejścia mogą stanowić wyzwanie.

Ponieważ \textit{Lex} i \textit{Yacc} zostały zaprojektowane do współpracy z językiem C, ich integracja z nowoczesnymi językami programowania bywa utrudniona.
Rozszerzanie tych narzędzi o dodatkowe, specyficzne funkcjonalności jest skomplikowane, co ogranicza ich elastyczność.
Brak wsparcia dla współczesnych środowisk programistycznych (IDE) dodatkowo obniża komfort użytkowania w porównaniu z nowoczesnymi alternatywami.

\begin{lstlisting}[language=c,float=!htbp,caption={Fragment definicji parsera Ruby w technologii Yacc},label={lst:lstlisting2}]
        | command_asgn
    | mlhs '=' command_call
        {
        /*%%%*/
        value_expr($3);
        $1->nd_value = $3;
        $$ = $1;
        /*%
        $$ = dispatch2(massign, $1, $3);
        %*/
        }
    | var_lhs tOP_ASGN command_call
        {
        value_expr($3);
        $$ = new_op_assign($1, $2, $3);
        }
    | primary_value '[' opt_call_args rbracket tOP_ASGN command_call
        {
        /*%%%*/
        NODE *args;

        value_expr($6);
        if (!$3) $3 = NEW_ZARRAY();
        args = arg_concat($3, $6);
        if ($5 == tOROP) {
            $5 = 0;
        }
        else if ($5 == tANDOP) {
            $5 = 1;
        }
        $$ = NEW_OP_ASGN1($1, $5, args);
        fixpos($$, $1);
        /*%
        $$ = dispatch2(aref_field, $1, escape_Qundef($3));
        $$ = dispatch3(opassign, $$, $5, $6);
        %*/
        }
    | primary_value '.' tIDENTIFIER tOP_ASGN command_call
        {
        value_expr($5);
        $$ = new_attr_op_assign($1, ripper_id2sym('.'), $3, $4, $5);
        }
    | primary_value '.' tCONSTANT tOP_ASGN command_call
        {
        value_expr($5);
        $$ = new_attr_op_assign($1, ripper_id2sym('.'), $3, $4, $5);
        }
\end{lstlisting}

\subsection{PLY, SLY}
\label{subsec:ply-sly}

\textit{PLY}\cite{ply} i jego nowszy odpowiednik \textit{SLY}\cite{sly} to biblioteki inspirowane narzędziami Lex i Yacc.
Oferują elastyczne podejście do budowy parserów, umożliwiając samodzielną implementację obsługi leksemów, budowę drzewa AST, czy dodatkowe funkcjonalności takie jak obliczanie numeru linii w lekserze.

Głównym ograniczeniem PLY i SLY jest implementacja w języku Python.
Ze względu na interpretowany charakter oraz dynamiczne typowanie, parsery te charakteryzują się niską wydajnością, a brak statycznego typowania utrudnia wykrywanie błędów na etapie kompilacji.
Przy implementacji parserów z użyciem biblioteki SLY w środowisku PyCharm obserwuje się wiele ostrzeżeń dotyczących potencjalnych naruszeń reguł, co często wymaga zastosowania mechanizmów supresji, aby uniknąć fałszywie pozytywnych wyników analizy statycznej kodu.
Ponadto należy zaznaczyć, iż autor projektu informuje o braku dalszego rozwoju tych narzędzi\cite{sly-github}.

Przykład ilustruje kilka nieintuicyjnych, automatycznych mechanizmów obecnych w bibliotece \textit{SLY}.
\begin{itemize}
    \item Operator \verb|@_()| jest zdefiniowany, aby automatycznie analizować tekst przy pomocy wyrażeń regularnych.
    Literały muszą być zawarte w cudzysłowie, a „zmienna” odpowiada za matchowany „typ”.
    \item Nazwa metody oznacza „typ” zwracany przez daną produkcję, czyli dla definicji \verb|IF| należy najpierw odszukać wszystkie metody, które mają nazwę \verb|condition|, gdyż są to możliwe produkcje.
    \item W krotce (sic!) \verb|precedence| definiujemy pierwszeństwo operatorów, jednakże dodanie \verb|% prec| pozwala nadpisać priorytet dla konkretnej reguły składniowej.
    \item  Argument \verb|p| pozwala na dostęp do kontekstu produkcji (np. numeru linii), ale także do zmiennych w patternu match w adnotacji.
    Jeśli zdefiniowany jest więcej niż jeden, to dodajemy numer do accesora, np. \verb|expr1| jest odwołaniem się do drugiego wyrażenie \verb|expr|.
    Jednocześnie, można to zrobić także poprzez odwołanie się do konkretnego indeksu obiektu \verb|p|.
\end{itemize}

\begin{lstlisting}[language=Python,float=!htbp,caption={Fragment definicji parsera w Pythonie, wykorzystujacy bibliotekę SLY},label={lst:lstlisting3}]
    class MatrixParser(Parser):
    tokens = MatrixScanner.tokens

    precedence = (
        ('nonassoc', 'IFX'),
        ('nonassoc', 'ELSE'),
        ('nonassoc', 'EQUAL'),
    )

    @_('"{" instructions "}"')
    def block(self, p: YaccProduction):
        raise NotImplementedError

    @_('instruction')
    def block(self, p: YaccProduction):
        raise NotImplementedError

    @_('IF "(" condition ")" block %prec IFX')
    def instruction(self, p: YaccProduction):
        raise NotImplementedError

    @_('IF "(" condition ")" block ELSE block')
    def instruction(self, p: YaccProduction):
        raise NotImplementedError

    @_('expr EQUAL expr')
    def condition(self, p: YaccProduction):
        args = [p.expr0, p.expr1]
        raise NotImplementedError

\end{lstlisting}

Komunikaty błędów w bibliotece \textit{SLY} są bardzo ograniczone.
Na przykład poniższy kod %todo(bkozak: lepsze wprowadzenie kodu)
\begin{lstlisting}[language=Python,float=!htbp,caption={Fragment niedziałajacego kodu w Pythonie, wykorzystujacy bibliotekę SLY},label=lst:maximum]
tokens = Scanner().tokenize("a = 1 + 2")
for tok in tokens:
    print(tok)

\end{lstlisting}

po uruchomieniu informuje użytkownika
\begin{lstlisting}[language=terminal, caption={Przykład komunikatu błędu w bibliotece \textit{SLY}},label={lst:lstlisting}]
  File "main.py", line 2, in <module>
    for tok in tokens:
               ^^^^^^
  File "Python\site-packages\sly\lex.py", line 374, in tokenize
    _set_state(type(self))
    ~~~~~~~~~~^^^^^^^^^^^^
  File "Python\site-packages\sly\lex.py", line 367, in _set_state
    _master_re = cls._master_re
                 ^^^^^^^^^^^^^^
AttributeError: type object 'Scanner' has no attribute '_master_re'
\end{lstlisting}

Okazuje się, że problemem był brak atrybutu \verb|ignore_comment| w definicji \verb|Lexer|.

\subsection{ANTLR}
\label{subsec:antlr}
%todo(tosz): a gdzie jakaś bibliografia do tego?

\textit{ANTLR} to kolejne rozwiązanie inspirowane narzędziami \textit{Lex} i \textit{Yacc}, oferujące zaawansowane mechanizmy analizy składniowej.
Jego twórcy opracowali dedykowany język DSL, znany jako Grammar v4, który umożliwia definiowanie składni analizowanego języka.
Na podstawie tej definicji \textit{ANTLR} generuje parser w wybranym przez użytkownika języku programowania, takim jak Python, Java, C++ lub JavaScript.

Wspomaganie pracy z \textit{ANTLR} w znacznym stopniu ułatwiają dedykowane wtyczki do środowisk Visual Studio Code oraz IntelliJ IDEA. Oferują one funkcjonalności, takie jak kolorowanie składni, autouzupełnianie kodu, nawigację do definicji leksemów oraz walidację błędów, co znacząco przyspiesza proces tworzenia parserów.

Jedną z kluczowych różnic \textit{ANTLR} w porównaniu do innych narzędzi jest wykorzystanie gramatyki LL(*), podczas gdy klasyczne rozwiązania, takie jak Yacc czy SLY, implementują LALR(1).
LL(*) jest bardziej intuicyjna i czytelna dla programistów, co ułatwia definiowanie reguł składniowych.
Jednakże, jej zastosowanie wiąże się z większym zużyciem pamięci oraz niższą wydajnością w porównaniu do LALR(1).

Dodatkowym wyzwaniem podczas korzystania z \textit{ANTLR} jest konieczność nauki składni DSL Grammar v4 oraz ograniczenie wsparcia dla narzędzi deweloperskich.
Pełne wykorzystanie możliwości \textit{ANTLR} wymaga korzystania z jednego z dedykowanych środowisk, co może stanowić istotne ograniczenie dla użytkowników preferujących inne IDE\@.

\subsection{Alternatywne rozwiązania w języku Scala}
\label{subsec:rozwiazania-w-jezyku-scal}

\subsubsection{Scala parser combinators}
\label{subsubsec:scala-parser-combinators}

Biblioteka \textit{Scala parser combinators}\cite{moors2008parser} była popularnym sposobem na tworzenie parserów, lecz jak wynika z dokumentacji, „Trudno jest jednak zrozumieć ich działanie i jak zacząć.
Po skompilowaniu i uruchomieniu kilku pierwszych przykładów, mechanizm działania staje się bardziej zrozumiały, ale do tego czasu może to być zniechęcające, a standardowa dokumentacja nie jest zbyt pomocna”\cite{parser-combinators-readme}.

\subsubsection{ScalaBison}
\label{subsubsec:scala-bison}

Z podsumowania artykułu na temat \textit{ScalaBison}\cite{boyland2010tool} wiadomo, że to praktyczny generator parserów dla języka Scala oparty na technologii rekurencyjnego wstępowania i zstępowania, który akceptuje pliki wejściowe w formacie \textit{bison}.
Parsery generowane przez \textit{ScalaBison} używają bardziej informacyjnych komunikatów o błędach niż te generowane przez pierwowzór \textit{bison}, a także szybkość parsowania i wykorzystanie miejsca są znacznie lepsze niż \textit{scala-combinators}, ale są nieco wolniejsze niż najszybsze generatory parserów oparte na JVM.

Dodatkowo należy zaznaczyć, iż jest to rozwiązanie już niewspierane i stworzone w celach akademickich.
Korzysta z przestarzałej wersji Scali, nie posiada wyczerpującej dokumentacji i liczba funkcjonalności jest bardzo ograniczona w porównaniu do np. technologii \textit{SLY}.

\subsubsection{parboiled2}
\label{subsubsec:parboiled-2}

\textit{parboiled2}\cite{myltsev2019parboiled2} to biblioteka w Scali umożliwiająca lekkie i szybkie parsowanie dowolnego tekstu wejściowego.
Implementuje ona oparty na makrach generator parsera dla gramatyk wyrażeń parsujących (PEG), który działa w czasie kompilacji i tłumaczy definicję reguły gramatycznej na odpowiadający jej bytecode JVM. Niestety próg wejścia ze względu na skomplikowany i nieintuicyjny DSL jest wysoki.
Zgodnie z przykładem poniżej, raportowanie błędów jest bardzo ograniczone (problem z implementacją wynika jedynie z różnic w liczbie parametrów funkcji).
%todo(bkozak): lepsze linkowanie

\begin{lstlisting}[language=terminal, caption={
    Fragment (sic!) błędu wygenerowanego przez bibliotekę \textit{parboiled2}, który pochodzi z prezentacji Li Haoyi na temat \textit{FastParse}\cite{fastparse-talk}.
},label={lst:lstlisting4}]
    [error] /Users/haoyi/Dropbox (Personal)/(...)/Parser.scala:16: type mismatch;
    [error] found : shapeless.::[Int,shapeless.::[scalatex.stages.Ast.Block,shapeless.HNil]]
    required: scalatex.stages.Ast.Block new Parser(input, offset).Body.run().get
    [error]
    [error] /Users/haoyi/Dropbox (Personal)/(...)/Parser.scala:60: overloaded
    method value apply with alternatives:
    [error] [I, J, K, L, M, N, O, P, Q, R, S, T, U, V, W, X, Y, Z, RR](f: (I, J, K, L, M, N, O, P, Q, R, S, T, U, V, W, X, Y, Z, scalatex.stages.Ast. Block.
    Text, scalatex.stages.Ast.Chain, Int, scalatex.stages.Ast.Block) => RR)(implicit j: org.parboiled2.support.ActionOps.SJoin[shapeless.::[!, shapeless.::[J,shapeless.::[K,shapeless.::[L,shapeless.::[M,shapeless.:: [N,shapeless.::[O,shapeless.::[P,shapeless.::[Q,shapeless.::[R, shapeless.::[S,shapeless.::[T,shapeless.::[U,shapeless.::[V,shapeless.:: [W,shapeless.::[X,shapeless.::[Y,shapeless.::[Z,shapeless.
    Ni/////], shapeless.HNil,RR], implicit c: org.parboiled2.support.FCapture[(I, J, K, L, M, N, O, P, Q, R, S, T, U, V, W, X, Y, Z, scalatex. stages.Ast.Block.Text, scalatex.stages.Ast.Chain, Int, scalatex.stages.Ast.Block) => RR])org.parboiled2.Rule[j.In,j.Out] <and>
    [error] [J, K, L, M, N, O, P, Q, R, S, T, U, V, W, X, Y, Z, RR](f: (J, K, L, M, N, O, P, Q, R, S, T, U, V, W, X, Y, Z, scalatex.stages.Ast.Block. Text, scalatex.stages.Ast.Chain, Int, scalatex.stages.Ast.Block) => RR)(implicit j: org.parboiled2.support.ActionOps.SJoin[shapeless.::[J, shapeless.::[K,shapeless.::[L,shapeless.::[M,shapeless.::[N,shapeless.:: [O,shapeless.::[P,shapeless.::[Q,shapeless.::[R,shapeless.::[S, shapeless.::[T,shapeless.::[U,shapeless.::[V,shapeless.:: [W,shapeless.::[X,shapeless
\end{lstlisting}

\subsubsection{FastParse}
\label{subsubsec:fastparse}

FastParse\textit{FastParse}\cite{fastparse-docs}, opracowana przez Li Haoyi, to wysokowydajna biblioteka kombinatorów parserów dla Scali, zaprojektowana w celu uproszczenia tworzenia parserów tekstu strukturalnego.
Umożliwia ona programistom definiowanie parserów rekurencyjnych, dzięki czemu nadaje się do parsowania języków programowania, formatów danych, takich jak JSON, czy DSL-i.
Cechą charakterystyczną FastParse jest równowaga między użytecznością a wydajnością.
Parsery są konstruowane poprzez łączenie mniejszych parserów za pomocą operatorów, takich jak \verb&~& dla sekwencjonowania i \verb&\|& dla alternatyw, przy jednoczesnym zachowaniu czytelności zbliżonej do formalnych definicji gramatyki.
Według dokumentacji\cite{fastparse-docs}, parsery \textit{Fastparse} zajmują 1/10 kodu w porównaniu do ręcznie napisanego parsera rekurencyjnego.
W porównaniu do narzędzi generujących parsery, takich jak \textit{ANTLR} lub \textit{Lex} i \textit{Yacc}, implementacja nie wymaga żadnego specjalnego kroku kompilacji lub generowania kodu.
To sprawia, że rozpoczęcie pracy z \textit{Fastparse} jest znacznie łatwiejsze niż w przypadku bardziej tradycyjnych narzędzi do generowania parserów.
Przykładowo, parser wyrażeń arytmetycznych może być zwięźle napisany, aby obsługiwać zagnieżdżone nawiasy, pierwszeństwo operatorów i raportowanie błędów w mniej niż 20 liniach kodu\cite{fastparse-slides}.
Biblioteka kładzie również nacisk na debugowanie, generując szczegółowe komunikaty o błędach, które wskazują dokładną lokalizację i przyczynę niepowodzeń parsowania, takich jak niedopasowane nawiasy lub nieprawidłowe tokeny.

\subsection{Podsumowanie}
\label{subsec:podsumowanie}

\begin{table}[ht]
    \centering
    \begin{tabular}{L|CCCC}
        \toprule
        \large{Narzędzie}       & \textbf{Lex\&Yacc}                        & \textbf{PLY/SLY}     & \textbf{ANTLR}     & \textbf{scala-bison} \\
        \midrule
        Język implementacji     & C                                         & Python               & Java               & Scala (nad Bisonem)  \\
        \arrayrulecolor{gray}
        \hline
        Język użycia            & regex, BNF, akcje w C                     & DSL                  & DSL oparty na EBNF & BNF, akcje w Scali   \\
        \hline
        Wydajność               & wysoka                                    & niska                & umiarkowana        & wysoka               \\
        \hline
        Łatwość użycia          & średnia                                   & umiarkowana          & wysoka             & średnia              \\
        \hline
        Aktywne wsparcie        & brak                                      & nie                  & tak                & nie                  \\
        \hline
        Diagnostyka błędów      & słaba                                     & średnia              & dobra              & słaba                \\
        \hline
        Dokumentacja            & dobra                                     & średnia, nieaktualna & dobra              & słaba                \\
        \hline
        Popularność             & wysoka                                    & średnia              & wysoka             & niska                \\
        \hline
        Integracja IDE          & nieoficjalny plugin                       & ograniczona          & oficjalny plugin   & brak                 \\
        \hline
        Wsparcie do debugowania & brak                                      & dobre                & częściowe          & dobre                \\
        \hline
        Generowania kodu        & nie                                       & nie                  & tak                & nie                  \\
        \hline
        \toprule
        Narzędzie               & \textbf{Scala parser\newline combinators} & \textbf{parboiled2}  & \textbf{FastParse} & \textbf{ALPACA} \\
        \midrule
        Język implementacji     & Scala                                     & Scala                & Scala              & Scala                \\
        \hline
        Język użycia            & DSL w Scali                               & DSL w Scali          & DSL w Scali        & Scala                \\
        \hline
        Wydajność               & wysoka                                    & umiarkowana          & wysoka             & TODO                 \\
        \hline
        Łatwość użycia          & niska                                     & średnia              & średnia            & TODO                 \\
        \hline
        Aktywne wsparcie        & nie                                       & nie                  & tak                & TODO                 \\
        \hline
        Diagnostyka błędów      & dobra                                     & niska                & dobra              & TODO                 \\
        \hline
        Dokumentacja            & słaba                                     & bardzo dobra         & bardzo dobra       & TODO                 \\
        \hline
        Popularność             & średnia                                   & niska                & rosnąca            & TODO                 \\
        \hline
        Integracja IDE          & wsparcie dla Scali                        & wsparcie dla Scali   & wsparcie dla Scali & TODO                 \\
        \hline
        Wsparcie do debugowania & dobre                                     & dobre                & dobre              & TODO                 \\
        \hline
        Generowania kodu        & nie                                       & nie                  & nie                & TODO                 \\
        \bottomrule
    \end{tabular}
    \caption{Porównanie wybranych narzędzi do generowania lekserów i parserów}
    \label{tab:porownanie-alternatyw}
\end{table}


    \chapter{Zakres funkcjonalności}
\label{ch:zakres-funkcjonalnosci}


\section{Wymagania funkcjonalne}
\label{sec:wymagania-funkcjonalne}

\begin{itemize}

    \item 1. Definicja parsera i gramatyki
    \begin{itemize}
        \item Możliwość definiowania reguł gramatycznych za pomocą czystego języka Scala z obsługą wyrażeń regularnych i hierarchii reguł (bez DSL).
        \item Generowanie drzewa składniowego: Automatyczne tworzenie struktury AST (Abstract Syntax Tree) z formatu EBNF\@.
        \item Możliwość definiowania reguł rekurencyjnych.
        \item Precyzyjne raportowanie błędów z lokalizacją w źródle, sugestiami napraw i trybem odzyskiwania (np.\ pomijanie błędnych tokenów).
        \item Obsługa precedencji i asocjacji operatorów.
    \end{itemize}

    \item \item 2. Integracja z narzędziami developerskimi
    \begin{itemize}
        \item Obsługa podświetlania składni, autouzupełniania i debugowania w środowiskach takich jak IntelliJ lub VS Code.
    \end{itemize}

    \item 3. Optymalizacje wydajnościowe
    \begin{itemize}
        \item Obsługa dużych plików poprzez przetwarzanie danych strumieniowo.
        \item Przechowywanie skompilowanych gramatyk w pamięci w celu przyspieszenia powtarzalnych operacji.
    \end{itemize}

    \item 4. Rozszerzalność
    \begin{itemize}
        \item Możliwość dołączania funkcji wykonywanych podczas parsowania (np.\ walidacja kontekstowa).
        \item Podział implementacji na niezależne komponenty (np.\ lexer, parser).
        \item Mechanizmy transformacji AST: struktury TreeMap oraz TreeTraverser.
    \end{itemize}

    \item
\end{itemize}


\section{Wymagania niefunkcjonalne}
\label{sec:wymagania-niefunkcjonalne}

\begin{itemize}

    \item \item 1. Wydajność
    \begin{itemize}
        \item Czas parsowania zbliżony lub lepszy od czasów narzędzi \textit{FastParse} i \textit{Sly}.
        \item Zużycie pamięci stałe, niezależne od rozmiaru parsowanego pliku.
    \end{itemize}

    \item 2. Skalowalność
    \begin{itemize}
        \item Obsługa dużych gramatyk.
        \item Modularność.
    \end{itemize}

    \item 3. Niezawodność
    \begin{itemize}
        \item Testy regresyjne: Pokrycie kodu testami ≥ 90\%, z automatycznym uruchamianiem po każdej zmianie.
    \end{itemize}

    \item 4. Kompatybilność
    \begin{itemize}
        \item Wsparcie dla IntelliJ i VS Code (Metals).
        \item Systemy operacyjne: Działanie na Windows, Linux i macOS bez modyfikacji kodu.
    \end{itemize}

    \item 5. Użyteczność
    \begin{itemize}
        \item Obszerna dokumentacja z przykładami.
        \item Narzędzia diagnostyczne, np.\ wizualizacja drzewa AST\@.
        \item Weryfikacja poprawności gramatyk na poziomie typów.
    \end{itemize}

\end{itemize}


\section{Charakterystyka użytkowników}
\label{sec:charakterystyka-uzytkownikow}

\textit{Alpaca} jest skierowana do wszystkich zajmujących się aalizą składniową, lub gramatyczną, oraz tworzeniem języków formalnych, w szczególności języków programowania.
W zależności od kontekstu można wyróżnić następujące grupy docelowe:

\begin{itemize}
    \item \textbf{Programiści tworzący języki domenowe (DSL):}
    Specjaliści projektujący rozwiązania specyficzne dla danego problemu (np.
    konfiguratory\footnote{Konfigurator to specjalizowany język lub narzędzie umożliwiające deklaratywne definiowanie zachowania systemu lub aplikacji, najczęściej poprzez pliki konfiguracyjne. Przykłady to \texttt{docker-compose.yml} czy pliki konfiguracji serwisów CI/CD.},
    silniki reguł\footnote{Silnik reguł (ang. \textit{rule engine}) to system przetwarzający zbiory warunków logicznych i wyzwalający odpowiednie akcje w oparciu o zdefiniowane reguły. Typowym zastosowaniem jest automatyzacja procesów biznesowych. Przykłady: Drools, CLIPS.},
    ), dla których kluczowa jest elastyczność definiowania składni. Dla tej grupy istotna jest także niezawodność oraz wydajność procesu parsowania.

    \item \textbf{Studenci kierunków technicznych:}
    Osoby uczące się podstaw teorii kompilacji, które korzystają z narzędzi typu lekser/parser w celach dydaktycznych.
    Dla tej grupy kluczowe znaczenie ma intuicyjny interfejs programistyczny (API), rozbudowana dokumentacja oraz możliwość szybkiego uruchomienia przykładów bez konieczności ręcznej konfiguracji środowiska.

    \item \textbf{Nauczyciele akademiccy i prowadzący zajęcia laboratoryjne:}
    Osoby przygotowujące kursy dotyczące języków programowania, parserów, automatów i kompilatorów.
    \textit{Alpaca} może służyć jako narzędzie wspierające zajęcia, umożliwiające praktyczną demonstrację działania gramatyk oraz parserów w sposób prostszy i szybszy niż za pomocą niskopoziomowych narzędzi, takich jak Lex i Yacc.

    \item \textbf{Entuzjaści języków programowania i narzędzi deweloperskich:}
    Osoby zainteresowane eksperymentowaniem z nowymi technologiami, tworzeniem i rozwojem własnych języków lub eksploracją działania analizatorów leksykalnych i składniowych.
    Ta grupa użytkowników ceni sobie możliwość rozszerzania i dostosowywania narzędzia do własnych potrzeb.
\end{itemize}

Uwzględnienie potrzeb i oczekiwań powyższych grup użytkowników stanowiło główny cel projektowania architektury systemu oraz definiowania jego funkcjonalności.

\section{Scenariusze użytkowania i testowania}
\label{sec:scenariusze-uzytkowania}

Poniższe scenariusze prezentują typowe przypadki użycia biblioteki \textit{Alpaca}, wraz z towarzyszącymi im kryteriami poprawności oraz celami testowymi.

\subsection*{Scenariusz 1: Definicja parsera}

\textbf{Cel:} Zdefiniowanie parsera dla prostego języka wyrażeń arytmetycznych.\\

\textbf{Kroki:}
\begin{enumerate}
    \item Użytkownik definiuje leksemy (np. liczby, operatory arytmetyczne).
    \item Tworzy nieterminale (np. \texttt{wyrazenie}, \texttt{iloczyn}) oraz odpowiadające im reguły gramatyczne (np. \texttt{potega = podstawa '**' wykladnik}).
    \item Deklaruje funkcje ewaluacyjne dla odpowiednich produkcji (np. \texttt{pow(base, exponent)} dla potęgowania).
    \item Uruchamia parser dla przykładowego ciągu wejściowego.
\end{enumerate}

\textbf{Oczekiwany rezultat:} Parser generuje poprawne drzewo składniowe (AST) oraz zwraca wynik zgodny z semantyką definiowanej gramatyki.\\

\textbf{Przykład testowy:} \texttt{parse("2 + 2")} zwraca wartość \texttt{4} oraz drzewo składniowe reprezentujące operację dodawania z dwoma operandami liczbowymi.

\subsection*{Scenariusz 2: Obsługa błędów składniowych}

\textbf{Cel:} Weryfikacja zachowania parsera w przypadku niepoprawnych danych wejściowych.\\

\textbf{Kroki:}
\begin{enumerate}
    \item Użytkownik korzysta z przykładowej definicji parsera udostępnionej w dokumentacji lub przez prowadzącego.
    \item Parser uruchamiany jest na niepoprawnym ciągu wejściowym (np. brakujący nawias zamykający).
    \item System zwraca komunikat o błędzie z podaniem lokalizacji problemu oraz jego opisu.
\end{enumerate}

\textbf{Oczekiwany rezultat:} System zgłasza czytelny komunikat błędu zawierający informacje o miejscu i charakterze błędu składniowego.

\subsection*{Scenariusz 3: Weryfikacja deterministyczności gramatyki}

\textbf{Cel:} Ocena, czy zadana gramatyka jest deterministyczna i wolna od niejednoznaczności.\\

\textbf{Kroki:}
\begin{enumerate}
    \item Użytkownik konstruuje złożoną gramatykę zawierającą alternatywne reguły.
    \item Parser analizuje zachowanie dla potencjalnie niejednoznacznych ciągów wejściowych.
\end{enumerate}

\textbf{Oczekiwany rezultat:} System wykrywa i zgłasza potencjalne konflikty składniowe, informując o możliwej niejednoznaczności gramatyki.

\subsection*{Scenariusz 4: Testowanie wydajności dla dużych wejść}

\textbf{Cel:} Pomiar wydajności działania parsera w przypadku dużych danych wejściowych.\\

\textbf{Kroki:}
\begin{enumerate}
    \item Parser uruchamiany jest na ciągu wejściowym o rozmiarze kilkunastu megabajtów.
    \item Rejestrowany jest czas działania oraz poziom wykorzystania zasobów systemowych (pamięć operacyjna, CPU).
\end{enumerate}

\textbf{Oczekiwany rezultat:} Parser przetwarza dane w akceptowalnym czasie, nie generując błędów oraz nie przekraczając założonego limitu zużycia pamięci.


    \printbibliography
    \listoffigures
    \listoftables
    \listofalgorithmes
    \lstlistoflistings

\end{document}
