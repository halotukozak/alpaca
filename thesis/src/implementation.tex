\chapter{Implementacja}
\label{ch:implementacja}


\section{Praktyczna implementacja analizatora leksykalnego z wykorzystaniem makr w Scali 3}\label{sec:praktyczna-implementacja-analizatora-leksykalnego-z-wykorzystaniem-makr-w-scali-3}

\subsection{Wprowadzenie do studium przypadku}\label{subsec:wprowadzenie-do-studium-przypadku}

Niniejszy rozdział prezentuje praktyczną implementację systemu analizy leksykalnej (leksera) wykorzystującego zaawansowane mechanizmy metaprogramowania Scali 3.
Przedstawiony kod stanowi przykład zastosowania technik opisanych w poprzednich rozdziałach do rozwiązania rzeczywistego problemu inżynierskiego: automatycznej generacji wydajnego analizatora leksykalnego z definicji wysokopoziomowej w formie języka dziedzinowego (DSL).

System \texttt{alpaca.lexer} implementuje transformację deklaratywnych reguł tokenizacji zapisanych jako funkcja częściowa (ang. \textit{partial function}) w kod procedualny wykonywany w czasie kompilacji.
Wykorzystuje przy tym pełne spektrum możliwości refleksji TASTy, włączając generację klas w czasie kompilacji, transformację drzew AST oraz wyspecjalizowane typy refinement.

\subsection{Architektura systemu leksera}\label{subsec:architektura-systemu-leksera}

\subsubsection{Interfejs użytkownika}\label{subsubsec:interfejs-uzytkownika}

System oferuje użytkownikowi przejrzysty interfejs DSL oparty na dopasowaniu wzorców:

\lstinputlisting[language=scala,caption={Definicja typu LexerDefinition},label={lst:lexer-01-lexerdefinition}, linerange={21}]{../../src/alpaca/lexer/Lexer.scala}

Definicja \texttt{LexerDefinition} reprezentuje reguły leksera jako funkcję częściową mapującą wzorce wyrażeń regularnych (jako ciągi znaków) na definicje tokenów.
Wykorzystanie funkcji częściowej pozwala na naturalne wyrażenie reguł leksykalnych w idiomatycznej składni Scali.

Główny punkt wejścia systemu stanowi metoda \texttt{lexer}:

\lstinputlisting[language=scala,caption={Punkt wejścia: transparent inline def lexer},label={lst:lexer-02-entrypoint}, linerange={44-52}]{../../src/alpaca/lexer/Lexer.scala}

Modyfikator \texttt{transparent inline} zapewnia, że zwracany typ będzie dokładnie odpowiadał wygenerowanej strukturze, włączając typy refinement dla poszczególnych tokenów.
Użycie parametrów kontekstowych (\texttt{using}) realizuje wzorzec dependency injection na poziomie systemu typów.

\subsubsection{Implementacja makra}\label{subsubsec:implementacja-makra}

Makro przyjmuje wyrażenie reprezentujące reguły leksera jako \texttt{Expr[Ctx ?=> LexerDefinition[Ctx]]} oraz instancje kontekstualnych klas pomocniczych.
Parametr \texttt{using Quotes} dostarcza dostępu do API refleksji TASTy.

\subsection{Analiza drzewa składni abstrakcyjnej}\label{subsec:analiza-drzewa-skadni-abstrakcyjnej}

\subsubsection{Dekonstrukcja funkcji częściowej}\label{subsubsec:dekonstrukcja-funkcji-czesciowej}

Kluczowym krokiem implementacji jest ekstrakcja reguł z definicji funkcji częściowej:

\lstinputlisting[language=scala,caption={Dekonstrukcja funkcji częściowej (dopasowanie AST do CaseDef)},label={lst:lexer-04-extract-cases}, linerange={72}]{../../src/alpaca/lexer/Lexer.scala}

Ten fragment kodu wykorzystuje dopasowanie wzorców w \textit{quotes} do dekonstrukcji typowanego AST funkcji częściowej.
Struktura \texttt{Lambda(\_, Match(\_, cases))} odpowiada wewnętrznej reprezentacji funkcji częściowej, gdzie \texttt{Match} zawiera listę przypadków \texttt{CaseDef}.

\subsection{Transformacja i adaptacja referencji}\label{subsec:transformacja-i-adaptacja-referencji}

\subsubsection{Klasa replacerefs}\label{subsubsec:klasa-replacerefs}

Kluczową techniką jest zastąpienie referencji do starego kontekstu nowymi referencjami:

\lstinputlisting[language=scala,caption={Zastąpienie referencji starego kontekstu nowymi (ReplaceRefs)},label={lst:lexer-06-replace-with-new-ctx}, linerange={75-78}]{../../src/alpaca/lexer/Lexer.scala}

Transformacja ta realizuje proces znany jako \("\)re-owning\("\) w terminologii kompilatorów — zmianę właściciela (owner) symboli w AST. Jest to konieczne, ponieważ kod oryginalnie odnoszący się do parametru makra musi zostać przepisany, aby odnosił się do parametru metody w wygenerowanej klasie.

Klasa \texttt{ReplaceRefs} udostępnia \texttt{TreeMap}, który podczas przejścia po AST podmienia referencje do wskazanych symboli na podane termy.
Umożliwia to tzw. re-owning — przeniesienie fragmentów kodu między różnymi właścicielami symboli bez ręcznego przepisywania drzew (por. \ref{subsubsec:core-replacerefs}).

\subsection{Ekstrakcja i kompilacja wzorców}\label{subsec:ekstrakcja-i-kompilacja-wzorcow}

\subsubsection{Funkcja extractSimple}\label{subsubsec:funkcja-extractsimple}

Funkcja \texttt{extractSimple} implementuje logikę dopasowania różnych typów definicji tokenów:

\lstinputlisting[language=scala,caption={Funkcja extractSimple: dopasowywanie definicji tokenów},label={lst:lexer-08-extract-simple}, linerange={80-85,99-108}]{../../src/alpaca/lexer/Lexer.scala}

Wykorzystuje ona dopasowanie wzorców w \textit{quotes} z ekstraktorem typów, umożliwiając rozróżnienie różnych wariantów definicji tokenów na poziomie typów.
Konstrukcja \texttt{type t <: ValidName} w wzorcu wiąże parametr typu do zmiennej wzorca \texttt{t}, umożliwiając jego późniejsze wykorzystanie.

\subsection{Analiza wzorców: klasa CompileNameAndPattern}
\label{subsec:compile-name-pattern}

Klasa \texttt{CompileNameAndPattern} stanowi kluczowy komponent systemu analizy leksykalnej, odpowiedzialny za ekstrakcję i walidację wzorców tokenów podczas ekspansji makra.
Jej głównym zadaniem jest transformacja różnorodnych form wzorców występujących w definicjach DSL na ujednolicone struktury \texttt{TokenInfo}, które następnie są wykorzystywane do generacji finalnego kodu leksera.

Implementacja wykorzystuje rekurencyjne przetwarzanie drzewa AST z zastosowaniem optymalizacji rekurencji ogonowej (\texttt{@tailrec}), co zapewnia efektywność działania nawet dla złożonych wzorców z wieloma alternatywami.

\subsection{Generacja klasy anonimowej}\label{subsec:generacja-klasy-anonimowej}

Anonimowa klasa implementująca \texttt{Tokenization[Ctx]} jest konstruowana programatycznie: (1) tworzymy symbol klasy przez \texttt{Symbol.newClass} wraz z listą deklaracji pól i typów; (2) budujemy ciało klasy (\texttt{ClassDef}) zawierające \texttt{ValDef} dla każdego zdefiniowanego tokena oraz pola \texttt{tokens} i \texttt{byName}; (3) określamy rodzica przez wywołanie konstruktora \texttt{Tokenization[Ctx]} z wymaganymi zależnościami; (4) instancjonujemy klasę i nadajemy jej typ zrafi­nowany przez kolejne \texttt{Refinement} odpowiadające polom-tokenom.

\lstinputlisting[language=scala,caption={Generacja i instancjowanie anonimowej klasy \texttt{Tokenization[Ctx]} z typem rafinowanym},label={lst:lexer-14-anon-class}, linerange={192-264}]{../../src/alpaca/lexer/Lexer.scala}

\subsubsection{Typy refinement}\label{subsubsec:typy-refinement}

Zwracany typ jest stopniowo rafinowany dla każdego tokena:

\lstinputlisting[language=scala,caption={Rafinowanie typu wynikowego o pola tokenów},label={lst:lexer-15-refinements}, linerange={253-258}]{../../src/alpaca/lexer/Lexer.scala}

\texttt{Refinement(tpe, name, memberType)} tworzy typ refinement dodający członka o podanej nazwie i typie do bazowego typu.
Pozwala to kompilatorowi śledzić, że zwrócony obiekt ma pola odpowiadające poszczególnym tokenom, umożliwiając dostęp do nich z pełnym wsparciem systemu typów.

\subsection{Walidacja i obsługa błędów}\label{subsec:walidacja-i-obsuga-bedow}

\subsubsection{Walidacja wzorców regularnych}\label{subsubsec:walidacja-wzorcow-regularnych}

System wykorzystuje pomocniczą klasę \texttt{RegexChecker} do walidacji wzorców:

Poniższy mechanizm sprawdza poprawność składni wyrażeń regularnych już w czasie kompilacji i raportuje błędy z dokładną lokalizacją wzorca.
Metoda \texttt{report.errorAndAbort} jest częścią API kompilatora do raportowania błędów w czasie kompilacji.
Przerwanie kompilacji w przypadku niepoprawnych wzorców zapewnia, że błędy konfiguracji są wykrywane możliwie wcześnie.

\subsubsection{Obsługa nieobsługiwanych konstrukcji}\label{subsubsec:obsuga-nieobsugiwanych-konstrukcji}

Kod jawnie sygnalizuje nieobsługiwane przypadki:
Obsługiwane są wyłącznie jasno zdefiniowane formy wzorców; w przypadku napotkania innej konstrukcji kompilacja jest przerywana z komunikatem zawierającym szczegóły AST, co upraszcza diagnostykę i utrzymuje zasadę fail-fast.
Ta strategia jest zgodna z zasadą fail-fast - lepiej jest wyraźnie odrzucić nieobsługiwane konstrukcje niż milcząco generować niepoprawny kod.
